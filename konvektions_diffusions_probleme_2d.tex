% \datum{07. Januar 2016}

\section{Konvektions-Diffusions-Probleme in 2D}
\label{sec:konv-diff-probl}

Wir betrachten im Folgenden das elliptische Randwertproblem
\begin{align*}
  Lu \coloneqq - \epsilon \Delta u + b \cdot \nabla u + cu = & f \In \Omega \subseteq \R^{2},\\
u =& g \quad \text{auf } \Gamma \coloneqq \partial \Omega. 
\end{align*}
Abhängig vom Konvektionskoeffizienten $b$ und der äußeren Normalen $n$ kann der Rand $\Gamma$ in drei offene Teilmengen unterteilt werden:
\begin{enumerate}
\item Einströmrand $\Gamma_{-} \coloneqq \set{x \in \Gamma: \, b(x) \cdot n(x) < 0}$, 
\item Ausströmrand $\Gamma_{+} \coloneqq \set{x \in \Gamma: \, b(x) \cdot n(x) > 0}$, 
\item charakteristischer Rand $\Gamma_{0} \coloneqq \set{x \in \Gamma: \, b(x) \cdot n(x) = 0}$, Fluss parallel zum Rand, 
\item und $\overline{\Gamma_{-} \cup \Gamma_{+} \cup \Gamma_{0}} = \Gamma$. 
\end{enumerate}
Das reduzierte Problem zu $Lu = f$ ist
\begin{align*}
  b \cdot \nabla u_{0} + c u_{0}  = f \In \Omega. 
\end{align*}
Diese PDGL ist von erster Ordnung, kann also pro Charakteristik $b(x), x \in \Gamma$ nur einen Randwert erfüllen:
\begin{align*}
  u_{0} = g \quad \text{auf } \Gamma_{-}. 
\end{align*}
Hierbei nehmen wir an, dass die Charakteristiken $\Omega$ durch $\Gamma_{-}$ betreten und durch $\Gamma_{+}$ verlassen (schneiden dürfen sie sich auch nicht). 

Damit erfüllt $u_{0}$ auf $\Gamma\setminus\Gamma_{-}$ im Allgemeinen nicht die Randbedingungen und muss durch die Grenzschichtkorrekturen ergänzt werden:
\begin{enumerate}
\item an $\Gamma_{+}$ mit regulären/exponentiellen Grenzschichten, 
\item an $\Gamma_{0}$ mit charakteristischen/parabolischen Grenzschichten. 
\end{enumerate}
Einheitskreis 'absolut langweilig', deshalb Einheitsquadrat.
\subsection{Analytische Eigenschaften der Lösung}
\label{sec:analyt-eigensch-der}

Für einen genauere Untersuchung der Lösung beschränken wir uns auf $\Omega = (0, 1)^{2}$, welches auf der einen Seite geometrisch einfach ist, auf der anderen Seite allerdings keinen glatten Rand besitzt aufgrund der Ecken. Für konstante $b = (b_{1}, b_{2})^{T}$ gilt dann mit
\begin{align*}
  - \epsilon \Delta u - b \cdot \nabla u + cu = & f 
\end{align*}
\begin{enumerate}
\item $b_{1} > 0, b_{2} > 0$: $\Gamma_{+} = (\set 0 \times [0, 1]) \cup([0, 1] \times\set 0)$, $\Gamma_{0} = \emptyset$, also existieren zwei exponentielle Grenzschichten.
\item $b_{1} > 0, b_{2} = 0$: $\Gamma_{+} = (\set 0 \times [0, 1])$, $\Gamma_{0} =  [0, 1] \times \set{0, 1}$, also existieren eine exponentielle und zwei charakteristische Grenzschichten.
\end{enumerate}
Dies sind unsere beiden Modellprobleme.

\paragraph{(a) Reguläre Grenzschichten}
\label{sec:a-regul-grenzsch}

Allgemein: Für die Korrekturen, die wir auf $\Gamma_{+}$ brauchen, betrachten wir die lokalen Koordinaten $(\rho, \phi)$ mit
\begin{align*}
  \phi \cdot n = 0, \quad \rho = -n. 
\end{align*}
Das Differentialgleichunsproblem entlang von Geraden durch einen Punkt $\bar x \in \Gamma_{+}$ parallel zu $n$ lautet dann
\begin{align*}
  - \epsilon v_{\rho\rho} - b(\bar x)\cdot n(\bar x) v_{\rho} + c(\bar x) v &= 0, \\
v (\rho = 0) &= (g - u_{0})(\bar x). 
\end{align*}
Diesen Problem haben wir schon betrachtet, es hat die Lösung
\begin{align*}
  v(x) = (g - u_{0})(\bar x) \exp\(- \frac 1 \epsilon  b(\bar x) \cdot n(\bar x) \rho\). 
\end{align*}
Mit Hilfe einer asymptotischen Entwicklung, vergleiche Kapitel \ref{sec:asympt-entw-fur}, und einer Restgliedabschätzung erhalten wir eine Lösungszerlegung. Im Gegensatz zum 1d Fall benötigen wir aufgrund der Ecken 
%!!!!!!!!!!!!!!!!!!!!!!!!!!!!!!
\markdef{Kompatibilitätsbedingungen} bezüglich $f$. In \cite{LS_JMAA} finden wir
\begin{align*}
  f(0, 0) =   f(1, 0) =   f(0, 1) =   f(1, 1) = 0,\\
  \partial_{y} \( \frac f {b_{1}}\) (1, 1) =   \partial_{x} \( \frac f {b_{2}}\) (1, 1),\\
  \partial_{y} \( \partial_{x} \(\frac f {b_{1}}\) - D_{0} \( \frac f {b_{1}}\)\) (1, 1) =   \partial^{2}_{x} \( \frac f {b_{2}}\) (1, 1),\\
  \partial_{y} \( \partial^{2}_{x} \(\frac f {b_{1}}\) - D_{0} \( \partial_{x} \(\frac f {b_{1}} \) - D_{0} \(\frac f {b_{1}} \)\) - 2 D_{1}\(\frac f {b_{1}} \)\) (1, 1) =   \partial^{3}_{x} \( \frac f {b_{2}}\) (1, 1),\\
(b_{2} \partial_{x}^{2} \(\frac f {b_{1}} \))(1, 1) = (b_{2} \partial_{x}^{2} \(\frac f {b_{1}} \))(1, 1), 
\end{align*}
wobei
\begin{align*}
  D_{0}v &\coloneqq - \partial_{y} v \cdot \frac{b_{2}}{b_{1}},\\
  D_{1}v &\coloneqq - \partial_{y} v \partial_{x} \(\frac {b_{2}}{b_{1}}\) - v \partial_{x} \(\frac {c}{b_{1}}\). 
\end{align*}
Die zweite Bedingung bezieht sich auf die Ecke 'aus der Information in das Gebiet getragen wird'. Ist diese Information gestört/schlecht, dann wird das auch in das gesamte $\Omega$ getragen. 
\begin{satz}\label{thm:7-1}
  Es seien $f \in C^{4, \alpha}(\bar \Omega)$ mit $\alpha \in (0, 1)$, $n \geq 2$ und $f$ erfülle die obigen Kompatibilitätsbedingungen. Sollte $n \geq 4$ gelten, so sei zusätzlich
  \begin{align*}
    \partial_{x} b_{2}(0, 0) =    \partial_{y} b_{1}(0, 0). 
  \end{align*}
Dann besitzt das Randwertproblem
\begin{align*}
  - \epsilon \Delta u - b \cdot \nabla u + cu &= f \In \Omega = (0, 1)^{2}\\
  u &= 0 \quad \text{auf } \Gamma = \partial \Omega
\end{align*}
eine Lösung $u \in C^{3, \alpha}(\bar \Omega)$ mit der Zerlegung $u = v + w_{1} + w_{2} + w_{12}$ und $b = (b_{1}, b_{2}) \geq (\beta_{1}, \beta_{2}) > 0$ mit
\begin{align*}
  \nnorm v _{C^{2}(\bar \Omega)} + \epsilon^{\alpha} \nnorm v_{C^{2, \alpha}(\bar \Omega)} \leq C
\end{align*}
und für $x, y \in [0, 1]$
\begin{align*}
  \norm{\partial_{x}^{j} \partial_{y}^{i} w_{1}(x, y)} &\leq C \cdot \epsilon^{-i} e^{- \frac {\beta_{1} x}{\epsilon}}, \quad 0 \leq i \leq n, 0 \leq j\leq 2, \\
  \norm{\partial_{x}^{i} \partial_{y}^{j} w_{2}(x, y)} &\leq C \cdot \epsilon^{-j} e^{- \frac {\beta_{2} y}{\epsilon}}, \quad 0 \leq i \leq 2, 0 \leq j\leq n, \\
  \norm{\partial_{x}^{i} \partial_{y}^{j} w_{12}(x, y)} &\leq C \cdot \epsilon^{-(i+j)} e^{- \frac {\beta_{1} x}{\epsilon}}e^{- \frac {\beta_{2} y}{\epsilon}}, \quad 0 \leq i \leq n, 0 \leq j\leq n. 
\end{align*}
Für höhere Ableitungen werden mehr Glätte von $f$ und weitere Kompatibilitätsbedingungen benötigt.
\end{satz}
\paragraph{(b) Charakteristische Grenzschichten}
\label{sec:b-char-grenzsch}
Hier führt die lokale Transformation auf eine parabolische partielle Differentialgleichung: 
\begin{align*}
  - \epsilon v_{\rho\rho} + v_{\rho} = 0
\implies \quad v \approx \exp\(- \frac \rho {\sqrt \epsilon}\) \approx \exp\(- \frac {\rho^{2}} \epsilon\). 
\end{align*}
Der Korrekturterm ist
\begin{align*}
  v = (g - u_{0})(\bar x) \exp\(- \frac \rho {\sqrt \epsilon}\).  
\end{align*}
Die Analysis für eine Lösungszerlegung ist hier deutlich schwieriger, als im Fall der exponentiellen Grenzschichten. In \cite{KS_JDE}und \cite{KS_AML} finden wir folgende Aussage:
\begin{satz}\label{thm:7-2}
  Angenommen, $b_{1} > 0$ und $c > 0$ sind konstant. Es sei $f \in C^{8}(\bar \Omega)$ mit der Kompatibilitätsbedingung
  \begin{align*}
  f(0, 0) =   f(1, 0) =   f(0, 1) =   f(1, 1) = 0.
  \end{align*}
Dann kann die Lösung $u$ von 
\begin{align*}
  - \epsilon \Delta u - b_{1} u_{x} + cu &= f \In \Omega = (0, 1)^{2},\\
  u &= 0 \quad \text{auf } \Gamma = \partial \Omega
\end{align*}
zerlegt werden in $u = v + w_{1}+ w_{2} + w_{12}$, wobei für alle $x, y \in [0, 1]$ und $0 \leq i + j \leq 2$ die punktweisen Abschätzungen
\begin{align*}
  \norm{\partial_{x}^{i}\partial_{y}^{j}v(x, y)} &\leq C,\\
  \norm{\partial_{x}^{i}\partial_{y}^{j}w_{1}(x, y)} &\leq C \epsilon^{-i} e^{ - \frac {b_{1}x}{\epsilon}},\\
  \norm{\partial_{x}^{i}\partial_{y}^{j}w_{2}(x, y)} &\leq C \epsilon^{-\frac j2} \(e^{ - \frac y {\sqrt \epsilon}} + e^{ - \frac {(1 - y)}{\sqrt\epsilon}} \),\\
  \norm{\partial_{x}^{i}\partial_{y}^{j}w_{12}(x, y)} &\leq C \epsilon^{-\frac {(i+j)}2} e^{ -\frac {b_{1}x}{\epsilon}} \(e^{ - \frac y {\sqrt \epsilon}} + e^{ - \frac {(1 - y)}{\sqrt\epsilon}} \)
\end{align*}
sowie für $0 \leq i + j \leq 3$ die $L_{2}$-Abschätzungen
\begin{align*}
  \nnorm{\partial_{x}^{i}\partial_{y}^{j} v}_{L_{2}} &\leq C,\\
  \nnorm{\partial_{x}^{i}\partial_{y}^{j} w_{1}}_{L_{2}} &\leq C \epsilon^{ -i + \frac 12},\\
  \nnorm{\partial_{x}^{i}\partial_{y}^{j} w_{2}}_{L_{2}} &\leq C \epsilon^{ -\frac j2 + \frac 14} \qquad \text{ außer }   \nnorm{\partial_{x}^{2}\partial_{y} w_{2}}_{L_{2}} \leq C \epsilon^{ -\frac 12},\\
  \nnorm{\partial_{x}^{i}\partial_{y}^{j} w_{12}}_{L_{2}} &\leq C \epsilon^{ -i -\frac j2 + \frac 34}.
\end{align*}
Diese Lösungszerlegung wird auch für variable Koeffizienten sowie unter weiteren Kompatibilitätsbedingungen und Glattheitsvoraussetzungen für höhere Ableitungen angenommen. 
\end{satz}

% \datum{08. Januar 2016}

\subsection{FEM-Analysis auf angepassten Gittern}
\label{sec:fem-analysis-auf}

\paragraph{Plan}
\label{sec:plan}
Wir beschäftigen uns mit Lösungszerlegungen, die wir für angepasste Gitter brauchen und beides verwenden wir dann für Interpolationsfehlerabschätzungen. Die angepassten Gitter führen uns auch zu Galerkin-FEM mit Stabilisierungen, das wiederung führt zu G-Orthogonalität (evtl. schwach) und Koerzitivität (evtl. inf-sup). Das, und die Interpolationsfehlerabschätzungen liefern uns Fehlerabschätzungen. 
\smallskip

Betrachten wir dazu das Modellproblem mit regulären Grenzschichten, das heißt
\begin{align*}
  - \epsilon \Delta u - b \cdot \nabla u + c u = f &\In (0, 1)^{2}\\
  u = 0 &\Auf \Gamma
\end{align*}
mit $b = (b_{1}, b_{2})^{T}\geq (\beta_{1}, b_{2})^{T}> 0$ und $c + \frac 12 \div b \geq \gamma > 0$. 
Nach \ref{thm:7-1} besitzt $u$ drei verschiedene Grenzschichten
\begin{enumerate}
\item $w_{1}$ mit $\norm{w_{1}(x, y)}\leq C \exp \(- \beta_{1} \frac x \epsilon\)$ (Randgrenzschicht), 
\item $w_{2}$ mit $\norm{w_{2}(x, y)}\leq C \exp \(- \beta_{2} \frac y \epsilon\)$ (Randgrenzschicht), 
\item $w_{12}$ mit $\norm{w_{12}(x, y)}\leq C \exp \(- \beta_{1} \frac x \epsilon\)\exp \(- \beta_{2} \frac y \epsilon\)$ (Eckgrenzschicht).
\end{enumerate}
S-Typ-Gitter hängen von dem Übergangspunkten ab, in denen die Grenzschicht abgeklungen sind. Wir verlangen:
\begin{align*}
  \norm{w_{1}(\lambda_{x}, y)}&\leq C \exp \(- \beta_{1} \frac {\lambda_{x}} \epsilon\) \stackrel ! \leq C N^{-\sigma}\\
  \norm{w_{2}(x, \lambda_{y})}&\leq C \exp \(- \beta_{2} \frac {\lambda_{y}} \epsilon\) \stackrel ! \leq C N^{-\sigma}\\
\implies \quad \lambda_{x} &= \frac{\sigma\epsilon}{\beta_{1}} \ln N, \quad\lambda_{y} = \frac{\sigma\epsilon}{\beta_{2}} \ln N. 
\end{align*}
Wir definieren:
\begin{align*}
  \lambda_{x} \coloneqq\min \set{ \frac{\sigma\epsilon}{\beta_{1}} \ln N, \frac 12}, \quad \lambda_{y} \coloneqq \min\set{\frac{\sigma\epsilon}{\beta_{2}} \ln N, \frac 12}. 
\end{align*}
Dabei ist $\frac 12$ eine sinnvolle, aber willkürliche Grenze. Annahme:
\begin{align*}
  \epsilon \ln N \leq \frac 12 \sigma^{-1} \min\set{\beta_{1}, \beta_{2}} \,\iff\, \epsilon \ln N \leq C.
\end{align*}
Damit sind die Übergangspunkte festgelegt.
\begin{definition}\label{def:7-3}
  Ein \markdef{S-Typ-Gitter} für Probleme mit regulären Grenzschichten ist ein Tensorprodukt-Gitter zweier 1D-S-Typ-Gitter, das heißt
  \begin{align*}
    x_{i} &=
    \begin{cases}
      \frac{\sigma\epsilon}{\beta_{1}} \phi(i \cdot N^{-1}), & i = 0, \dots, \frac N2, \\
      \lambda_{x} + (2i\cdot N^{-1} - 1)(1 - \lambda_{x}), & i = N\cdot2^{-1}, \dots, N, 
    \end{cases}\\
    y_{j} &=
    \begin{cases}
      \frac{\sigma\epsilon}{\beta_{2}} \phi(j \cdot N^{-1}), & j = 0, \dots, \frac N2, \\
      \lambda_{y} + (2j\cdot N^{-1} - 1)(1 - \lambda_{y}), & y = N\cdot2^{-1}, \dots, N, 
    \end{cases}
  \end{align*}
und Gitterzellen
\begin{align*}
  \psi_{ij} = (x_{i-1}, x_{i}) \times (y_{j-1}, y_{j}), \quad i, j = 1, \dots, N. 
\end{align*}
\begin{figure}[ht!]
  \centering
   \begin{tikzpicture}
    
 \newcommand*{\xMin}{0}%
 \newcommand*{\xMax}{6}%
 \newcommand*{\yMin}{0}%
 \newcommand*{\yMax}{6}%
 \def\Ncoarse {5};
 \def\Nfine {10};
 \def\lambday  {0.3};
 \def\lambdax {0.3};
 \def\h { (\xMax -\lambdax)/\Ncoarse};
 \def\picscale {0.00125};
 \pgfmathsetmacro\newvalue{\lambdax + \h * \picscale * \Nfine * \Nfine * \Nfine};

 \begin{scope}
   \foreach \i in {0,...,\Nfine} {
     \pgfmathsetmacro\xcoord{(\lambdax + \h*\i*\i*\i) * \picscale};
     \draw [very thin,gray] (\xcoord, \yMin) -- (\xcoord, \yMax);% node [below] at (\xcoord,\yMin) {$\i$};
   }
   \foreach \i in {0,...,\Nfine} {
     \pgfmathsetmacro\ycoord{(\lambday + \h*\i*\i*\i) * \picscale};
     \draw [very thin,gray] (\xMin,\ycoord) -- (\xMax,\ycoord);% node [left] at (\xMin,\ycoord) {$\i$};
   }

   \foreach \i in {0,...,\Ncoarse} {
     \pgfmathsetmacro\xcoord{\newvalue * (\i * 0.2) + \xMax * (\Ncoarse - \i) * 0.2};
     \draw [very thin,gray] (\xcoord, \yMin) -- (\xcoord, \yMax);% node [below] at (\xcoord,\yMin) {$\i$};
   }

   \foreach \i in {0,...,\Ncoarse} {
     \pgfmathsetmacro\ycoord{(\newvalue * (\i * 0.2)) + (\yMax * (\Ncoarse - \i) * 0.2)};
     \draw [very thin,gray] (\xMin, \ycoord) -- (\xMax, \ycoord);
   }
   \draw[thick,black] (\newvalue, \yMin) -- (\newvalue, \yMax) node [below] at (\newvalue, \yMin) {$\lambda_{x}$};
   \draw[thick,black] (\xMin, \newvalue) -- (\xMax, \newvalue) node [left] at (\xMin, \newvalue) {$\lambda_{y}$};
   \node [below left] at (0, 0) {$0$};
   \node [below] at (\xMax, 0) {$1$};
   \node [left] at (0, \yMax) {$1$};
 \end{scope}
 \begin{scope}[right=8cm]
   \draw[thin,gray] (0, \yMin) -- (0, \yMax);
   \draw[thin,gray] (\xMin, 0) -- (\xMax, 0);
   \draw[thin,gray] (\xMax, \yMin) -- (\xMax, \yMax);
   \draw[thin,gray] (\xMin, \yMax) -- (\xMax, \yMax);
   \draw[thick,black] (\newvalue, \yMin) -- (\newvalue, \yMax) node [below] at (\newvalue, \yMin) {$\lambda_{x}$};
   \draw[thick,black] (\xMin, \newvalue) -- (\xMax, \newvalue) node [left] at (\xMin, \newvalue) {$\lambda_{y}$};
   \node[below left] at (0, 0) {$0$};
   \node[below] at (\xMax, 0) {$1$};
   \node[left] at (0, \yMax) {$1$};
   \node[anchor=center] at (\newvalue * 0.5, \newvalue * 0.5) {$\Omega_{22}$};
   \node[anchor=center] at (\newvalue * 0.5, \yMax * 0.5 + \newvalue * 0.5) {$\Omega_{12}$};
   \node[anchor=center] at (\xMax * 0.5 + \newvalue * 0.5, \newvalue * 0.5) {$\Omega_{21}$};
   \node[anchor=center] at (\xMax * 0.5 + \newvalue * 0.5, \yMax * 0.5 + \newvalue * 0.5) {$\Omega_{11}$};
 \end{scope}
\end{tikzpicture}

\caption{Bezeichnung in 2D-Gittern}
\label{fig:2d-grid}
\end{figure}

\begin{enumerate}
\item in $\Omega_{11}$: isotrope Zellen, das heißt
  \begin{align*}
    \max\set{ \frac{x_{i}, x_{i-1}}{y_{j}, y_{j-1}}, \frac{y_{j}, y_{j-1}}{x_{i}, x_{i-1}}}\leq C
  \end{align*}
unabhängig von $\epsilon$. 'Das Verhältnis der Seiten ist beschränkt.'
\item in $\Omega_{12} \cup \Omega_{21}$: anisotrope Zellen, das heißt
  \begin{align*}
    \max\set{ \frac{x_{i}, x_{i-1}}{y_{j}, y_{j-1}}, \frac{y_{j}, y_{j-1}}{x_{i}, x_{i-1}}}\sim C \epsilon^{-1},
  \end{align*}
Stichwort \markdef{'aspect ratio'}. 
\end{enumerate}
\end{definition}
\begin{bemerkung*}
  Sei der diskrete Raum gegeben durch stückweise bilineare Elemente $Q_{1}$. Standard-Abschätzung des Interpolationsfehlers mit Lemma von Bramble-Hilbert liefert zum Beispiel
  \begin{align*}
    \nnorm{(u - u^{I} x}_{L_{2}(\psi_{ij})}&\leq C\(h_{x} + \frac{h_{y}^{2}}{h_{x}}\)\cdot \norm u_{H^{2}(\psi_{ij})}
  \end{align*}
Auf anisotropen Gittern ist $\frac{h_{y}^{2}}{h_{x}}$ problematisch, da ausufernd. Außerdem liefert die Multiplikation, dass die Kleinheit der Zelle in eine Richtung Ableitungen in diese Richtung nicht auffangen kann. 
Das ist schlecht. Wir benötigen also eine genauere Fehleranalysis. 
\end{bemerkung*}

\subsection{Interpolationsfehlerabschätzungen}
\label{sec:interp}

Wir folgen dazu der Theorie nach Apel \cite{A_BOOK}. Wir nutzen trotz der Ausführungen oben eine Version von Bramble-Hilbert, die funktioniert:
\begin{lemma}\label{lem:7-4} \cite{BH_NM}
  
Es sei $R$ ein Rechteck, $l \geq 1$, $p \in [1, \infty)$ und $u \in W^{l}_{p}(R)$. Außerdem sei $K$ eine Menge von Multiindizes, sodass die Menge geschachtelt werden kann:
\begin{align*}
  \set{(l, 0), (0, l)} \subset K \subset \set{ \alpha: \, \norm \alpha = l}
\end{align*}
gilt. Bezeichne $\cP_{K}$ die Menge der Polynome $w$ mit der Eigenschaft, dass sie unter allen Ableitungen $\alpha \in K$ verschwinden, also
\begin{align*}
  D^{\alpha}w = 0. 
\end{align*}
Dann existieren $C_{1}, C_{2} > 0$ mit
\begin{align*}
  C_{1} \sum_{\alpha \in K} \nnorm{D^{\alpha} u}_{L_{p}(R)}& \leq \inf_{w \in \cP_{K}} \nnorm{u - w}_{W_{p}^{l}(R)} \leq  C_{2} \sum_{\alpha \in K} \nnorm{D^{\alpha} u}_{L_{p}(R)}
\end{align*}
('so etwas wie eine Semi-Norm, ist aber keine'). 
\end{lemma}
\begin{lemma}\label{lem:7-5}
  Es sei $R$ ein Rechteck und $\gamma$ ein Multiindex mit der Ordnung $m = \norm \gamma$. Weiterhin sei $u \in L_{1}(R)$ (betragsmäßig integrierbar) mit
  \begin{align*}
    D^{\gamma}(u) \in W_{p}^{l-m}(R)
  \end{align*}
mit $0 \leq m\leq l$ und $p \in [1, \infty)$ für ein geeignetes $l$. Dann exisitert ein Polynom $w \in  Q_{l-1}(R)$ mit
\begin{align*}
  \nnorm{D^{\gamma} (u-w)}_{W_{p}^{l-m}(R)} \leq C [D^{\gamma}u]_{W_{p}^{l-m}(R)} = C \( \nnorm{\partial_{x}^{\,l-m} D^{\gamma} u}_{L_{p}(R)} + \nnorm{\partial_{y}^{\,l-m} D^{\gamma} u}_{L_{p}(R)} \).
\end{align*}
\end{lemma}
\begin{beweis}Fallunterscheidung:
  \begin{enumerate}
  \item $m = 0 \iff \gamma =(0, 0)$: Es sei $K = \set{\alpha = l\cdot \beta, \norm \beta = 1} =   \set{(l, 0), (0, l)} $. Mit Lemma \ref{lem:7-4} folgt die Aussage. 
\item $m \geq 1$: Es sei $v = D^{\gamma} u$. Dann liefert Lemma \ref{lem:7-5} für $\gamma = (0, 0)$, $l \coloneqq l-m$ die Existenz von $w_{0} \in Q_{l-m-1}(R)$ mit
\begin{align*}
  \nnorm{v - w_{0}}_{W_{p}^{l-m}(R)} \leq C [v]_{W_{p}^{l-m}(R)}. 
\end{align*}
Mit $w \in Q_{l-1}(R)$ mit $D^{\gamma}w = w_{0}$ folgt dann
\begin{align*}
  \nnorm{D^{\gamma}(u - w)}_{W_{p}^{l-m}(R)} \leq C [D^{\gamma}_{u}]_{W_{p}^{l-m}(R)}. 
\end{align*}
\end{enumerate}
\end{beweis}

\begin{lemma}\label{lem:7-6}
  Es sei $R$ ein Rechteck und $I:C(R) \to Q_{k}(R)$ ein linearer (Interpolations-)Operator. Mit den festen Zahlen $m \in \N$, $p\in [1, \infty)$ und $q \in [1, \infty]$, die die Bedingungen 
  \begin{align*}
    &0 \leq m < k+1, \\
    &W_{p}^{k+1-m}(R) \emb L_{q}(R), 
  \end{align*}
erfüllen, betrachten wir einen Multiindex $\gamma$ mit $\norm\gamma = m$ und definieren $j = \dim D^{\gamma} Q_{k}(R)$. Angenommen, es existieren $j$ lineare Funktionale $F_{i}$ mit den drei Eigenschaften
\begin{enumerate}
\item $F_{i} \in (W_{p}^{k+1-m} (R))$ (stetiges, lineares Funktional), 'einfach', 
\item $F_{i}(D^{\gamma}(u- Iu)) = 0$, $i = 1, \dots, j$, Konsistenz, 'schwieriger', 
\item $w \in Q_{k}(R), \, F^{i}(D^{\gamma}w) = 0, \, i = 1, \dots, j \, \implies\, D^{\gamma}w = 0$, Unisolvenz auf $D^{\gamma}Q_{k}(R)$, 
\end{enumerate}
dann gilt für alle $u \in C(R)$ mit $D^{\gamma}u \in W_{p}^{k+1-m}(R)$
\begin{align*}
  \nnorm{D^{\gamma}(u-Iu)}_{ L_{q}(R)}\leq C [D^{\gamma}u]_{W_{p}^{k+1-m}}(R). 
\end{align*}
\end{lemma}

\begin{beweis}
  Es sei $v \in Q_{k}(R)$. Dann gilt mit der Dreiecksungleichung
  \begin{align*}
    \nnorm{D^{\gamma}(u-Iu)}_{ L_{q}(R)}&\leq \nnorm{D^{\gamma}(u-v)}_{ L_{q}(R)} + \nnorm{D^{\gamma}(v-Iu)}_{ L_{q}(R)}. 
  \end{align*}
Mit $v - Iu \in Q_{k}(R)$ ist $D^{\gamma}(v - Iu)\in D^{\gamma}Q_{k}(R)$. In endlichdimensionalen Vektorräumen wie $Q_{k}(R)$ sind alle Normen äquivalent. Also gilt, auch wegen \ref{num:iii} und dann \ref{num:ii} und dann \ref{num:i},
\begin{align*}
  \nnorm{D^{\gamma}(v-Iu)}_{ L_{q}(R)} &\leq C_{1} \sum_{i = 1}^{j} \norm{F_{j}(v - Iu)}\\
&= C_{1} \sum_{i = 1}^{j} \norm{F_{i}(v - u)}\\
&\leq C_{1} C_{2} \nnorm{D^{\gamma}(v - u)}_{W_{p}^{k+1-m}(R)}. 
\end{align*}
Somit folgt mit der Einbettung und im zweiten Schritt mit Lemma \ref{lem:7-5}
\begin{align*}
    \nnorm{D^{\gamma}(u-Iu)}_{ L_{q}(R)}&\leq C \nnorm{D^{\gamma}(v-u)}_{W_{p}^{k+1-m}(R)}\\
    &\leq C [D^{\gamma}u]_{W_{p}^{k+1-m}(R)}. 
\end{align*}
\end{beweis}
Nun erhalten wir eine anisotrope Fehlerabschätzung:
\begin{satz}\label{thm:7-7}
  Angenommen, $\tau$ ist ein Rechteck mit den Seiten parallel zu den Achsen und Seitenlängen $h = (h_{x}, h_{y})$. Sei $\gamma$ ein Multiindex mit der Ordnung $m = \norm \gamma$ und $u \in C(\tau)$ mit $D^{\gamma} u \in W_{p}^{k+1-m}(\tau)$ und $m \in \N_{0}$, $p \in [1, \infty]$, sowie $0 \leq m < k+1$. Existieren dann Funktionale $F_{i}$ gemäß Lemma \ref{lem:7-6}, so gilt die \markdef{anisotrope Interpolationsfehlerabschätzung}
  \begin{align*}
    \nnorm{D^{\gamma}(u - Iu)}_{L_{p}(\tau)} \leq C \sum_{\norm \alpha = 1} h^{k+1-m} \nnorm{D^{\gamma + (k+1-m)\alpha}u}_{L_{p}(\tau)}. 
  \end{align*}
\end{satz}

\begin{beweis}
Falluntescheidung:
\begin{enumerate}
\item $p < \infty$: Lemma \ref{lem:7-6} mit Transformation auf $R$. 
\item $p = \infty$: Der Beweis kann in $m = k$ ($[\cdot]_{W_{p}^{1}} = \norm \cdot _{W^{1}_{p}}$, Beweis genauso)
und $m \leq k-1$ ($p' < \infty$, Lemma \ref{lem:7-6}, $p' \to \infty$) geteilt werden.
\end{enumerate}
\end{beweis}
% \datum{14. Januar 2016}

Für die bilinearen Elemente und $u^{I}$ als Langrangeinterpolierende in den Eckpunkten des Rechtecks gilt dann im Speziellen:
\begin{itemize}
\item $\gamma = (0, 0)$:
  \begin{align*}
    \nnorm{u- u^{I}}_{L_{p}(\tau_{ij})} \leq C \( h_{i}^{2} \nnorm{u_{xx}}_{L_{p}(\tau_{ij})} + k_{j}^{2} \nnorm{u_{yy}}_{L_{p}(\tau_{ij})} \) 
  \end{align*}
\item $\gamma = (1, 0)$:
  \begin{align*}
    \nnorm{(u- u^{I})_{x}}_{L_{p}(\tau_{ij})} \leq C \( h_{i} \nnorm{u_{xx}}_{L_{p}(\tau_{ij})} + k_{j} \nnorm{u_{xy}}_{L_{p}(\tau_{ij})} \) 
  \end{align*}
\end{itemize}
Wir benötigen meist nur $p = 2$ bzw. $p = \infty$. Offen ist noch die Existenz der Funktionale für Lemma \ref{lem:7-6}. Sei dazu $R = (0, 1)^{2}$.

Sei $\gamma = (0, 0)$: $k = 1$, $m = 0$, dann ist $j = 4$, vier Funktionale (Punktauswertungen), die die drei Eigenschaften erfüllen:
\begin{align*}
  F_{1}(v) &= v(0, 0),\\
  F_{2}(v) &= v(1, 0),\\
  F_{3}(v) &= v(0, 1),\\
  F_{4}(v) &= v(1, 1).
\end{align*}
Sei $\gamma = (1, 0)$, $j = 2$, $m = 1$,  zwei Funktionale (die wieder die drei Eigenschaften erfüllen):
\begin{align*}
  \hat F_{1}(v) &= \int_{0}^{1} v(x, 0) dx\\
  \hat F_{2}(v) &= \int_{0}^{1} v(x, 1) dx
\end{align*}
Zur dritten Eigenschaft: $x \in Q_{1}$: $\partial_{x} w = a + by$:
\begin{align*}
  F_{1}(\partial_{x} w) &= a = 0\\
  F_{2}(\partial_{x} w) &= a + b = 0\\
\implies \quad a = 0, b &= 0, \\
\implies \quad \partial_{x} w &= 0.
\end{align*}
Bevor wir den Interpolationsfehler auf S-Typ-Gittern abschätzen, benötigen wir noch ein Hilfsresultat, Stabilitätsabschätzungen der Interpolation.
\begin{lemma} \label{lem:7-8}
  Für die bilineare Interpolierende gilt auf einem Rechteck $\tau$
  \begin{align*}
    \nnorm{u^{I}}_{L_{\infty}(\tau)} &\leq \nnorm u_{L_{\infty}(\tau)}, \\
    \nnorm{(u^{I})_{x}}_{L_{\infty}(\tau)} &\leq \nnorm {u_{x}}_{L_{\infty}(\tau)}. 
  \end{align*}
\end{lemma}
\begin{beweis}
  Es reicht aus, $\tau = (0, 1)^{2}$ zu betrachten. Mit den Basisfunktionen $\phi_{0}(x) =1 - x$ und $\phi_{1}(x) = x$ gilt
  \begin{align*}
    u^{I}(x, y) &= \(u_{00} \phi_{0}(x) + u_{10}\phi_{1}(x)\)\phi_{0}(y) + \(u_{01} \phi_{0}(x) + u_{11}\phi_{1}(x)\)\phi_{1}(y),\\
    u^{I}_{x}(x, y) &= \(u_{10}  - u_{00}\)\phi_{0}(x) + \(u_{11} - u_{01}\)\phi_{1}(x).
  \end{align*}
Da $\phi_{0}$ und $\phi_{1} \geq0$ in $\tau$ gilt
\begin{align*}
      \nnorm{u^{I}(x, y)}_{L_{\infty}(\tau)} &\leq \nnorm{\(\( \phi_{0}(x) + \phi_{1}(x)\) \phi_{0}(y) +  \( \phi_{0}(x) + \phi_{1}(x)\) \phi_{1}(y)\)\nnorm u_{L_{\infty}(\tau)} }_{L_{\infty}(\tau)}, \\
 &= \nnorm u_{L_{\infty}(\tau)}, \\
      \nnorm{u_{x}^{I}(x, y)}_{L_{\infty}(\tau)} &= \nnorm{ \int_{0}^{1}u_{x}(t, 0) dt \phi_{0}(y) + \int_{0}^{1} u_{x}(t, 1) dt \phi_{1}(y) }_{L_{\infty}(\tau)}\\
 &\leq \nnorm { (\phi_{0}(y) + \phi_{1}(y)) \nnorm{u_{x}}_{L_{\infty}(\tau)}}L_{\infty}(\tau)\\
& = \nnorm{u_{x}}_{L_{\infty}(\tau)}. 
\end{align*}
\end{beweis}
\begin{bemerkung*}
  \begin{enumerate}
  \item 
  Die Aussagen des Lemmas \ref{lem:7-8} gelten nicht, falls $L_{\infty}$ durch $L_{2}$ ersetzt wird:
  \begin{align*}
    u = \max \set{ 1 - \frac x \epsilon, 0}, \epsilon \in (0, 1) \implies \nnorm{u^{I}}^{2}_{L_{2}(\tau)} &= \int_{0}^{1}\int_{0}^{1} (1 - x)^{2} dxdy = \frac 13\\
\implies \nnorm{u}^{2}_{L_{2}(\tau)} &= \int_{0}^{1}\int_{0}^{\epsilon} (1 - \frac x \epsilon)^{2} dxdy = \frac \epsilon 3
  \end{align*}
Also gilt nicht gleichmäßig:
\begin{align*}
  \nnorm{u^{I}} _{L_{2}(\tau)} \leq \frac 1 {\sqrt \epsilon}   \nnorm{u} _{L_{2}(\tau)}. 
\end{align*}
Analog für $u_{x} = \max \set{1 - \frac x \epsilon, 0}$
\item Analoge Abschätzungen gelten für Elemente höherer Ordnung, dann mit
  \begin{align*}
    \nnorm{Iu}_{L_{\infty}(\tau)} \leq C \nnorm u _{L_{\infty}(\tau)}, \quad     \nnorm{(Iu)_{x}}_{L_{\infty}(\tau)} \leq C \nnorm {u_{x}} _{L_{\infty}(\tau)}. 
  \end{align*}
  \end{enumerate}
\end{bemerkung*}
Kommen wir nun zur Abschätzung auf S-Typ-Gittern:
\begin{satz} \label{thm:7-9}
  Auf einem S-Typ-Gitter mit $\sigma > 2$ gilt für $\epsilon \leq CN^{-1}$ ($\iff \leq CN^{-1}$)
  \begin{align*}
    \nnorm{u - u^{I}}_{L_{2}(\Omega_{11})} \leq CN^{-2},\\
    \nnorm{u - u^{I}}_{L_{2}(\Omega\setminus\Omega_{11})} \leq C(N^{-1} \max \norm{\psi'})^{2},\\
  \end{align*}
und
\begin{align*}
  \epsilon^{\frac 12} \nnorm{\nabla(u - u^{I})}_{L_{2}(\Omega_{11})} &\leq C N^{-1},\\ 
  \epsilon^{\frac 12} \nnorm{\nabla(u - u^{I})}_{L_{2}(\Omega\setminus\Omega_{11})} &\leq C (N^{-1} \max \norm{\psi'}), 
\end{align*}
also insgesamt
\begin{align*}
  \nnnorm{u-u^{I}}_{\epsilon} \leq C N^{-1} \max \norm{\psi'}. 
\end{align*}
\end{satz}
\begin{beweis}
  Der Beweis ist nicht schwierig, aber langwierig, da
  \begin{enumerate}
  \item drei Normen sind beteiligt ($\nnorm \cdot$, $\nnorm {(\cdot)_{x}}$,$\nnorm {(\cdot)_{y}}$)
  \item vier Lösungsbestandteile
  \item vier Teilgebiete von $\Omega$
  \item zwei Abschätzungstechniken (lokale anisotrope Abschätzungen (wenn die Ableitungen nichts kosten oder von den Gitterweiten aufgefangen werden) und Stabilitätsaussagen nutzen (wenn Lösungsbestandteil abgeklungen ist, eventuell inverse Ungleichung nutzen))
  \end{enumerate}
Hier nur $\nnorm{(\cdot)_{x}}$, da $\nnorm{(\cdot)_{y}}$ aufgrund der Symmetrie des Problems analog folgt und $\nnorm \cdot$ ähnlich funktioniert. Starten wir mit $\Omega_{11}$:
\begin{align*}
  \epsilon\nnorm{(v- v^{I})_{x}}^{2}_{L_{2}(\Omega_{11})}& = \epsilon \sum_{\tau_{ij}\subseteq \Omega_{11}} \nnorm{(v- v^{I})_{x}}^{2}_{L_{2}(\tau_{ij})} \\
  & \leq C \epsilon \sum_{\tau_{ij}\subseteq \Omega_{11}} N^{-2} \(\nnorm{v_{xx}}^{2}_{L_{2}(\tau_{ij})}  + \nnorm{v_{xy}}^{2}_{L_{2}(\tau_{ij})}\)\\
  & = C \epsilon N^{-2} \(\nnorm{v_{xx}}^{2}_{L_{2}(\Omega_{11})}  + \nnorm{v_{xy}}^{2}_{L_{2}(\Omega_{11})}\)\\
  &\leq C \epsilon N^{-2}
\end{align*}
die erste Ungleichung folgt mit lokaler anisotroper, die zweite mit Satz \ref{thm:7-1}.
Betrachte
\begin{align*}
  \epsilon \nnorm{((w_{1} + w_{12}) - (w_{1} + w_{12})^{I})_{x}}^{2}_{L_{2}(\Omega_{11})} &\leq 2\epsilon \( \nnorm{(w_{1} + w_{12})_{x}}^{2}_{L_{2}(\Omega_{11})} + \nnorm{((w_{1} + w_{12})^{I})_{x}}^{2}_{L_{2}(\Omega_{11})} \)\\
&\leq 2\epsilon \( \nnorm{(w_{1} + w_{12})_{x}}^{2}_{L_{2}(\Omega_{11})} + \nnorm{((w_{1} + w_{12})^{I})_{x}}^{2}_{L_{2}(\Omega_{11})} \)\\
&\leq C\epsilon^{-1} \int_{\Omega_{11}} e^{-2 \beta_{1} \frac x \epsilon} (1 + \underbrace{e^{-2 \beta_{2} \frac y \epsilon}}_{\leq1}) dxy + C \epsilon N^{2} \nnorm{(w_{1} + w_{12})^{I}}^{2}_{L_{2}(\Omega_{11})}\\
&\leq C N^{ - 2 \sigma} + C \epsilon N^{2} \( N^{-1} + \epsilon\) N^{-2\sigma}\\
&\leq C ( 1 + \epsilon N)^{2}  N^{- 2\sigma}. \\
%%%%%%%%%%%
  \epsilon^{\frac 12} \nnorm{(w_{2} - w_{2}^{I})_{x}}_{L_{2}(\Omega_{11})} &\leq \epsilon^{\frac 12} \nnorm{(w_{2})_{x}}_{L_{2}(\Omega_{11})} + \epsilon^{\frac 12} \nnorm{(w_{2}^{I})_{x}}_{L_{2}(\Omega_{11})}\\
&\leq \epsilon^{\frac 12} \nnorm{(w_{2})_{x}}_{L_{2}(\Omega_{11})} + \epsilon^{\frac 12} \nnorm{(w_{2})_{x}}_{L_{\infty}(\Omega_{11})}\\
&\leq C \epsilon^{\frac 12} N^{-\sigma}
\end{align*}
mit Stabilität. 

Grenzschichtbereich $\Omega_{12} = [0, \lambda_{x}] \times [\lambda_{y}, 1]$: 
\begin{align*}
    \epsilon \nnorm{((v + w_{1}) - (v + w_{1})^{I})_{x}}^{2}_{L_{2}(\Omega_{12})} &\leq C \epsilon \sum_{\tau_{ij} \subseteq \Omega_{12}} \( h_{i}^{2} \nnorm{(v + w_{1})_{xx}}^{2}_{L_{2}(\tau_{ij})}  + k_{j}^{2} \nnorm{(v + w_{1})_{xy}}^{2}_{L_{2}(\tau_{ij})}\)\\
&\leq C \epsilon N^{-2} \( \nnorm{v_{xx}}^{2}_{L_{2}(\Omega_{12})}  +\nnorm{v_{xy}}^{2}_{L_{2}(\Omega_{12})}\) + C \epsilon^{-1} (N^{-1} \max \norm{\psi'})^{2} \\
&\int_{0}^{\tau_{x}} e^{\(\frac 2 \sigma - 2\) \beta_{1} \frac x \epsilon} dx + C \epsilon^{-1} N^{-2} \int_{0}^{ \tau_{x}} e^{- 2 \beta_{1} \frac x \epsilon} dx\\
& \leq C\( \epsilon N^{-2}  + (N^{-1} \max\norm{\psi'})^{2} +  N^{-2}\)\\
& \leq C\( N^{-1} \max\norm{\psi'}\)^{2}\\
%%%%
  \epsilon^{\frac 12} \nnorm{(w_{2} - w_{2}^{I})_{x}}_{L_{2}(\Omega_{12})} &\leq C \epsilon^{\frac 12} \nnorm{(w_{2})_{x}}_{L_{\infty}(\Omega_{12})} \\
& \leq C \epsilon^{\frac 12} N^{- \sigma}. \\
%%%%%
  \epsilon \nnorm{(w_{12} - w_{12}^{I})_{x}}^{2}_{L_{2}(\Omega_{12})} &\leq C \epsilon \sum_{\tau_{ij} \subseteq \Omega_{12}} h_{i}^{2}  \nnorm{(w_{12})_{xx}}^{2}_{L_{2}(\tau_{ij})} + k_{j}^{2}\nnorm{(w_{12})_{xy}}^{2}_{L_{2}(\tau_{ij})} \\
 &\leq C( \epsilon^{-1} (N^{-1} \max\norm{\psi'})^{2} \int_{\lambda_{y}}^{1} \int_{0}^{\lambda_{x}}  e^{(2/\sigma - 2) \beta_{1} \frac x \epsilon} dx e^{- 2 \beta_{2} \frac y \epsilon} dy + \\
&\epsilon^{-3} N^{-2} \int_{\lambda_{y}}^{1} \int_{0}^{\lambda_{x}} e^{ - 2 \beta_{1}\frac x \epsilon}  e^{- 2 \beta_{2} \frac y \epsilon} dxdy )\\
& \leq C \( \epsilon (N^{-1} \max\norm{\psi'})^{2} N^{- 2 \sigma} + \epsilon^{-1} N^{-2}N^{-2\sigma}\)
\end{align*}
Das $\epsilon^{-1}$ macht es uns kaputt.. Zweiter Weg:
\begin{align*}
  \epsilon \nnorm{(w_{12} - w_{12}^{I})_{x}}^{2}_{L_{2}(\Omega_{12})} &\leq 2 \epsilon \( \nnorm{(w_{12})_{x}}^{2}_{L_{2}(\Omega_{12})} + \nnorm{(w_{12}^{I})_{x}}^{2}_{L_{2}(\Omega_{12})} \) \\
&\leq 2 \epsilon \( \nnorm{(w_{12})_{x}}^{2}_{L_{2}(\Omega_{12})} + \meas (\Omega_{12}) \nnorm{(w_{12}^{I})_{x}}^{2}_{L_{\infty}(\Omega_{12})} \) \\
&\leq C \(\epsilon N^{-2\sigma} + \epsilon^{2} \ln N \epsilon^{-2} N^{-2\sigma}\) \\
&\leq C  \ln N N^{-2 \sigma}
\end{align*}
Mit den gleichen Techniken kann auf $\Omega_{21}$ und $\Omega_{22}$ abgeschätzt werden. 
\end{beweis}
% \datum{15. Januar 2016}
\begin{lemma}\label{lem:7-10}
  Unter den Voraussetzungen von Satz \ref{thm:7-9} gilt zusätzlich
  \begin{align*}
    \nnorm{u - u^{I}}_{L_{\infty}(\Omega_{11})} \leq C N^{-2}
  \end{align*}
und
\begin{align*}
    \nnorm{u - u^{I}}_{L_{\infty}(\Omega\setminus\Omega_{11})} \leq C (N^{-1} \max\norm{\psi'})^{2}. 
\end{align*}
\end{lemma}
\begin{beweis}
  $L_{\infty}$ statt $L_{2}$, ansonsten die Techniken des Satzes \ref{thm:7-9}. 
\end{beweis}

Schauen wir uns nun den Fall der charakteristischen Grenzschichten an, das heißt
\begin{align*}
  - \epsilon \Delta u - b u_{x} + cu = f &\In \Omega = (0, 1)^{2} \\
u = 0& \Auf \partial \Omega 
\end{align*}
mit $b \geq \beta > 0$ und $ c + \frac 12 b_{x} \geq \gamma> 0$. 
Mit den gleichen Überlegungen wie im vorhergehenden Fall für reguläre Grenzschichten definieren wir uns die Übergangspunkte
\begin{align*}
  \lambda_{x} = \frac{\sigma\epsilon \ln N} \beta, \quad   \lambda_{y} = \sigma \sqrt\epsilon \ln N. 
\end{align*}
Diese Größen tauchten schon bei der 1d Konvektion-Diffusion und bei der 1d Reaktion-Diffusion auf. 
\begin{definition}\label{def:7-11}
  Ein \markdef{S-Typ-Gitter} für Probleme mit charakteristischen Grenzschichten ist ein Tensorprodukt zweier 1d-S-Typ-Gitter, das heißt
  \begin{align*}
    x_{i} &=
    \begin{cases}
      \sigma \frac \epsilon\beta \phi (i/N), & i = 0, \dots, N/2, \\
\lambda_{x} + (2 i/N - 1)(1 - \lambda_{x}), & i = N/2, \dots, N,
    \end{cases}
\\
    y_{j} &=
    \begin{cases}
      \sigma \epsilon^{\frac 12} \phi (2j/N), & j = 0, \dots, N/4, \\
      \lambda_{y} + (4 j/N - 1)(1/2 - \lambda_{y}), & j = N/4, \dots, 3N/4,\\
      1 - \sigma\epsilon^{\frac 12} \phi (2-2j/N), & j = 3N/4, \dots, N,
    \end{cases}
  \end{align*}
($N$ sei durch $4$ teilbar).

\begin{figure}[ht!]
  \centering
   \begin{tikzpicture}
    
 \newcommand*{\xMin}{0}%
 \newcommand*{\xMax}{6}%
 \newcommand*{\yMin}{0}%
 \newcommand*{\yMax}{6}%
 \def\Ncoarse {5};
 \def\Nfine {10};
 \def\lambday  {0.3};
 \def\lambdax {0.3};
 \def\h { (\xMax -\lambdax)/\Ncoarse};
 \def\picscale {0.00125};
 \pgfmathsetmacro\newvalue{\lambdax + \h * \picscale * \Nfine * \Nfine * \Nfine};

 \begin{scope}
   \foreach \i in {0,...,\Nfine} {
     \pgfmathsetmacro\xcoord{(\lambdax + \h*\i*\i*\i) * \picscale};
     \draw [very thin,gray] (\xcoord, \yMin) -- (\xcoord, \yMax);% node [below] at (\xcoord,\yMin) {$\i$};
   }
   \foreach \i in {0,...,\Nfine} {
     \pgfmathsetmacro\ycoord{(\lambday + \h*\i*\i*\i) * \picscale};
     \draw [very thin,gray] (\xMin,\ycoord) -- (\xMax,\ycoord);% node [left] at (\xMin,\ycoord) {$\i$};
   }

   \foreach \i in {0,...,\Nfine} {
     \pgfmathsetmacro\ycoord{(\lambday + \h*\i*\i*\i) * \picscale};
     \draw [very thin,gray] (\xMin,\yMax-\ycoord) -- (\xMax,\yMax-\ycoord);% node [left] at (\xMin,\ycoord) {$\i$};
   }


   \foreach \i in {0,...,\Ncoarse} {
     \pgfmathsetmacro\xcoord{\newvalue * (\i * 0.2) + \xMax * (\Ncoarse - \i) * 0.2};
     \draw [very thin,gray] (\xcoord, \yMin) -- (\xcoord, \yMax);% node [below] at (\xcoord,\yMin) {$\i$};
   }

   \foreach \i in {3,...,5} {
     \pgfmathsetmacro\ycoord{(\newvalue * (\i * 0.2)) + (\yMax * (\Ncoarse - \i) * 0.2)};
     \draw [very thin,gray] (\xMin, \ycoord) -- (\xMax, \ycoord);
   }
   \draw[thick,black] (\newvalue, \yMin) -- (\newvalue, \yMax) node [below] at (\newvalue, \yMin) {$\lambda_{x}$};
   \draw[thick,black] (\xMin, \newvalue) -- (\xMax, \newvalue) node [left] at (\xMin, \newvalue) {$\lambda_{y}$};
   \draw[thick,black] (\xMin, \yMax -\newvalue) -- (\xMax, \yMax -\newvalue) node [left] at (\xMin,\yMax- \newvalue) {$1- \lambda_{y}$};
   \node [below left] at (0, 0) {$0$};
   \node [below] at (\xMax, 0) {$1$};
   \node [left] at (0, \yMax) {$1$};
 \end{scope}


 \begin{scope}[right=8cm]
   \draw[thin,gray] (0, \yMin) -- (0, \yMax);
   \draw[thin,gray] (\xMin, 0) -- (\xMax, 0);
   \draw[thin,gray] (\xMax, \yMin) -- (\xMax, \yMax);
   \draw[thin,gray] (\xMin, \yMax) -- (\xMax, \yMax);
   \draw[thick,black] (\newvalue, \yMin) -- (\newvalue, \yMax) node [below] at (\newvalue, \yMin) {$\lambda_{x}$};
   \draw[thick,black] (\xMin, \newvalue) -- (\xMax, \newvalue) node [left] at (\xMin, \newvalue) {$\lambda_{y}$};
   \draw[thick,black] (\xMin,\yMax- \newvalue) -- (\xMax, \yMax-\newvalue) node [left] at (\xMin,\yMax- \newvalue) {$1-\lambda_{y}$};
   \node[below left] at (0, 0) {$0$};
   \node[below] at (\xMax, 0) {$1$};
   \node[left] at (0, \yMax) {$1$};
   \node[anchor=center] at (\newvalue * 0.5, \newvalue * 0.5) {$\Omega_{22}$};
   \node[anchor=center] at (\newvalue * 0.5, \xMax -\newvalue * 0.5) {$\Omega_{22}$};
   \node[anchor=center] at (\newvalue * 0.5, \yMax * 0.5 ) {$\Omega_{12}$};
   \node[anchor=center] at (\xMax * 0.5 + \newvalue * 0.5, \newvalue * 0.5) {$\Omega_{21}$};
   \node[anchor=center] at (\xMax * 0.5 + \newvalue * 0.5, \yMax-\newvalue * 0.5) {$\Omega_{21}$};
   \node[anchor=center] at (\xMax * 0.5 + \newvalue * 0.5, \yMax * 0.5 ) {$\Omega_{11}$};
 \end{scope}
\end{tikzpicture}

\caption{Bezeichnung in 2D-Gittern mit charakteristischer Grenzschicht}
\label{fig:2d-char-grid}
\end{figure}

\end{definition}
\begin{satz}\label{thm:7-12}
  Auf einem S-Typ-Gitter mit $\sigma>2$ gilt für $\epsilon^{\frac 12} < CN^{-1}$ für Probleme mit charakteristischen Grenzschichten
  \begin{align*}
    \nnorm{u - u^{I}}_{L_{\infty}(\Omega_{11})} \leq C N^{-2},
    \nnorm{u - u^{I}}_{L_{\infty}(\Omega\setminus\Omega_{11})} \leq C (N^{-1} \max\norm{\psi'})^{2}. 
  \end{align*}
und
\begin{align*}
  \epsilon^{\frac 12} \nnorm{\nabla(u - u^{I})}_{L_{2}(\Omega_{11})} \leq C N^{-1}  \epsilon^{\frac 12} \nnorm{\nabla(u - u^{I})}_{L_{2}(\Omega\setminus\Omega_{11})} \leq C N^{-1} \max \norm{\psi'}, 
\end{align*}
insgesamt
\begin{align*}
  \nnnorm{u - u^{I}}_{\epsilon} \leq C N^{-1} \max \norm{\psi'}. 
\end{align*}
\end{satz}
\begin{beweis}
  von uns...
\end{beweis}

% \datum{21. Januar 2016}

\subsection{Konvergenzanalysis für die Galerkin-FEM}
\label{sec:konv-fur-die}
Betrachten wir zuerst das Galerkin-Verfahren für Probleme mit exponentiellen Grenzschichten. Es sei dazu
\begin{align*}
  a(u, v) &= \epsilon(\nabla u, \nabla v) + (cu -b \cdot \nabla u, v)\\
  (u, v) &= \int_{\Omega} u\cdot v
\end{align*}
und
\begin{align*}
  f(v) = (f, v). 
\end{align*}
Die (eigentlich: eine) schwache Formulierung lautet dann: Gesucht ist $u \in H_{0}^{1}(\Omega)$, sodass
\begin{align*}
  a(u, v) = f(v) \quad \forall v \in H_{0}^{1}(\Omega). 
\end{align*}
Unter $C + \frac 12 \div b \geq \gamma > 0$ folgt
\begin{align*}
  a(v, v) &= \epsilon \nnorm{\nabla v}_{L_{2}}^{2} + (c v - b \cdot \nabla v, v)\\
  &= \epsilon \nnorm{\nabla v}_{L_{2}}^{2} + ((c + \frac 12 \div b)v, v)\\
& \geq \epsilon \nnorm{\nabla v}_{L_{2}}^{2} + \gamma\nnorm{v}_{L_{2}}^{2} \eqqcolon \nnnorm{v}^{2}_{\epsilon}
\end{align*}
also Koerzitivität bezüglich $\nnnorm \cdot _{\epsilon}$ unter Zuhilfenahme von
\begin{align*}
  - (b \cdot \nabla v, v) &= (\div b \cdot v, v) + (v, b \cdot \nabla v),\\
\implies \quad   - (b \cdot \nabla v, v) &= (\frac 12 \div b \cdot v, v).
\end{align*}
Die Galerkin-Formulierung für $V_{h} \subset H_{0}^{1}(\Omega)$ lautet dann: 

Gesucht ist $u^{N} \in V_{h}$, sodass
\begin{align*}
  a(u^{N}, v) = f(v)  \quad \forall v \in V_{h}. 
\end{align*}
Galerkin-Orthogonalität:
\begin{align*}
  a(u - u^{N}, v) = 0 \quad \forall v \in V_{h}. 
\end{align*}
\begin{satz}\label{thm:7-13}
  Auf einem S-Typ-Gittter mit $\sigma > 2$ gilt für $h \leq CN^{-1}$ und $N^{-1} \max \norm{\psi'} (\ln N)^{\frac 12} \leq C$ und bilineare Elemente
  \begin{align*}
    \nnnorm{u - u^{N}}_{\epsilon} \leq C N^{-1}\max\norm{\psi'}. 
  \end{align*}

\end{satz}
\begin{beweis}
  Zerlegen wir $u - u^{N} = \chi - \eta$ mit $\chi = u^{I} - u^{N}$, $\eta = u^{I} - u$, so folgt
  \begin{align*}
    \nnnorm{u - u^{N}}_{\epsilon} \leq \nnnorm \chi_{\epsilon}  + \nnnorm \eta_{\epsilon}
  \end{align*}
mit
\begin{align*}
\nnnorm{\eta}_{\epsilon} \leq C N^{-1} \max \norm{\psi'} 
\end{align*}
nach Satz \ref{thm:7-9}. Die Koerzitivität, Galerkin-Orthogonalität und partielle Integration liefern
\begin{align*}
  \nnnorm{\chi}_{\epsilon}^{2} &\leq a(\chi, \chi) = a (\eta, \chi) = \epsilon(\nabla \eta, \nabla \chi) + (\eta, b \cdot \nabla \chi) + ((c + \div b)\eta, \chi)\\
&\leq C \nnnorm \eta_{\epsilon} \nnnorm \chi_{\epsilon} + C\( \nnorm{\eta}_{L_{2}(\Omega_{11})}  \nnorm{b\cdot \nabla \chi}_{L_{2}(\Omega_{11})} + \nnorm{\eta}_{L_{\infty}(\Omega\setminus\Omega_{11})}  \nnorm{b\cdot \nabla \chi}_{L_{1}(\Omega\setminus\Omega_{11})} \).
\end{align*}
Beim Übergang von $L_{1}$ nach $L_{2}$ gewinnen wir
\begin{align*}
  \nnorm{\nabla \chi}_{L_{1}(\Omega \setminus \Omega_{11})} &\leq C \meas^{\frac 12}(\Omega \setminus \Omega_{11})\nnorm{\nabla \chi}_{L_{2}(\Omega \setminus \Omega_{11})}\\
&\leq \epsilon^{\frac 12} (\ln N)^{\frac 12}   \nnorm{\nabla \chi}_{L_{2}(\Omega \setminus \Omega_{11})}\\
&\leq (\ln N)^{\frac 12} \nnnorm{\chi}_{\epsilon}. 
\end{align*}
In $\Omega_{11}$ nutzen wir eine inverse Ungleichung und erhalten
\begin{align*}
  \nnorm{\nabla \chi}_{L_{2}(\Omega_{11})} \leq C N \nnorm{\chi}_{L_{2} (\Omega_{11})} \leq C N \nnnorm \chi_{2}. 
\end{align*}
Mit den Interpolationsfehlerabschätzungen
\begin{align*}
  \nnnorm \eta _{\epsilon} &\leq C N^{-1} \max\norm{\psi'}\\
  \nnorm \eta _{L_{2}(\Omega_{11})} &\leq C N^{-2}\\
  \nnorm \eta _{L_{\infty}(\Omega\setminus\Omega_{11})} &\leq C (N^{-1} \max \norm{\psi'})^{2}
\end{align*}
aus Satz \ref{thm:7-9} und Lemma \ref{lem:7-10} folgt
\begin{align*}
  \nnnorm \chi_{\epsilon}^{2} & \leq C (N^{-1} \max\norm{\psi'} + N^{-2}N + (N^{-1}\max\norm{\psi'})^{2}(\ln N)^{\frac 12})\nnnorm \chi_{\epsilon}\\
& \leq C N^{-1} \max\norm{\psi'} \nnnorm \chi_{\epsilon}. 
\end{align*}
\end{beweis}
Für Probleme mit charakteristischen Grenzschichten ist
\begin{align*}
  a(u, v) = \epsilon(\nabla u, \nabla v) + (c u - b u_{x}, v). 
\end{align*}
Alle anderen Eigenschaften bleiben und wir erhalten die Konvergenzabschätzungen:
\begin{satz}\label{thm:7-14}
  Auf einem S-Typ-Gitter mit $\sigma>2$ gilt für $h \leq C N^{-1}$ und $N^{-\frac 12} \max \norm{\psi'} (\ln N)^{\frac 12} \leq C$
  \begin{align*}
    \nnnorm{u - u^{N}}_{\epsilon} \leq C N^{-1} \max \norm{\psi'}. 
  \end{align*}
\end{satz}
\begin{beweis}
  Wie in Satz \ref{thm:7-13} gilt
  \begin{align*}
    \nnnorm \chi _{\epsilon}^{2} \leq C \nnnorm \eta _{\epsilon} \nnnorm \chi _{\epsilon} + C\norm{(\eta, b \cdot  \chi_{x})}.
  \end{align*}
In $\Omega_{11}$ und $\Omega_{12} \cup \Omega_{22}$ gilt wie zuvor
\begin{align*}
  \norm{(\eta,b \cdot \chi_{x} )_{\Omega_{11}}} &\leq C \nnorm{\eta}_{L_{2}(\Omega_{11})} N \nnorm{\chi}_{L_{2}(\Omega_{11})} \leq C N^{-1} \nnnorm{\chi}_{\epsilon}\\
\norm{(\eta,b \cdot \chi_{x} )_{ \Omega_{12} \cup \Omega_{22}}} &\leq C \meas^{\frac 12}(\Omega_{12} \cup \Omega_{22}) \nnorm{\eta}_{L_{\infty}(\Omega_{12} \cup \Omega_{22})} \nnorm{\chi_{x}}_{L_{2}}\\
&\leq C (N^{-1}\max \norm{\psi'})^{2}(\ln N)^{\frac 12} \nnnorm{\chi}_{\epsilon}. 
\end{align*}
Aber $\meas^{\frac 12}(\Omega_{11}) \leq C \epsilon^{\frac 14}(\ln N)^{\frac 12}$ und somit funktioniert obige Technik hier nicht (ganz).
\begin{align*}
  \nnorm{\chi_{x}}_{L_{2}(\Omega_{21})} &\leq C \epsilon^{\frac 12} \nnnorm{\chi}_{\epsilon}\\
  \nnorm{\chi_{x}}_{L_{2}(\Omega_{21})} &\leq C N \nnnorm{\chi}_{\epsilon}\\
\implies \quad   \nnorm{\chi_{x}}^{2}_{L_{2}(\Omega_{21})} &\leq C N \epsilon^{- \frac 12} \nnnorm{\chi}^{2}_{\epsilon}.
\end{align*}
Dann
\begin{align*}
  \norm{(\eta, b \cdot \chi_{x})_{\Omega_{21}}} &\leq C \nnorm{\eta}_{L_{2}(\Omega_{21})} \nnorm{\chi_{x}}_{L_{2}(\Omega_{21})}\\
  &\leq C \meas^{\frac 12}(\Omega_{21}) \nnorm{\eta}_{L_{\infty}(\Omega_{21})}\epsilon^{- \frac 14} N^{\frac 12} \nnnorm{\chi}_{\epsilon}\\
  &\leq C (N^{-1} \max \norm{\psi'})^{2} (\ln N)^{\frac 12} N^{\frac 12} \nnnorm{\chi}_{\epsilon}\\
\implies \quad \nnnorm \chi_{\epsilon}^{2} &\leq C \( N^{-1}\max \norm{\psi'} + N^{-1} + (N^{-1} \max \norm{\psi'})^{2} (\ln N)^{\frac 12} + (N^{-1} \max \norm{\psi'})^{2}(\ln N)^{\frac 12} N^{\frac 12}\) \nnnorm \chi_{\epsilon}\\
& \leq C N^{-1} \max \norm{\psi'} \nnnorm \chi_{\epsilon}. 
\end{align*}
\end{beweis}
Auch in 2D kann bezüglich der Standart-Interpolierenden in $Q_{1}$ Supercloseness gezeigt werden. Als Hilfsmittel können dazu die Lin-Formeln verwendet werden.
\begin{lemma}\label{lem:7-15}
  Es sei $\psi_{ij}$ ein Rechteck mit dem Mittelpunkt $(\tilde x_{i}, \tilde y_{i})$ und den Seiten $l_{1}, l_{2}$, die parallel zur $y$-Achse verlaufen. Für jede Funktion $w \in H^{3}(\bar \tau_{ij})$ und $\chi \in Q_{1}(\tau_{ij})$ gilt dann mit
  \begin{align*}
    E_{i}(x) &= \frac{(x - \tilde x_{i})^{2}} 2 - \frac{h_{i}^{2}}8, \\
    F_{j}(y) &= \frac{(y - \tilde y_{j})^{2}} 2 - \frac{k_{j}^{2}}8:\\
    \int_{\tau_{ij}} (w - w^{I})_{x} \chi_{x} &= \int_{\tau_{ij}} (F_{j} \chi_{x} - \frac 13 (F_{j}^{2})' \chi_{xy})w_{xyy}\\
    \int_{\tau_{ij}} (w - w^{I})_{x} \chi &= H_{ij}(w, \chi) + \frac{h_{i}^{2}}{12}(\int_{l_{1}} - \int_{l_{2}}) \chi w_{xx} dy\\
  \end{align*}
wobei
\begin{align*}
  H_{ij}(w, \chi) = \int_{\tau_{ij}} (F_{j}(\chi - E_{i}' \chi_{x}) - \frac{(F_{j}^{2})'}3(\chi_{y} - E_{i}' \chi_{xy})) w_{xyy} + \int_{\tau_{ij}}(\frac{E_{i}^{2}} 6 \chi_{x} - \frac{h_{i}^{2}}{12} \chi)w_{xxx}. 
\end{align*}
\end{lemma}
\begin{beweis}
  Ohne Indizes $i, j$. Es gilt
  \begin{align*}
    F'(y) &= y - \tilde y, \\
    F''(y) &= 1\\
    (F^{2}(y))''' &= 6(y - \tilde y)\\
    v &\coloneqq w - w^{I}. 
  \end{align*}
Mit Taxlor folgt
\begin{align*}
  \chi_{x} (x, y) = \chi_{x}(\tilde x, \tilde y) + (y - \tilde y) \chi_{xy}(\tilde x, \tilde y) = F''(y) \chi_{x}(\tilde x, \tilde y) + \frac 16(F^{2}(y))''' \chi_{xy}(\tilde x, \tilde y). \\
\implies \quad \int_{\tau} v_{x}\chi_{x} = \int_{\tau}v_{x}F''(y) \chi_{x}(\tilde x, \tilde y) + \frac 16 \int_{\tau} v_{x} (F^{2}(y))''' \chi_{xy}(\tilde x,\tilde y). 
\end{align*}
Wir fangen mit dem ersten Term an und wenden partielle Integration an:
\begin{align*}
  \int_{\tau}v_{x} F''(y) &= (\int_{l_{4}}- \int_{l_{3}}) v_{x} F'(y) dx - \int_{\tau} v_{xy}F'(y)\\
  &= F'(y)|_{l_{4}} \( v(P_{3}) - v(P_{2})\) - F'(y)|_{l_{3}} \( v(P_{1}) - v(P_{0})\) - (\int_{l_{4}} - \int_{l_{3}})v_{xy} F(y) dx + \int_{\tau}v_{xyy}F(y). 
\end{align*}

\begin{figure}[ht!]
  \centering
  \begin{tikzpicture}
    \draw (-1,-1) -- (1,-1) -- (1,1) -- (-1,1) -- cycle;
    \draw[fill] (-1,-1)  circle (1.5pt);
    \draw[fill] (1,-1)  circle(1.5pt);
    \draw[fill] (1,1) circle(1.5pt);
    \draw[fill] (-1,1) circle(1.5pt);

    \node at (0, 0) {$\tau$};
    \node[right] at (1, 0) {$l_{2}$};
    \node[above] at (0, 1) {$l_{4}$};
    \node[above right] at (1, 1) {$p_{3}$};
    \node[left] at (-1, 0) {$l_{0}$};
    \node[below] at (0, -1) {$l_{3}$};
    \node[below left] at (-1, -1) {$p_{0}$};
    \node[above left] at (-1, 1) {$p_{3}$};
    \node[below right] at (1, -1) {$p_{1}$};
  \end{tikzpicture}
  \caption{Bezeichnungen für $\tau$}
  \label{fig:tau}
\end{figure}




 Da $v = w - w^{I}$ ist $v(P_{j}) = 0$, $F(y)|_{l_{3}} = F(y)|_{l_{4}} = 0$.
\begin{align*}
  \implies \quad \int_{\tau} v_{x} F''(y) = \int_{\tau} v_{xyy}F(y). 
\end{align*}
Weiter
\begin{align*}
  \int_{\tau} v_{x}(F^{2}(y))''' &= \underbrace{(\int_{l_{4}} - \int_{l_{3}}) v_{x}(F^{2}(y))''dx}_{= 0} - \int_{\tau} v_{xy} (F^{2}(y))'' \\
  &= - \underbrace{(\int_{l_{4}} - \int_{l_{3}}) v_{xy}(F^{2}(y))'dx}_{= 0} - \int_{\tau} v_{xyy} (F^{2}(y))' \\
&= \int_{\tau} v_{xyy}(F^{2}(y))'\\
\implies \quad \tau (w - w^{I})_{x} \chi_{x} &= \int_{\tau} (F(y) \chi_{x}(\tilde x, \tilde y) + \frac 16 (F^{2}(y))' \chi_{xy}(\tilde x, \tilde y)) w_{xyy}.  
\end{align*}
Mit
\begin{align*}
  \chi_{x}(\tilde x, \tilde y) &= \chi_{x}(x, y) - (y - \tilde y)\chi_{xy}(x, y)\\
%  &= \chi_{x}(x, y) - \frac 12(F^{2}(y))'\chi_{xy}(x, y)\\
\implies \quad \int_{\tau}(w - w^{I})_{x} \chi_{x} &= \int_{\tau} F(y) \chi_{x}(x, y) w_{xyy}- \int_{\tau} F(y)(y - \tilde y) \chi_{xy} w_{xyy} + \frac 16 \int_{\tau}(F^{2}(y))' \chi_{xy} w_{xyy}.  
\end{align*}
Mit $(F^{2}(y))' = 2 F(y) F'(y) = 2 F(y)(y - \tilde y)$ folgt
\begin{align*}
  \int_{\tau} (w - w^{I})_{x} \chi_{x} = \int_{\tau}(F(y) \chi_{x} - \frac 13 (F^{2}(y))' \chi_{xy})w_{xyy}.
\end{align*}
Die zweite Formel kann ebenfalls mit einer Taylorentwicklung $\chi(x, y) = \chi(\tilde x, \tilde y) + (x - \tilde x)\chi_{x}(\tilde x, \tilde y) + (y - \tilde y)\chi_{y}(\tilde x, \tilde y)+ (x - \tilde x)(y - \tilde y)\chi_{xy}$ + Ableitungen von $E, F$ und partieller Integration gezeigt werden.
\end{beweis}

% \datum{22. Januar 2016}

Mit Hilfe des Lemmas \label{lem:7-15}, Cauchy-Schwarz-Ungleichung und inverser Ungleichung erhalten wir
\begin{align*}
  \norm{((w-w^{I})_{x}, \chi_{x})_{\tau_{ij}}} \leq C k_{j}^{2} \nnorm{w_{xyy}}_{L_{2}(\tau_{ij})} \nnorm{\chi_{x}}_{L_{2}(\tau_{ij})}
\end{align*}
sowie
\begin{align*}
  \norm{H_{ij}(w, \chi)} \leq C \(k_{j}^{2} \nnorm{w_{xyy}}_{L_{2}(\tau_{ij})} + h_{i}^{2} \nnorm{w_{xxx}}_{L_{2}(\tau_{ij})}\) \nnorm{\chi}_{L_{2}(\tau_{ij})}. 
\end{align*}
Einen alternativen Weg für die obere Ungleichung nutzt ein Resultat von Zlámal (1978 % Geburtsjahr S.Franz
), siehe \cite{DPL_IMAJNA}.
\begin{lemma}\label{lem:7-16}
  Es sei $\tau_{ij}$ ein achsenparalleles Rechteck und $w \in H^{3}(\tau_{ij})$ sowie $\chi \in Q_{1}(\tau_{ij})$. Dann gilt
  \begin{align*}
    \norm{((w-w^{I})_{x}, \chi_{x})_{\tau_{ij}}} \leq C\( k_{j}^{2} \nnorm{w_{xyy}}_{L_{2}(\tau_{ij})} + h_{i}k_{j}\nnorm{w_{xxx}}_{L_{2}(\tau_{ij})} + h_{i}^{2}\nnorm{w_{xxy}}_{L_{2}(\tau_{ij})}\)\nnorm{\chi_{x}}_{L_{2}(\tau_{ij})}. 
  \end{align*}
\end{lemma}
\begin{beweis}
  Es sei $p \in P_{2}(\tau_{ij})$. Dann gilt mit der Mittelpunktsregel
  \begin{align*}
    \int_{\tau_{ij}}(p - p^{I})_{x}\chi_{x} &= \int_{x_{i-1}}^{x_{i}} \int_{y_{i-1}}^{y_{i}} \underbrace{(p - p^{I})_{x}}_{\in P_{1(\tau_{ij})}} \underbrace{\chi_{x}}_{= a+ by} dydx \\
&= h_{i}\int_{y_{i-1}}^{y_{i}} (p-p^{I}_{x})|_{x = i- \frac 12} \chi_{x} dy. 
  \end{align*}
Nun ist $(p-p^{I})_{x}|_{x= x_{i- \frac 12}} = 0$. Nun ist oBdA. $\tau= (0, 1)^{2}$ und $p = a + bx + cy + d xy + e x^{2} + fy^{2}$
\begin{align*}
  p_{x} &= b + dy + 2ex\\
  p^{I}_{x} &= b + dy + e\\
\implies \quad p_{x}(\frac 12 , y) &= (p^{I}_{x})(\cdot, y)\\
\implies \quad \int_{\tau_{ij}} (p-p^{I})_{x} \chi_{x} &= 0. 
\end{align*}
Weiter
\begin{align*}
  \norm{((w-w^{I})_{x}, \chi_{x})_{\tau_{ij}}} &=  \norm{(((w-p) - (w-p)^{I})_{x}, \chi_{x})_{\tau_{ij}}} \\
 &\leq  \nnorm{((w-p) - (w-p)^{I})_{x}}_{L_{2}(\tau_{ij})} \nnorm{ \chi_{x}}_{L_{2}(\tau_{ij})} \\
 &\leq C\(h_{i} \nnorm{(w-p)_{xx}}_{L_{2}(\tau_{ij})} + k_{j} \nnorm{(w-p)_{xy}}_{L_{2}(\tau_{ij})}\) \nnorm{ \chi_{x}}_{L_{2}(\tau_{ij})}. 
\end{align*}
Da $p \in P_{2}$ noch frei ist, können wir es geeignet wählen und erhalten zum Beispiel für ein gemitteltes Taylor-Polynom zu $w$
\begin{align*}
  \nnorm{(w-p)_{xx}}_{L_{2})(\tau_{ij})} & \leq C \( h_{i} \nnorm{w_{xxx}}_{L_{2}(\tau_{ij})} + k_{j} \nnorm{w_{xxy}}_{L_{2}(\tau_{ij})}\)\\
  \nnorm{(w-p)_{xy}}_{L_{2})(\tau_{ij})} & \leq C \( h_{i} \nnorm{w_{xxy}}_{L_{2}(\tau_{ij})} + k_{j} \nnorm{w_{xyy}}_{L_{2}(\tau_{ij})}\)\\
\implies \quad   \norm{(w-w^{I})_{x}, \chi_{x}}_{\tau_{ij}} & \leq C \( h^{2}_{i} \nnorm{w_{xxx}}_{L_{2}(\tau_{ij})} + h_{i}k_{j} \nnorm{w_{xxy}}_{L_{2}(\tau_{ij})} + k_{j}^{2} \nnorm{w_{xyy}}_{L_{2}(\tau_{ij})}\) \nnorm{\chi_{x}}_{L_{2}(\tau_{ij})}. 
0
\end{align*}
\end{beweis}

\begin{satz}\label{thm:7-17}
  Auf einem S-Typ-Gitter mit $\sigma \geq \frac 52$ und $h \leq CN^{-1}$ gilt für die Galerkin-Lösung des Problems mit exponentiellen bzw. des Problems mit charakteristischen Grenzschichten
  \begin{align*}
    \nnnorm{u^{N} - u^{I}}_{\epsilon} &\leq C(N^{-2} (\ln N)^{\frac 12} + (N^{-1} \max \norm{\psi'})^{2}).  
  \end{align*}
\end{satz}
\begin{beweis}
Exponentielle Grenzschichten:  \cite{L_NMPDE},
Charakteristische Grenzschichten:  \cite{FL_NMPDE}.

Clevere Kombination der Interpolations- , Stabilitätsabschätzungen und der Lin-Formeln liefern das Ergebnis. 

Für die darin auftretenden Linienintegrale $I_{3}$ in $\Omega_{12} \cup \Omega_{22}$ lässt sich für den glatten Anteil $v$
\begin{align*}
  \norm{I_{3}} & \leq C (h^{2} + N^{-2})\nnorm{\chi_{x}v_{xx} + \chi v_{xxx}}_{L_{1}(\Omega_{12} \cup \Omega_{22})}\\
  &\leq C (h^{2} + N^{-2})\epsilon^{\frac 12} (\ln N)^{\frac 12} \nnorm{\chi_{x}v_{xx} + \chi v_{xxx}}_{L_{2}(\Omega_{12} \cup \Omega_{22})}\\
  &\leq C  N^{-2} (\ln N)^{\frac 12} \nnnorm{\chi}_{\epsilon}
\end{align*}
zeigen, was vermutlich suboptimal ist. 
\end{beweis}
Mit biquadratischer Interpolation auf einem geeigneten Makrogitter (siehe 1D) erhält man dann
\begin{align*}
  \nnnorm{u - Pu^{N}}_{\epsilon} \leq C (N^{-2}(\ln N)^{\frac 12} + (N^{-1} \max \norm{\psi'})^{2}). 
\end{align*}

\subsection{Lokale Projektionsstabilisierung - LPS}
\label{sec:lokale-proj-lps}

Die Idee des LPS hatten wir bereits im Kapitel 6.4 anklingen lassen. Das Galerkin-Verfahren stabilisiert selbst alle 'groben' Moden, bis auf die feinste. Die SDFEM-Stabilisierung hilft, indem mit $(\delta(Lu - f), b \cdot \nabla v)$ stabilisiert wird. Dabei ist der Term
\begin{align*}
(  \delta b \cdot \nabla u, b \cdot\nabla v)
\end{align*}
der hauptsächlich hilfreiche Term. Die restlichen Terme werden für die Konsistenz benötigt, rufen allerdings zusätzliche Kopplungen, zum Beispiel in Systemen von PDEs hervor. 

Wir betrachten daher LPS als modernere Stabilisierungsmethode (die allerdings nicht ganz so gut ist). Wir definieren diese für beliebige Polynomgrade $p$ als \markdef{1-Level-Ansatz}, und hier nur für Probleme mit charakteristischen Grenzschichten. Sei dazu $\Pi_{\tau}$ die lokale $L_{2}$-Projektion in den Funktionenraum $D(\tau) = \cP_{p-1}(\tau)$. Darauf aufbauend gibt es einen Fluktuationsoperator $\kappa_{\tau}: L_{2}(\tau) \to L_{2}(\tau)$ ist definiert als
\begin{align*}
  \kappa_{\tau}(v) \coloneqq v - \Pi_{\tau}(v)
\end{align*}
und 'schneidet' die groben Skalen ab. 
\begin{beispiel*} $p = 3$, 

  $w = x(1 - x)\in \cP_{2}[0, 1]$, $\chi = \sin(42 \pi x) \implies v = w +\chi$
  \begin{align*}
    \kappa(v) = v - \Pi v = v + \chi - v = \chi
  \end{align*}
Oszillationen
\end{beispiel*}
Es gilt die $L_{2}$-Stabilität
\begin{align*}
  \nnorm{\kappa_{\tau}v}_{L_{2}(\tau)} \leq C \nnorm v _{L_{2}(\tau)} 
\end{align*}
und die anisotropen Fehlerabschätzungen
\begin{align*}
  \nnorm{\kappa_{\tau} v}_{L_{2}(\tau_{ij})} \leq C \sum_{r = 0}^{s} \nnorm{h^{s-r}_{i} k_{j}^{r} \frac{\partial^{s} v}{\partial^{s-r} x\partial^{r}y}}_{L_{2}(\tau_{ij})}
\end{align*}
für $0 \leq s \leq p$. 

Der Stabilisierungsterm lautet mit einer noch zu definierenden Parameterfunktion $\delta$
\begin{align*}
  s(u, v) = \sum_{\tau \subseteq \Omega} (\delta_{\tau}\kappa_{\tau}(bu_{x}), \kappa_{\tau}(b v_{x}))_{\tau}
\end{align*}
Aufgrund der Struktur von $s(\cdot, \cdot)$ gilt
\begin{align*}
  \norm{s(u, v)} \leq (c(u, u))^{\frac 12} (s(v, v))^{\frac 12}
\end{align*}
(Cauchy-Schwarz). 
Bilinearformenbetrachtung: 

Die Galerkin-Bilinearform lautet
\begin{align*}
  a_{Gal}(u, v) &= \epsilon(\nabla u, \nabla v) + (c u - bu_{x}, v), 
\end{align*}
und die stabilisierte LPS-Bilinearform
\begin{align*}
  a_{LPS}(u, v) &= a_{Gal}(u, v) + s(u, v). 
\end{align*}
Das stabilisierte Problem lautet dann mit einem diskreten Raum $V_{N} \subset H_{0}^{1}(\Omega)$: 

Gesucht ist ein $u^{N} \in V_{N}$ mit $a_{LPS}(u^{N}, v) = f(v) = (f, v)$ für alle $v \in V_{N}$. 

Die LPS-Bilinearform erfüllt nur noch eine 'schwache' Galerkin-Orthogonalität
\begin{align*}
  a_{LPS}(u - u^{N}, v) = s(u, v) \quad \forall v \in V_{N}. 
\end{align*}
Zugeordnete Normen sind
\begin{align*}
  \nnnorm{u}_{\epsilon} = (\epsilon \nnorm{\nabla u}_{L_{2}}^{2} + \gamma \nnorm u_{L_{2}}^{2})^{\frac 12}, 
\end{align*}
koerzitiv mit $a_{Gal}(\cdot, \cdot)$ und 
\begin{align*}
    \nnnorm{u}_{LPS} = (\epsilon \nnorm{\nabla u}_{L_{2}}^{2} + \gamma \nnorm u_{L_{2}}^{2} + s(u, u))^{\frac 12}, 
\end{align*}
koerzitiv mit $a_{LPS}(\cdot, \cdot)$.
%%% Vorlesungsteile fehlen!
Für $\delta$ wurde 
NICHT LESBAR
%%% ein Wort fehlt
eine stückweise konstante Funktion gewählt. 
Da die Stabilisierung in der axponentiellen Grenzschicht vernachlässigbar ist ($\delta \to 0$ für $\epsilon \to 0$) wählen wir

\begin{align*}
  \delta|_{\tau} =
  \begin{cases}
    0, & \tau \subset \Omega_{12} \cup \Omega_{22}\\
    \delta_{11}, & \tau \subset \Omega_{11} \\
    \delta_{21}, & \tau \subset \Omega_{21} \\
  \end{cases}
\end{align*}
Der erste Fall entspricht der exponentiellen Grenzschicht und den Eckbereichen, der zweite Fall ist der grobe Bereich und der dritte Fall stellt die charakteristische Grenzschicht dar. Mithilfe der anisotropen Interpolationsfehlerabschätzung von Satz \ref{thm:7-1} kann folgendes nützliches Resultat gezeigt werden:
\begin{lemma}\label{lem:7-18}
  Für den glatten Anteil der Lösung $u$ gilt
  \begin{align*}
    \norm{S(v - Iv, v-Iv)} & \leq (\delta_{11} + \delta_{21}\epsilon^{\frac 12} \ln N)N^{-2p}, 
  \end{align*}
falls $h \leq CN^{-1}$. 
\end{lemma}
\begin{beweis}
  Mit der $L_{2}$-Stabilität von NICHT LESBAR und der Interoplationsfehlerabschätzung von $I$ gilt
  \begin{align*}
\norm{S(v - Iv, v-Iv)} & \leq \sum_{\tau \subset \Omega_{11} NICHT LESBAR} \delta_{\tau} \nnorm{\kappa_{\tau}(b(v-Iv)_{x})}_{L_{2}(\tau)}^{2}\\
& \leq C \sum_{\tau \subset NICHT LESBAR} \delta_{\tau} \nnorm{(v-Iv)_{x}}_{L_{2}(\tau)}^{2}\\
& \leq C (\delta_{11}(N^{-2p} \nnorm{\partial_{x}^{p+1}v}_{L_{2}(\Omega_{21})}^{2} + N^{-2p} \nnorm{\partial_{x}\partial_{y}^{p}v}^{2}_{L_{2}(\Omega_{11})}+ \delta_{21} (N^{-2p}\nnorm{\partial_{x}^{p+1}v}^{2}_{L_{2}(\Omega_{21})} + k^{2p} \nnorm{\partial_{x}\partial_{y}^{p}v}^{2}_{L_{2}(\Omega_{21})})\\
& \leq C (\delta_{11} + \delta_{21}\epsilon^{\frac 12} \ln N)N^{-2p}. 
  \end{align*}
\end{beweis}

Lemma \ref{lem:7-18} reicht nicht aus, um $\nnnorm{u - Iu}_{LPS}$ abzuschätzen. Aber $\norm{S(w - Iw, w- Iw)}$ lässt sich für $w = w_{1} + w_{2} + w_{12}$ NICHT LESBAR gleichmäßig abschätzen. 

%\datum{28. Januar 2016}
Kombinieren wir die Idee von Satz \ref{thm:6-22}, das heißt mit
\begin{align*}
  S &= [\lambda_{x}, \lambda_{x} + H]\times [0, 1], \\
\delta|_{\tau} &=
\begin{cases}
  0, & \tau \subset \Omega_{12} \cup \Omega_{22}\\
\frac{x- \lambda_{x}}H \delta_{11},& \tau \subset \Omega_{11} \cap S\\
\delta_{11}, &\tau \subset \Omega_{11}\setminus S\\
\frac{x- \lambda_{x}}H \delta_{21},&\tau \subset \Omega_{21} \cap S\\
\delta_{21},&\tau \subset \Omega_{21} \setminus S.
\end{cases}
\end{align*}
mit der LPS-Stabilisierung, so folgt:
\begin{lemma}\label{lem:7-19}
  Auf einem S-Typ-Gitter mit $\sigma \geq p + \frac 12$ gilt für $\epsilon \norm{\ln \epsilon} \leq \beta \cdot H$
  \begin{align*}
\norm{S(u - Iu, u - Iu)} & \leq C \(\delta_{11}N^{-2p} + \delta_{21}\epsilon^{\frac 12} \( \ln N N^{-2p} + (N^{-1} \max \norm{\psi'})^{2p}\)\), 
  \end{align*}
falls $h \leq CN^{-1}$. 
\end{lemma}
\begin{beweis}
  $\Omega_{11}$: Nutze die Stabilitätsabschätzung
  \begin{align*}
    \nnorm{(Iw)_{x}}_{L_{\infty}(\tau)} \leq C \cdot \nnorm{w_{x}}_{L_{\infty}(\tau)}. 
  \end{align*}
Fangen wir mir $w = w_{1} + w_{12}$ an:
\begin{align*}
  \sum_{\tau \subset \Omega_{11}\setminus S} \delta_{11} \nnorm{\kappa_{\tau} (b(w - Iw)_{x})}_{L_{2}(\tau)}^{2} & \leq C  \sum_{\tau \subset \Omega_{11}\setminus S} \delta_{11} \nnorm{ b(w - Iw)_{x}}_{L_{2}(\tau)}^{2} \\
& \leq C \cdot \delta_{11} \( \nnorm{w _{x}}_{L_{2}(\Omega_{11}\setminus S)}^{2} + \nnorm{(Iw) _{x}}_{L_{2}(\Omega_{11}\setminus S)}^{2}\) \\ 
& \leq C \cdot \delta_{11} \nnorm{w _{x}}_{L_{\infty}(\Omega_{11}\setminus S)}^{2}\\
& \leq C \cdot \delta_{11} \cdot \epsilon^{-2} w_{x}^{2}|_{x = \lambda_{x} + H}\\
& \leq C \cdot \delta_{11} \cdot \epsilon^{-2} e^{- \frac{2\beta (\lambda_{x} + H)}{\epsilon}}\\
& \leq C \cdot \delta_{11} \cdot \epsilon^{-2} N^{-2\sigma} \epsilon^{2}\\
& \leq C \cdot \delta_{11}  N^{-2\sigma}. 
\end{align*}
Für die Abschätzungen auf $\Omega_{11}\cap S$ benötigen wir das Hilfsresultat
\begin{align*}
  \int_{\lambda_{x}}^{\lambda_{x} + H} (x - \lambda_{x})\kappa^{2}(v)dx & \leq 2  \int_{\lambda_{x}}^{\lambda_{x} + H} (x - \lambda_{x})v^{2}(x)dx + 2 p^{2}\nnorm v^{2}_{L_{1}(\lambda_{x}, \lambda_{x}+H)}, 
\end{align*}
welches wir hier nicht beweisen werden ($p$ ist der Polynomgrad). 

Auf $\Omega_{11}\cap S$:
\begin{align*}
  \delta_{11} &\underbrace{\int_{\lambda_{x}}^{\lambda_{x} + H}\int_{\lambda_{y}}^{1-\lambda_{y}}}_{= \Omega_{11} \cap S} \frac {x - \lambda_{x}} H \kappa^{2}(b(w - Iw)_{x})dydx\\
&\leq C \frac{\delta_{11}}H \(\int_{\Omega_{11}\cap S} (x - \lambda_{x})((w - Iw)_{x})^{2} + \nnorm{(w - Iw)_{x}}^{2}_{L_{1}(\Omega_{11}\cap S)} \)\\
&\leq C \frac{\delta_{11}}H 
\(\int_{\Omega_{11}\cap S} (x - \lambda_{x})w_{x}^{2} +
\int_{\Omega_{11}\cap S} (x - \lambda_{x})(Iw)_{x}^{2} 
+ \nnorm{w_{x}}^{2}_{L_{1}(\Omega_{11}\cap S)}
+ \nnorm{(Iw)_{x}}^{2}_{L_{1}(\Omega_{11}\cap S)} \)\\
&\leq C \frac{\delta_{11}}H \( \epsilon^{-2}\epsilon^{2} N^{-2\sigma} + N^{2}N^{-2\sigma} N^{-2} + \epsilon^{-2}\epsilon^{2} N^{-2\sigma} + N^{2}N^{-2}N^{-2\sigma}\)\\
& \leq C \delta_{11} N^{-(2\sigma -1)}. 
\end{align*}
Inverse Ungleichung, usw.

Für $w_{2}$ erhalten wir analog
\begin{align*}
\sum_{\tau \subset \Omega_{11}} \nnorm{ delta_{\tau}^{\frac 12} \kappa_{\tau} (b(w_{2} - Iw_{2})_{x})}_{L_{2}(\tau)}^{2} & \leq C \delta_{21} N^{-2\sigma}. 
\end{align*}
In $\Omega_{21}$ verfahren wir für $w = w_{1} + w_{12}$ ähnlich, da diese abgeklungen sind und erhalten
\begin{align*}
  \sum_{\tau \subset \Omega_{21}} \nnorm{ \delta_{\tau}^{\frac 12} \kappa_{\tau} (b(w - Iw_{x})}_{L_{2}(\tau)}^{2} & \leq C \delta_{21} N^{-(2\sigma - 1)} \epsilon^{\frac 12} \ln N.  
\end{align*}
Da $w_{2}$ nicht abgeklungen ist, nutzen wir $\delta_{\tau} \leq \delta_{21}$ und die Standard-Fehlerabschätzungen und erhalten damit
\begin{align*}
  \sum_{\tau \subset \Omega_{21}} \nnorm{ \delta_{\tau}^{\frac 12} \kappa_{\tau} (b(w_{2} - Iw_{2})_{x})}_{L_{2}(\tau)}^{2} & 
\leq C \delta_{21} \sum_{\tau \subset \Omega_{21}} \nnorm{(w_{2} - Iw_{2})_{x}}_{L_{2}(\tau)}^{2}\\
&\leq C \delta_{21} \( N^{-2p}\nnorm{\partial_{x}^{p+1} w_{2}}^{2}_{L_{2}(\Omega_{21})} + \nnorm{k_{j}^{p} \partial_{x} \partial_{y}^{p} w_{2}}^{2}_{L_{2}(\Omega_{21})}\)\\
&\leq C \delta_{21} \( N^{-2\sigma} \epsilon^{\frac 12} + (N^{-1}\max\norm{\psi'})^{2p} \epsilon^{\frac 12} \)\\
&\leq C \delta_{21} \epsilon^{\frac 12} (N^{-1}\max\norm{\psi'})^{2p}. 
\end{align*}
Mit $\sigma\geq p + \frac 12$ und Lemma \ref{lem:7-18} folgt die Aussage. 
\end{beweis}

Analog zum bilinearen Fall in Satz \ref{thm:7-12} gilt
\begin{satz}\label{thm:7-20}
  Auf einem S-Typ-Gitter mit $\sigma \geq p+1$ und $h \leq CN^{-1}$ gilt
  \begin{align*}
    \nnorm{u - Iu}_{L_{\infty}(\Omega_{11})} & \leq CN^{-(p+1)}\\
    \nnorm{u - Iu}_{L_{\infty}(\Omega\setminus\Omega_{11})} & \leq C(N^{-1} \max \norm{\psi'})^{p+1}
  \end{align*}
sowie
\begin{align*}
    \nnnorm{u - Iu}_{\epsilon} & \leq C(N^{-1}\max\norm{\psi'})^{p}. 
\end{align*}
\end{satz}
Die Konvergenz der LPS-FEM ist in der LPS-Norm also beweisbar:

\begin{satz}\label{thm:7-21}
  Auf einem S-Typ-Gitter mit $\sigma \geq p+1$, $\epsilon \norm{\ln \epsilon} \leq \beta \cdot H$ und $N^{-\frac 12} \max \norm{\psi'} (\ln N)^{\frac 12} \leq C$, sowie $h \leq C N^{-1}$ gilt für $0 \leq \delta_{11} \leq C$ und $0 \leq \delta_{21} \leq C \epsilon^{- \frac 12}$:
  \begin{align*}
\nnnorm{u - n^{N}}_{LPS} \leq C \( (N^{-1} \max \norm{\psi'})^{p} + (\ln N)^{\frac 12} N^{-p} \).  
  \end{align*}
\end{satz}
\begin{beweis}
  Wie üblich fangen wir mit
  \begin{align*}
    \nnnorm{u - u^{N}}_{LPS} &\leq \nnnorm{u - Iu}_{LPS} + \nnnorm{Iu - u^{N}}_{LPS} 
  \end{align*}
an. Mit Lemma \ref{lem:7-19} und Satz \ref{thm:7-20} folgt
\begin{align*}
    \nnnorm{u - Iu}_{LPS} &\leq C\( (N^{-1} \max \norm{\psi'})^{p} + (\ln N)^{\frac 12} N^{-p} \).  
\end{align*}
Mit Koerzitivität und schwacher Galerkin-Orthogonalität folgt für $\chi = Iu - u^{N}$, $\eta = Iu- u$
\begin{align*}
    \nnnorm{Iu - u^{N}}^{2}_{LPS} &\leq a_{LPS}(Iu - u^{N}, \chi)\\
&= a_{LPS}(\eta, \chi) + a_{LPS}(u - u^{N}, \chi)\\
&= a_{LPS}(\eta, \chi) + S(u, \chi)\\
&= a_{Gal}(\eta, \chi) + S(Iu, \chi)
\end{align*}
Für die Anteile der Galerkin-Bilinearform verfahren wir wie im Beweis zu Satz \ref{thm:7-14}:
\begin{align*}
  \norm{a_{Gal}(\eta, \chi)} & \leq C \nnnorm{\eta}_{\epsilon}\nnnorm{\chi}_{\epsilon} + C\norm{(\eta, b \chi_{x})}\\
\norm{(\eta, b \chi_{x})_{\Omega_{11}}} & \leq C \nnorm \eta_{L_{2}(\Omega_{21})} N \nnorm \chi _{L_{2}(\Omega_{21})} \leq C N^{-p} \nnnorm \chi_{\epsilon}\\
\norm{(\eta, b \chi_{x})_{\Omega_{12} \cup \Omega_{22}}} & \leq C \meas^{\frac 12}(\Omega_{12} \cup \Omega_{22}) \nnorm \eta_{L_{2}(\Omega_{12} \cup \Omega_{22})} \nnorm {\chi_{x}} _{L_{2}(\Omega_{12} \cup \Omega_{22})}\\
& \leq C (\ln N)^{-\frac 12}(N^{-1} \max \norm{\psi'})^{\tau+1} \nnnorm \chi_{\epsilon}\\
\norm{(\eta, b \chi_{x})}_{\Omega_{21}} & \leq C \meas^{\frac 12}(\Omega_{21}) \nnorm \eta_{L_{\infty}(\Omega_{21}} \epsilon^{- \frac 14} N^{\frac 12} \nnnorm \chi _{\epsilon}\\
& \leq  C (N^{-1} \max \norm{\psi'})^{p+1} + (\ln N)^{\frac 12} N^{-\frac 12} \nnnorm \chi_{\epsilon}\\
\implies \quad  \norm{a_{Gal}(\eta, \chi)} & \leq C(N^{-1}\max \norm{\psi'})^{p} \nnnorm{\chi}_{LPS}. 
\end{align*}
%%% \tau ???
Den Stabilisierungsterm erweitern wir
\begin{align*}
  \norm{S(Iu, \chi)} & \leq \norm{S(Iu - u, \chi)}  + \norm{S( u, \chi)} \\
& \leq \norm{S(Iu - u, Iu - u)}^{\frac 12} S(\chi, \chi)^{\frac 12}  + \norm{S( u, \chi)} \\
& \leq C \( (N^{-1}\max \norm{\psi'})^{p} + (\ln N)^{\frac 12} N^{-p}\) \nnnorm \chi_{LPS} + \norm{S(u, \chi)}. 
\end{align*}
mit CSU und Lemma \ref{lem:7-19}. Um $\norm{S(u, \chi)}$ abzuschätzen können wir die anisotropen Interpolationsabschätzungen von $\kappa_{\tau}$ nutzen. Für $v$:
\begin{align*}
  \norm{S(u, \chi)} & \leq S(v, v)^{\frac 12} S(\chi, \chi)^{\frac 12} \leq S(v, v)^{\frac 12} \nnnorm \chi_{LPS}. \\
S(v, v) & = \sum_{\tau \subset \Omega_{11} \cup \Omega_{21}} (\delta_{\tau} \kappa_{\tau} (b v_{x}), \kappa_{\tau} (bv_{x}))_{\tau}\\
& \leq \nnorm{\delta^{\frac 12} \kappa(b v_{x})}^{2}_{L_{2}(\Omega_{11})}  +\nnorm{\delta^{\frac 12} \kappa(b v_{x})}^{2}_{L_{2}(\Omega_{21})} \\
& \leq C \delta_{11} \sum_{r = 0}^{p}\nnorm{h_{i}^{p-r} k_{j}^{r} \partial_{x}^{p-r} \partial_{y}^{r} v}^{2}_{L_{2}(\Omega_{11})} + 
 C \delta_{21} \sum_{r = 0}^{p}\nnorm{h_{i}^{p-r} k_{j}^{r} \partial_{x}^{p-r} \partial_{y}^{r} v}^{2}_{L_{2}(\Omega_{21})}\\
& \leq C (\delta_{11} N^{-2p} +  \delta_{21} N^{-2p} \epsilon^{\frac 12} \ln N) \\
& \leq C N^{-2p} \ln N. 
\end{align*}
Stillschweigend wurde
\begin{align*}
  \nnorm{\partial_{x}^{p-r}\partial_{y}^{r} (bv)}_{L_{2}(\tau)} \leq\nnorm{\partial_{x}^{p-r}\partial_{y}^{r} v}_{L_{2}(\tau)}
\end{align*}
angenommen.
Für $w = w_{1} + w_{12}$ teilen wir $\Omega_{11} \cup \Omega_{21}$ in $S = [\lambda_{x}, \lambda_{x} + H] \times [0, 1]$ und den Rest in
\begin{align*}
  S(w, w) &\leq C \left( \delta_{11} \nnorm{\kappa (b w_{x})}^{2}_{L_{2}(\Omega_{11}\setminus S)} + \right.
%\\
%& \quad
\delta_{11} \nnorm{\(\frac{x - \lambda_{x}} H \)^{\frac 12}\kappa (b w_{x})}^{2}_{L_{2}(\Omega_{11}\cap S)}
\\
& \quad
+\delta_{21} \nnorm{\kappa (b w_{x})}^{2}_{L_{2}(\Omega_{21}\setminus S)}+ 
%\\
%& \quad 
\left.\delta_{21} \nnorm{\(\frac{x - \lambda_{x}} H \)^{\frac 12}\kappa (b w_{x})}^{2}_{L_{2}(\Omega_{21}\cap S)}\right)\\
&= I + II + III + IV \\
I &\leq  \delta_{11} \nnorm{w_{x}}^{2}_{L_{2}(\Omega_{11}\setminus S)} \\
&\leq C \delta_{11} \epsilon^{-2} \epsilon e^{- \frac {2 \beta (\lambda_{x} + H)} \epsilon} \leq C \delta_{11} \epsilon N^{-2\sigma}\\
III & \leq C \delta_{21} \nnorm{ w_{x}}^{2}_{L_{2}(\Omega_{21}\setminus S)} \leq C \delta_{21} \epsilon^{ \frac 32} \ln N N^{-2\sigma}\\
II &\leq C\frac {\delta_{11}} H \epsilon^{-2} \( \int_{\Omega_{11} \cap S} (x - \lambda_{x}) e^{- \frac {2 \beta x} \epsilon} + \nnorm{e^{- \frac \beta x} \epsilon}^{2}_{L_{1}(\Omega_{11} \cap S)} \) \leq C \delta_{11} N^{-(2 \sigma - 1)}\\
IV &\leq \dots \leq C \delta_{21} \epsilon^{\frac 12} \ln N N^{-(2 \sigma - 1)}. 
\end{align*}
Es folgt
\begin{align*}
  S(w, w) \leq C (\delta_{11} + \delta_{21} \epsilon^{\frac 12} \ln N)N^{- (2 \sigma - 1)} \leq C \ln N N^{-2p}. 
\end{align*}
Bleibt noch $w_{2}$:
\begin{align*}
  S(w_{2}, w_{2}) &\leq C (\delta_{11} \nnorm{\kappa (b w_{2})_{x}}_{L_{2}(\Omega_{11})}^{2} + \delta_{21} \nnorm{\kappa (b w_{2})_{x}}_{L_{2}(\Omega_{21})}^{2} )\\
&\leq C (\delta_{11} \nnorm{ (w_{2})_{x}}_{L_{2}(\Omega_{11})}^{2} + \delta_{21} \sum_{r = 0}^{p} \nnorm{h_{i}^{p-r} k_{j}^{r} \partial_{x}^{p-r} \partial_{y}^{r} w_{2}}_{L_{2}(\Omega_{21})}^{2} )\\
& \leq C (\delta_{11}  \epsilon^{\frac 12} N^{-2 \sigma} + \delta_{21} (N^{-1} \max \norm{\psi'})^{2p} \epsilon^{\frac 12})\\
& \leq C (N^{-1} \max \norm{\psi'})^{2p}. 
\end{align*}
\end{beweis}

% \datum{29. Januar 2016}

\subsection{Reduktion der Räume}
\label{sec:reduktion-der-raume}

Die Konvergenzabschätzungen auf anisotropen Gittern basieren auf anisotropen Interpolationsfehlerabschätzungen, wie in Satz \ref{thm:7-7} für $Q_{p}(\tau)$. Allerdings benötigt der Raum $Q_{p}$ lokal $(p+1)^{2}$ und global $(pN+1)^{2} \sim p^{2} N^{2}$ Freiheitsgrade (entspricht Dimension der Matrix). Das resultiert in großem Speicher- und Rechenaufwand. In \cite{FM_ANM} werden Teilräume des $Q_{p}$ untersucht, die ebenfalls solche anisotropen Fehlerabschätzungen der gleichen Ordnung erfüllen und die stetig zusammensetzbar sind. Wir wollen die kleinste Klasse dieser Räume anschauen, die \markdef{Serendipityräume} oder auch \markdef{Rumpfräume}. 

Der bekannteste davon ist der quadratische $S_{2}$. Dieser wird über die Freiheitsgrade definiert als:   

\begin{figure}[ht!]
  \centering
  \begin{tikzpicture}
    \draw (-1, -1) -- (-1, 1) -- (1,1) -- (1, -1) -- cycle;

    \draw[fill] (-1,-1)  circle (1.5pt);
    \draw[fill] (1,-1)  circle(1.5pt);
    \draw[fill] (1,1) circle(1.5pt);
    \draw[fill] (-1,1) circle(1.5pt);
    \draw[fill] (0,-1)  circle (1.5pt);
    \draw[fill] (-1,0)  circle (1.5pt);
    \draw[fill] (0,1)  circle (1.5pt);
    \draw[fill] (1,0)  circle (1.5pt);
  \end{tikzpicture}
  \caption{Freiheitsgrade für $S_2$ }
  \label{fig:s2}
\end{figure}

Es fehlt zum vollen $Q_{2}$ nur ein Freiheitsgrad und damit eine Basisfunktion (Polynom, schwierig aufzuschreiben). Analog kann man das für den $q_{3}$ durchführen:


\begin{figure}[ht!]
  \centering
  \begin{tikzpicture}
    \def\x{1}
    \draw (-\x, -\x) -- (-\x, \x) -- (\x,\x) -- (\x, -\x) -- cycle;

    \draw[fill] (-\x,-\x)  circle (1.5pt);
    \draw[fill] (\x,-\x)  circle(1.5pt);
    \draw[fill] (\x,\x) circle(1.5pt);
    \draw[fill] (-\x,\x) circle(1.5pt);

    \draw[fill] (-0.33,-\x)  circle (1.5pt);
    \draw[fill] (-\x,-0.33)  circle (1.5pt);
    \draw[fill] (-0.33,\x)  circle (1.5pt);
    \draw[fill] (\x,-0.33)  circle (1.5pt);

    \draw[fill] (0.33,-\x)  circle (1.5pt);
    \draw[fill] (-\x,0.33)  circle (1.5pt);
    \draw[fill] (0.33,\x)  circle (1.5pt);
    \draw[fill] (\x,0.33)  circle (1.5pt);

  \end{tikzpicture}
  \caption{Freiheitsgrade für $S_3$ }
  \label{fig:s3}
\end{figure}


Für höhere Polynomgrade entstehen innere Freiheitsgrade, deshalb geht dieses Verfahren dort nicht mehr. Wir wollen den Raum $S_{p}$ über den aufgespannten Polynomraum definieren. Für diesen gilt
\begin{align*}
  S_{p}(\tau) &= \Span\set{ \set{1, x} \times \set{1, y, y^{2}, \dots, y^{p}} \cup \set{1, x, x^{2}, \dots, x^{p}} \times \set{1, y} \cup x^{2}y^{2} \cdot\cP_{p-4}(\tau)}\\
&= \Span\set{I \cup II \cup III}\\
&= \cP_{p}(\tau) \oplus \Span \set{(1+y)(1-y^{2})y^{p-2}, (1+y)(1-x^{2})x^{p-2}} \\
&= IV \oplus V.
\end{align*}
Es ist im Übrigen
\begin{align*}
  Q_{p} &= \Span \set{x^{i}\cdot y^{j}, 0 \leq i, j \leq p},\\
  P_{p} &= \Span \set{x^{i}\cdot y^{j}, 0 \leq i + j \leq p}. 
\end{align*}
Bildliche Darstellung: Ein Kästchen an der Position $(i, j)$ repräsentiert ein Monom $x^{i}y^{j}$ im Polynomraum. 

\begin{figure}[ht!]
  \centering
   \begin{tikzpicture}
    
 \newcommand*{\xMin}{0}%
 \newcommand*{\xMax}{6}%
 \newcommand*{\yMin}{0}%
 \newcommand*{\yMax}{6}%
 \def\Ncoarse {5};
 \def\Nfine {10};
 \def\lambday  {0.3};
 \def\lambdax {0.3};
 \def\h { (\xMax -\lambdax)/\Ncoarse};
 \def\picscale {0.00125};
 \pgfmathsetmacro\newvalue{\lambdax + \h * \picscale * \Nfine * \Nfine * \Nfine};

 \begin{scope}
   \foreach \i in {0,...,5} {
%     \pgfmathsetmacro\xcoord{(\lambdax + \h*\i*\i*\i) * \picscale};
     \draw [very thin,gray] (\i, \yMin) -- (\i, \yMax);% node [below] at (\xcoord,\yMin) {$\i$};
     \node[below] at (\i+0.5, 0) {\i};
   }
   \foreach \i in {0,...,5} {
%     \pgfmathsetmacro\xcoord{(\lambdax + \h*\i*\i*\i) * \picscale};
     \draw [very thin,gray] (\xMin, \i) -- (\xMax, \i);% node [below] at (\xcoord,\yMin) {$\i$};

     \node[left] at (0, \i+0.5) {\i};
   }

   \fill[red!30] (4, 3) rectangle (5, 4);

     \draw [very thin,gray] (\xMin, 6) -- (\xMax, 6);% node [below] at (\xcoord,\yMin) {$\i$};
     \draw [very thin,gray] (6, \xMin) -- (6, \xMax);% node [below] at (\xcoord,\yMin) {$\i$};

     \draw [very thin,black, ->, thick] (0, \yMin) -> (0, \yMax+1) node [left] at (0,\yMax+1) {$j$};
     \draw [very thin,black, ->, thick] (\xMin, 0) -> (\xMax+1, 0) node [below] at (\xMax+1,0) {$i$};

     \draw [very thin,black, thick] (4.5, 3.5) edge [bend left = 30, ->] (4.5, \yMax+0.5) node [below] at (4.5, \yMax + 1) {$x^{4}y^{3} \in Q_{5}$};
 \end{scope}


 \begin{scope}[right=8cm]

   \foreach \i in {0,...,5} {
%     \pgfmathsetmacro\xcoord{(\lambdax + \h*\i*\i*\i) * \picscale};
     \node[below] at (\i+0.5, 0) {\i};
   }
   \foreach \i in {0,...,5} {
%     \pgfmathsetmacro\xcoord{(\lambdax + \h*\i*\i*\i) * \picscale};

     \node[left] at (0, \i+0.5) {\i};
   }

   \draw [very thin,gray] (0, \yMin) -- (0, \yMax);% node [below] at (\xcoord,\yMin) {$\i$};
   \draw [very thin,gray] (1, \yMin) -- (1, \yMax);% node [below] at (\xcoord,\yMin) {$\i$};
   \draw [very thin,gray] (2, \yMin) -- (2, \yMax);% node [below] at (\xcoord,\yMin) {$\i$};
   \draw [very thin,gray] (3, \yMin) -- (3, \yMax-2);% node [below] at (\xcoord,\yMin) {$\i$};
   \draw [very thin,gray] (4, \yMin) -- (4, \yMax-3);% node [below] at (\xcoord,\yMin) {$\i$};
   \draw [very thin,gray] (5, \yMin) -- (5, \yMax-4);% node [below] at (\xcoord,\yMin) {$\i$};
   \draw [very thin,gray] (6, \yMin) -- (6, \yMax-4);% node [below] at (\xcoord,\yMin) {$\i$};


   \draw [very thin,gray] (\xMin, 0) -- (\xMax, 0);% node [below] at (\xcoord,\yMin) {$\i$};
   \draw [very thin,gray] (\xMin, 1) -- (\xMax, 1);% node [below] at (\xcoord,\yMin) {$\i$};
   \draw [very thin,gray] (\xMin, 2) -- (\xMax, 2);% node [below] at (\xcoord,\yMin) {$\i$};
   \draw [very thin,gray] (\xMin, 3) -- (\xMax-2, 3);% node [below] at (\xcoord,\yMin) {$\i$};
   \draw [very thin,gray] (\xMin, 4) -- (\xMax-3, 4);% node [below] at (\xcoord,\yMin) {$\i$};
   \draw [very thin,gray] (\xMin, 5) -- (\xMax-4, 5);% node [below] at (\xcoord,\yMin) {$\i$};
   \draw [very thin,gray] (\xMin, 6) -- (\xMax-4, 6);% node [below] at (\xcoord,\yMin) {$\i$};

   \draw [very thin,black, ->, thick] (0, \yMin) -> (0, \yMax+1) node [left] at (0,\yMax+1) {$j$};
   \draw [very thin,black, ->, thick] (\xMin, 0) -> (\xMax+1, 0) node [below] at (\xMax+1,0) {$i$};
   


%   \fill[pattern color = red!30, pattern=north east lines] (0, 0) rectangle (2, 6);

   \fill[color = red!30, opacity = 0.5] (0, 0) rectangle (2, 6);
   \fill[blue!30,opacity = 0.5] (0, 0) rectangle (6, 2);
   \fill[color = yellow!50, opacity = 0.5] (2, 2) rectangle (4, 3);
   \fill[yellow!50, opacity = 0.5] (2, 3) rectangle (3, 4);

   \fill[pattern color = black!70, pattern=north east lines] (3, 0) rectangle (4, 2);
   \fill[pattern color = black!70, pattern=north east lines] (5, 0) rectangle (6, 2);

   \fill[pattern color = black!70, pattern=north east lines] (0, 3) rectangle (2, 4);
   \fill[pattern color = black!70, pattern=north east lines] (0, 5) rectangle (2, 6);


%   \draw[color = red] (0, 0) rectangle (2, 6);
%   \draw[color = blue] (0, 0) rectangle (6, 2);
   \draw[color = orange, thick] (0,0) -- (6,0) --(6,1) --(5,1) -- (5,2) --(4,2) --(4, 3) -- (3,3) --(3,4) --(2,4) --(2,5) --(1,5) --(1,6) --(0,6) -- cycle;

 \end{scope}
\end{tikzpicture}

\caption{Veranschaulichung von $Q_{5}$ und $S_{5}(\tau)$}
\text{$I$ rot $II$ blau, $III$ gelb, $IV$ schwarz schraffiert, $V$ orange}
\label{fig:Q5_S5_tau}
\end{figure}


Der Serendipity-Raum $S_{p}$ reichert den $\cP_{p}$ um genau zwei Kanten-Blasenfunktionen an. Es gilt
\begin{align*}
  \cP_{p} \subset S_{p} \subset Q_{p}. 
\end{align*}
Die Anzahl der Freiheitsgrade ist

\begin{tabular}[ht!]{l l}
  lokal: & $\frac 12 (p+1)(p+2) + 2$\\
  global: & $\sim \frac 12 (p(p-1) +4)N^{2} \sim \frac 12 p^{2} N^{2}$.  
\end{tabular}

Das ist rund die Hälfte der Anzahl der Freiheitsgrade vom gesamten $Q_{p}$. Die Apel'sche Theorie von Lemma \ref{lem:7-6} und Satz \ref{thm:7-7} lässt sich auf die Räume $S_{p}$ anpassen. Wie sehen nun die geeigneten Funktionale aus?

Dazu betrachten wir zwei verschiedene Interpolatoren auf $\tau = [-1, 1]^{2}$. 

\paragraph{a) Lagrange-Interpolator}
\label{sec:lagr-interp}

Den Interpolator bezeichnen wir mit $Iu \in S_{p}$. Dieser ist definiert über die Bedingungen 

\begin{align*}
Iu (\pm 1, \pm 1) &= u(\pm 1, \pm1)\\
Iu (t_{i}, \pm 1) &= u(t_{i}, \pm1), \quad i = 1, \dots, p-1\\
Iu (\pm 1, t_{j}) &= u(\pm 1, t_{j}), \quad j = 1, \dots, p-1\\
Iu (t_{i}, t_{j}) &= u(t_{i}, t_{j}), \quad i = 1, \dots, p-3; \, j = 1, \dots, p-i-2\\
t_{i} = -1 + \frac {2i}p. 
\end{align*}
Das ergibt $4 + 2(p-1)+ 2(p-1)+ \frac 12 (p-3)(p-2) = \frac 12 ((p+1)(p+2) +4)$ Bedingungen. 

\begin{figure}[hb!]
  \centering
  \begin{tikzpicture}

    \foreach \i in {0,...,5} {
     \draw [very thin,gray!60] (\i, -1) -- (\i, 6);% node [below] at (\xcoord,\yMin) {$\i$};
     \draw (\i, 0) circle (1.5 pt);
     \draw (\i, 5) circle (1.5 pt);
   }
    \foreach \i in {0,...,5} {
     \draw [very thin,gray!60] (-1, \i) -- (6, \i);% node [below] at (\xcoord,\yMin) {$\i$};
     \draw (0, \i) circle (1.5 pt);
     \draw (5, \i) circle (1.5 pt);
   }

     \draw (1, 1) circle (1.5 pt);
     \draw (2, 1) circle (1.5 pt);
     \draw (1, 2) circle (1.5 pt);
     \node[below left] at (0, 0) {$t_{1}$};
     \node[below right] at (5, 0) {$t_{6}$};

     \draw (0, 0) rectangle (5, 5);
  \end{tikzpicture}
  \caption{Lagrange $S_5$}
  \label{fig:lagrange_s5}
\end{figure}


Die für $\gamma = (0, 0)$ (siehe vorherige Kapitel), das heißt $\nnorm{u - Iu}$ benötigten Funktionale entsprechen denen der Definition von $I$. Für $\gamma = (0, 1)$, das heißt $\nnorm{(u - Iu)_{y}}$ nutzen wir
\begin{align*}
  N_{j}^{\pm}(v) &= \int_{-1}^{t_{j}} v(\pm 1, y) dy, \quad j = 1, \dots, p\\
  N_{ij}(v) &= \int_{-1}^{t_{j}} v(t_{i}, y) dy, \quad i = 1, \dots, p-1, j = p+1 \text{ oder } i = 1, \dots, p-3, j = 1, \dots, p-i-2. 
\end{align*}
Das sind offensichtlich stetige lineare Funktionale. Sei $N$ ein beliebiges davon, das heißt
\begin{align*}
  N(v) &= \int_{-1}^{b} v(a, y) dy, 
\end{align*}
mit $a = t_{i}$ für $i \in \set{0, 1, \dots, p}$ und $b = t_{j}$ für $j \in \set{1, \dots, p}$. Dann
\begin{align*}
  N((Iv - v)_{y}) = \int_{-1}^{b} (Iv - v)_{y} dy = (Iv-v)(a, y)|_{y = -1}^{b} = 0. 
\end{align*}
Da alle Funktionale Integrale entlang unterschiedlicher Linien sind, sind sie linear unabhängig. Mit
\begin{align*}
  \dim(\partial y S_{p}) = \dim(S_{p})- (p+1) = \frac{p(p+2)}2 + 2 = 2p + (p-1) + \frac {(p-3)(p-2)}2 = \# \set{N_{j}^{\pm}} + \# \set{N_{ij}}
\end{align*}
gelten die Voraussetzungen des Lemmas \ref{lem:7-6}. 

\paragraph{b) Momenten-basierter Operator}

$Ju \in S_{p}$ ist definiert über
\begin{align*}
  Ju (\pm 1, \pm 1) &= u(\pm 1, \pm 1)\\
\int_{l_{i}} (Ju- u)\cdot q &= 0, \quad i =1, \dots, 4, q \in \cP_{p-2}(l_{i})\\
\int_{\tau}(Ju - u)\cdot q &= 0, \quad q \in \cP_{p-4}(\tau). 
\end{align*}
Für $\gamma = (0, 0)$ funktionieren die Funktionale wie in der Definition von $J$. Für $\gamma = (1, 0)$ nutzen wir die Funktionale 
\begin{align*}
  N_{i}^{\pm} v &= \int_{-1}^{1} v(x, \pm 1)x^{i} dx , \quad i = 0, \dots, p-1\\
  N_{ij} v &= \int_{\tau} vx^{i}y^{j}, \quad x^{i}y^{j} \in \partial_{x} \partial^{2}_{y} S_{p} 
\end{align*}
mit
\begin{itemize}
\item stetigen linearen Funktionalen $v$, 
\item Unter Benutzung von partieller Integration
  \begin{align*}
      N_{i}^{\pm} ((Jv-v)_{x}) &= \int_{-1}^{1} (Jv - v)_{x} (x, \pm 1)x^{i} dx = \underbrace{(Jv- v)(x, \pm 1)|_{x = -1}^{1}}_{=0} - \underbrace{i \int_{-1}^{1} (Jv - v)(x, \pm1)x^{i-1} dx}_{=0} = 0\\
  N_{ij} ((Jv-v)_{x}) &= \int_{-1}^{1}\int_{-1}^{1}(Jv - v)_{x}x^{i}y^{j} dxdy \\
&=\underbrace{\int_{-1}^{1} (Jv - v)(x, y)x^{i}|_{x = -1}^{1} y^{j} dxdy}_{=0} - \underbrace {i \int_{\tau} (Jv - v)x^{i-1}y^{j}}_{=0}. 
  \end{align*}
da $y^{j}\in \cP_{p-2}$ und $x^{i-1}y^{j} \in \partial_{x}^{2} \partial_{y}^{2} S_{p} = \cP_{p-4}$.  
\item Aus $N_{i}^{\pm}v = 0$ für $i = 1, \dots, p-1$ folgt $v|_{\norm y = 1} = 0$ und somit $v = (1 - y^{2})q, q \in \partial_{x} \partial_{y}^{2} S_{p}$. 

Außerdem $N_{ij}v = 0$, $ij$ mit $x^{i}y^{j} \in\partial_{x} \partial_{y}^{2} S_{p}$ genau dann, wenn $\int_{\tau} (1 - y^{2})q^{2} = 0 \implies q = 0 \implies v = 0. $
\end{itemize}
Also ist auch hier Lemma \ref{lem:7-6} anwendbar.

%% ENDE

%Prüfung: Keine Beweise runterbeten, Anisotropie, Abschätzungen

%%% local Variables: 
%%% mode: latex
%%% TeX-master: "vorlesung"
%%% End: 
