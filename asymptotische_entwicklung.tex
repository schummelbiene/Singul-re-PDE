
\section{Asymptotische Entwicklungen}
\label{sec:asympt-entw}
Mit dieser Technik versuchen wir, eine gegebene Funktion mit einer Einfacheren zu approximieren. Dabei werden wir eine Art 'Potenzreihenentwicklung' in $\eps$ ausführen. Wir starten zunächst mit Problemen, die nicht singulär gestört sind.
\begin{beispiel}\label{ex:3-1}
  Betrachten wir die Lösung der algebraischen Gleichung
  \begin{align}\label{eq:3-1}
    u^{2} + \eps u -1 = 0, \quad 0 <\eps \ll 1.
  \end{align}
Falls $\eps = 0$, so folgt $u = 1$ als eine Lösung. Falls $\eps > 0$ erwarten wir eine Lösung nahe $1$. Wir nutzen den Ansatz
\begin{align*}
  u_{\eps} = 1 + \eps u_{1} + \eps^{2} u_{2} + \dots  \sum_{i = 0}^{n} \eps^{i} u_{i}, \quad u_{0} = 1
\end{align*}
Setzen wir diesen Ansatz in \eqref{eq:3-1} ein, so erhalten wir formal
\begin{align*}
  0 &= (1 + \eps u_{1} + \eps^{2} u_{2} + \dots )^{2} + \eps (1 + \eps u_{1} + \eps^{2} u_{2} + \dots ) - 1\\
  &= 1 + 2 \eps u_{1} + \eps^{2} (2u_{2} + u_{1}^{2}) + \dots +  \eps + \eps^{2} u_{1} + \eps^{3} u_{2} + \dots - 1  
\end{align*}
Koeffizientenvergleich:
\begin{align}\label{eq:3-2}
  &\eps^{0}: \quad 1 -1 = 0\notag\\
  &\eps^{1}: \quad 2u_{1} + 1 = 0 \implies u_{1} = -\frac 1 2\notag\\
  &\eps^{2}: \quad 2u_{2} + u_{1}^{2} + u_{1} = 0 \implies u_{2} = \frac 1 8\notag\\
  & \vdots \notag\\
\implies \quad & u_{\eps} = 1 - \frac 12 \eps + \frac 1 8 \eps ^{2} \dots
\end{align}
Dies ist eine (formale) asymptotische Entwicklung von $u$. Das heißt nicht, dass \eqref{eq:3-2} die Gleichung \eqref{eq:3-1} löst. Dafür müssen wir noch nachweisen, dass \eqref{eq:3-2} wirklich eine Lösung ist. 

Lösung und dann Taylorentwicklung zu $\eps = 0$:
\begin{align*}
  u &= \frac {- \eps + \sqrt{\eps^{2} + 4}} 2\\
  &= - \frac \eps 2 + \frac 12 [2 + 0\cdot \eps + \frac 1 4 \eps^{2} \pm \dots] \\
  &= 1 - \frac \eps 2 + \frac {\eps^{2}} 8 +\dots
\end{align*}
Alternative: auf anderem Weg zeigen, dass $u_{\eps} - u = 0$ (zum Beispiel: Binomische Formeln).
\end{beispiel}
Wir werden im Folgenden die Landau Symbole $\cO(\cdot)$ und $o(\cdot)$ verwenden, das heißt,
\begin{align*}
  f(\eps) = \cO(g(\eps)) \iff \norm{f(\eps)}\leq c \norm{g(\eps)} \quad \forall \norm \eps \in (0, \eps_{0})\\
  f(\eps) = o(g(\eps)) \text{ für }\eps \to 0 \iff \lim_{\eps \to 0} \norm{\frac {f(\eps)}{g(\eps)}} = 0
\end{align*}
\begin{definition}\label{def:3-2}
  Eine \markdef{asymptotische Sequenz} $\set{\phi_{n}(\eps)}_{n = 1}^{\infty}$ ist eine Folge von Funktionen in $\eps$, sodass
  \begin{align*}
    \phi_{n+1} (\eps) = o(\phi_{n}(\eps)) \text{ für }\eps \to 0 \quad \forall n \in \N. 
  \end{align*}
\end{definition}
\begin{beispiel*}
  \begin{enumerate}
  \item $1, \eps, , \eps^{2}, \eps^{3}, \dots$
  \item $\ln \eps, 1,\eps \ln \eps, \eps, \eps^{2}\ln \eps, \dots$
  \end{enumerate}
\end{beispiel*}
\begin{definition}\label{def:3-3}
  Es sei $u(x, \eps)$ für alle $x \in X$ und $\eps$ ausreichend klein definiert. Sei $\set{\phi_{n}(\eps)}$ eine asymptotische Sequenz. Dann nennen wir die Reihe
  \begin{align*}
    \sum_{n = 1}^{N} u_{n}(x)\phi_{n}(\eps)
  \end{align*}
mit $N$ (un)endlich eine \markdef{asymptotische Entwicklung} von $u$ bezüglich $\set{\phi_{n}}$ für $\eps \to 0$, falls für jedes $M \in \set{1, 2, \dots, N}$ gilt:
\begin{align}\label{eq:3-3}
  u(x, \eps) - \sum_{i = 1}^{M} u_{n}(x)\phi_{n}(\eps) = o(\phi_{M}) \quad \text{für } \eps \to 0. 
\end{align}
alternative Notation: $u(x, \eps) \sim \sum_{n =1}^{M}u_{n}(x)\phi_{n}(\eps)$. Die Entwicklung heißt \markdef{gleichmäßig}, wenn \eqref{eq:3-3} gleichmäßig in $x$ gilt. 

Ist $u$ und $\set{\phi_{n}}$ gegeben, so ist die asymptotische Entwicklung eindeutig. Die Koeffizienten können (im Prinzip) wie folgt bestimmt werden:
\begin{align*}
  u_{1}(x) &= \lim_{\eps \to 0} \frac{u(x, \eps)}{\phi_{1}(\eps)}\\
  u_{m}(x) &= \lim_{\eps \to 0} \frac{u(x, \eps) - \sum_{n = 1}^{m-1} u_{n}(x) \phi_{n}(\eps)}{\phi_{m}(\eps)}, \quad m > 1.
\end{align*}
Allerdings ist eine gute Wahl von $\set{\phi_{n}}$ nicht immer klar. 
\end{definition}
\begin{beispiel}\label{ex:3-4}
  Betrachte das Anfangswertproblem
  \begin{align*}
    \frac{dy}{dz} + y = \frac 1 \eps, \quad - \infty < z <0\\
    y(- \infty) = 0. 
  \end{align*}
Gesucht ist die asymptotische Entwicklung für $\norm z \gg 1$, $z <0$, also $\eps = \frac 1 z$. Genutzt wir die asymptotische Sequenz $\set{\phi_{n}}$ mit $\phi_{1} = 1$, $\phi_{2} = \frac 1 z$, $\phi_{3}= \frac 1 {z^{2}}$. 
%\datum{23. Oktober 2015}

Nutzen wir die asymptotische Sequenz $\phi_{n}$ mit
\begin{align*}
  \phi_{1} &= 1, \quad \phi_{2} = \frac 1z, \quad \phi_{3} = \frac 1{z^{2}}\ \dots\\
\implies \quad y(z) &= \sum_{m = 0}^{\infty} u_{m} z^{-m}\\
0 &= y(- \infty) = u_{0}
\end{align*}
Ableiten nach $z$: 
\begin{align*}
  -\sum_{m = 0}^{\infty} m\cdot u_{m} z^{-m+1} +\sum_{m = 1}^{\infty} u_{m} z^{-m} = z^{-1}\\
\end{align*}
Koeffizientenvergleich bezüglich $\phi_{n}$:
\begin{align*}
%  z^{-1}&:\, u_{1}  = 1\\
%  z^{-2}&:\, -u_{1} + u_{2} = 0 \implies u_{2} = u_{1} = 1\\
%  z^{-3}&:\, -2\cdot u_{2} + u_{3} = 0 \implies u_{3} = 2\cdot u_{2} = 2 \dots\\
& u_{m} = (m-1)!\quad m\geq 1, u_{0} = 0\\
\implies \quad y(z) &= \sum_{m = 1}^{\infty} \frac{(m-1)!}{z^{m}}
\end{align*}
ist die formale Lösung. Die Richtigkeit muss noch gezeigt werden! Lösen wir das Anfangswertproblem direkt: 

Homogene Gleichung:
\begin{align*}
  y' + y = 0\implies y_{h}(z) = c\cdot e^{-z}
\end{align*}
Partielle Lösung (Variation der Konstanten):
\begin{align*}
  y_{p}(z) &= c(z)e^{-z}\\
  y_{p}'(z) &= c(z)'e^{-z} - c(z)e^{-z}\\
  y_{p}'(z) + y_{p}(z) &= c(z)'e^{-z}  = z^{-1}\\
\implies \quad c(z) &= \int \frac {e^{z}} z dz\\
\implies \quad y(z) &= c e^{-z} + e^{-z} \int_{- \infty}^{z} \frac{e^{t}}{t}dt\\
 y(-\infty)&= 0 \implies c = 0\\
\implies \quad y(z) &= e^{-z} \int_{- \infty}^{z} \frac{e^{t}}{t}dt, 
\end{align*}
das Integral konvergiert für alle $z < 0$. Geschickte Umformung per partieller Integration führt auf die formale Darstellung:
\begin{align*}
  y(z) &= z^{-1} + e^{-z} \int_{-\infty}^{z} t^{-2} e^{t}dt\\
&= z^{-1} + z^{-2} + 2e^{-z} \int_{-\infty}^{z} t^{-3} e^{t}dt = \dots 
\end{align*}
Für jedes $M> 0$ bekommen wir
\begin{align}\label{eq:3-4}
  y(z) - \sum_{m = 1}^{M}(m-1)! z^{-m} = M! \cdot e^{-z} \int_{-\infty}^{z}t^{-(M+1)}e^{t}dt
\end{align}
Für $z <0$ folgt:
\begin{align}\label{eq:3-5}
  \norm {M! \cdot e^{-z} \int_{- \infty}^{z} t^{-(M+1)} e^{t}dt}&\leq M!\cdot \norm z^{-(M+1)}\underbrace{e^{-z}\cdot\int_{-\infty}^{z}e^{t}dt}_{= 1}\\
&= o(z^{-M})\notag
\end{align}
für $z \to -\infty$. Also gilt $y(z) \sim \sum_{m = 1}^{\infty} \frac{(m-1)!}{z^{m}}$ für $z \to -\infty$.
\end{beispiel}
Es bleibt zu beachten, dass die asymptotischen Entwicklungen und Taylor-Reihen nich viel gemeinsam haben. Es lässt sich leicht zeigen, dass die obige Reihe
\begin{align*}
  \sum_{ m = 1}^{\infty} \frac{(m-1)!}{z^{m}}
\end{align*}
für jedes $z < 0$ divergiert! Das stört nicht; trotzdem ist die Entwicklung nützlich, da wir nie bis $\infty$ summieren. Dafür ist diese Reihendarstellung sehr nützlich. Üblicherweise werden nur eine Hand voll Terme genutzt. Setzen wir im Beispiel $z  = - 10$ ein. Dann folgt aus \eqref{eq:3-4} und \eqref{eq:3-5}
\begin{align*}
  \norm{y(z) - \sum_{m = 1}^{2}\frac{(m-1)!} {z^{-m}}} \leq 2! \cdot 10^{-3}, 
\end{align*}
also eine sehr gute Approximation von $y(-10)$ mit $2$ Termen. Typischerweise verhalten sich die Terme wie in Abbilung \ref{fig:terms} dargestellt. 

\begin{figure}[ht!]

\tikzset{cross/.style={cross out, draw, 
         minimum size=2*(#1-\pgflinewidth), 
         inner sep=0pt, outer sep=0pt}}
  \centering
  \begin{tikzpicture}
    \begin{axis}[width=10cm, height=7cm, yticklabels = {,,}, 
ytick=\empty, 
ymax=7.5,ymin=0,xmin=0.5, xmax = 8.5,  xticklabels={1,2, 3, 4, 5, 6, 7, 8}, xtick={1, ..., 8},axis y line = left, axis x line = bottom]
      \addplot[ybar, fill=gray!60] plot coordinates {(1,6) (2,4.5) (3,2.7) (4, 1.7) (5, 2) (6, 3.3) (7, 4.8) (8, 6)};
    \end{axis}
      \draw [thick, ->] (1.5,4) -- (3.5, 1.5){};
      \draw [thick, ->] (4.1,1.5) -- (6.6, 4)node[cross=7pt, rotate=35, orange]at (5.35, 2.75){};
%      \node [cross] at (6, 2.75){};
  \end{tikzpicture}
  \caption{Termverhalten}
  \label{fig:terms}
\end{figure}

\paragraph{Rechenregeln}
\label{sec:rechenregeln}
\begin{enumerate}
\item 
Mit $o(\cdot)$ kann auch gerechnet werden. Gilt 
\begin{align*}
  &f(\eps) \sim \sum_{n =1}^{\infty}a_{n}(x)\phi_{n}(\eps) \quad \eps \to \infty\\
  &g(\eps) \sim \sum_{n =1}^{\infty}b_{n}(x)\phi_{n}(\eps) \quad \eps \to \infty
\end{align*}
so folgt
\begin{align*}
    (f\pm g)(\eps) \sim \sum_{n =1}^{\infty}(a_{n}\pm b_{n})(x)\phi_{n}(\eps) \quad \eps \to \infty. 
\end{align*}
\item Produkte sich auch möglich, insofern in den entstehenden Reihen ungeordnet werden kann (absolut konvergente Reihen). Asymptotische Entwicklungen ineinander einzusetzen ist gefährlich. 
\item Es darf termweise integriert werden. 
\item Termweises Differenzieren ist im Allgemeinen nicht erlaubt. 
\end{enumerate}
\begin{beispiel}\label{ex:3-5}
  Betrachten wir das Randwertproblem
  \begin{align}\label{eq:3-6}
    Lu = - u''(x) + (k^{2} + \eps w(x)) u(x) = f(x)&\In (0, 1)\\
    u(0) = u(1) = 0 &\notag
  \end{align}
  mit festem $k> 0$ und $w, f \in C[0, 1]$, außerdem sei sei $0< \eps \ll 1$. Dann gilt nach Satz \ref{thm:2-3} und Lemma \ref{lem:2-6}, dass \eqref{eq:3-6} eine eindeutige Lösung $u(x, \eps)$ besitzt. Für diese wollen wir eine (kurze) asymptotische Entwicklung herleiten, das heißt
\begin{align*}
  u(x, \eps) = u_{0}(x) + \eps u_{1}(x) + \cO(\eps^{2}). 
\end{align*}
Eingesetzt in \eqref{eq:3-6} und Vergleich der $\eps$-Potenzen liefert:
\begin{align}\label{eq:3-7}
  \eps^{0}:& - u_{0}'' + k^{2} u_{0} = f\\
  &u_{0}(0) = u_{0}(1) = 0 \notag 
\end{align}
\begin{align}\label{eq:3-8}
  \eps^{1}:& - u_{1}'' + k^{2} u_{1} = -w(x) u_{0}(x)\\
  &u_{1}(0) = u_{1}(1) = 0 \notag. 
\end{align}
\eqref{eq:3-7} besitzt eine eindeutige Lösung $u_{0}$, welche in \eqref{eq:3-8} eingesetzt werden kann. Also besitzt \eqref{eq:3-8} eine eindeutige Lösung $u_{1}$. 

Beispiel im Beispiel: $k = 1$, $f(x) = 2\cdot(x-1) \sin(x) - 2\cdot \cos (x)$ und $w = 1$. Es folgt
\begin{align*}
  u_{0}(x) &= (x-1) \sin(x)\\
  \implies \quad u_{1}(x) &= \dots
\end{align*}
(ein bisschen aufwändig, aber berechenbar).

\vspace{5mm}

Wie zeigen wir, dass $u(x) = u_{0}(x) + \eps u_{1}(x) + \cO(\eps^{2})$? Das ist äquivalent zu
\begin{align*}
  w(x) = u(x) - (u_{0}(x) + \eps u_{1}(x)) = \cO(\eps^{2}). 
\end{align*}
Die Randdaten $w(0)$, $w(1)$ mit dem Residuum $(Lw)(x)$ liefern uns mit Schrankenfunktionen eine Abschätzung $\norm q \leq \dots$

\vspace{5mm}

Wir wollen das Maximumprinzip für $L$ nutzen, also muss $\eps$ klein genug sein, damit $k^{2} + \eps w(x) \geq 0$. Nach Definition:
\begin{align*}
  (Lw)(x) &= (Lu)(x) - ((Lu_{0})(x) + \eps(Lu_{1})(x))\\
  &= f(x) - ((f + \eps w(x) u_{0}(x)) + \eps(-w(x)u_{0}(x) + \eps w(x)u_{1}(x)))\\
  &= \eps^{2} w(x) u_{1}(x)\\
\implies \quad \norm{(Lw)(x)}&\leq \eps^{2}\nnorm w_{L_{\infty}} \nnorm {u_{1}}_{L_{\infty}} \leq C \eps ^{2}
\end{align*}
mit $C \geq \nnorm w_{L_{\infty}} \nnorm {u_{1}}_{L_{\infty}}$. Randbedingung: 
\begin{align*}
  w(0) = (u- (u_{0} + \eps u_{1}))(0) = 0=   (u- (u_{0} + \eps u_{1}))(1) = w(1). 
\end{align*}
Sei nun $\eps$ so klein, dass $k^{2} + \eps w(x) \geq \frac {k^{2}}2$ und die Schrankenfunktion $b(x) \frac{2C}{k^{2}}\eps^{2}$. Dann ist $b$ konstant und
\begin{align*}
  (Lb)(x) &= - b'' + (k^{2} + \eps w(x))b(x) \geq \frac {k^{2}} 2 \frac{2C}{k^{2}} \eps^{2} \geq \norm{(Lw)(x)}\\
  b(0) = b(1) &= \frac{2C}{k^{2}} \eps^{2} \geq 0 = w(0) = w(1)\\
\implies \norm{w(x)} &\leq \frac{2C}{k^{2}}\eps^{2}\\
\implies u(x) &= u_{0} + \eps u_{1} + \cO(\eps^{2})
\end{align*}
für $\eps \to 0$. Diese asymptotische Entwicklung ist gleichmäßig in $x \in (0, 1)$. 
\end{beispiel}
Maximumprinzip: Aus der Kleinheit der Daten folgt die Kleinheit der Lösung. 

Mit dieser Technik können auch höhere Entwicklungen durchgeführt werden, es werden nur mehr Terme benötigt. Also hat man regulär gestörte Probleme im Griff *check*!
%%% Local Variables: 
%%% mode: latex
%%% TeX-master: "vorlesung"
%%% End: 
