
\section{Asymptotische Entwicklung für singulär gestörte Probleme}
\label{sec:asympt-entw-fur}

Sei nun wieder
\begin{align}\label{eq:4-1}
  (Lu)(x)\coloneqq - \eps u''(x) - b(x)u'(x) + c(x)u(x) = f(x) &\In (0, 1)\\
  u(0) = A, \, u(1) = B &\notag
\end{align}
mit $b(x)\geq \beta > 0$ und $c(x)\geq0$, $b, c, f \in C[0, 1]$. Nach Satz \ref{thm:2-3} und Lemma \ref{lem:2-6} exisitert eine eindeutige Lösung $u(x) = u(x, \eps)$ für jedes $\eps> 0$. 
\begin{lemma}\label{lem:4-1}
  Für die Lösung $u$ von \eqref{eq:4-1} existiert eine generische Konstante $C> 0$, sodass
  \begin{align*}
    \nnorm u_{L_{\infty}} \leq C \text{ und } \norm{u'(1)} \leq C. 
  \end{align*}
\end{lemma}
\begin{bemerkung*}\markdef{Generische Konstanten} $C, C_{1}, C_{2}, \dots$ dürfen an jeder Stelle, an denen sie auftauchen, einen anderen Wert annehmen, sind aber immer unabhängig von $\eps$.

  Zum Beispiel: $\nnorm{u}_{L_{\infty}} \leq C$, $\nnorm{u_{h}}_{L_{\infty}} \leq C$, $\nnorm{u - u_{h}}_{L_{\infty}} \leq \nnorm{u}_{L_{\infty}} + \nnorm{u_{h}}_{L_{\infty}}  \leq C$. Der Schreibaufwand ist wesentlich geringer, es wird allerdings uneindeutiger. Der Wert dieser Konstanten ist allerdings unwichtig, die $\eps$-Abhängigkeit zählt bei den Betrachtungen viel mehr! 
\end{bemerkung*}

% \datum{29. Oktober 2015}

\begin{beweis}
  $z(x) = u(x) -B$
  \begin{align*}
    \norm{z(0)} = \norm{A-B}\\
    \norm{z(1)} = 0\\
    \norm{(Lz)(x)} =     \norm{(Lu)(x) - (LB)(x)} = \norm{f - cB}\leq \nnorm f_{L_{\infty}} + \norm B \nnorm c_{L_{\infty}}. 
  \end{align*}

  Schrankenfunktionen: 

  \begin{align*}
    S(x) &= \frac{1- x}\beta \( \beta \norm {A-B} + \nnorm f_{L_{\infty}} + \norm B \nnorm c_{L_{\infty}}\)\\
    S(0) &= \norm {A-B} \frac{\nnorm f_{L_{\infty}} + \norm B \nnorm c_{L_{\infty}}} \beta \geq \norm{A-B} = \norm{z(0)}\\
    S(1) &= 0 \geq \norm{z(1)}\\
    (LS)(x) &= \frac{b(x)}\beta \( \beta \norm {A-B} + \nnorm f_{L_{\infty}} + \norm B \nnorm c_{L_{\infty}}\) + c(x)(1-x)\( \beta \norm {A-B} + \nnorm f_{L_{\infty}} + \norm B \nnorm c_{L_{\infty}}\)\\
    &\geq  \nnorm f_{L_{\infty}} + \norm B \nnorm c_{L_{\infty}}\geq \norm{(Lz)(x)}\\ 
    \implies \quad & \norm{z(x)} \leq S(x) \leq \norm {A-B} \frac{\nnorm f_{L_{\infty}} + \norm B \nnorm c_{L_{\infty}}} \beta\\
    \nnorm{u(x)}&\leq \norm{z(x)} + B \leq C
  \end{align*}
  Außerdem gilt
  \begin{align*}
    \norm{u'(1)} &=   \norm{z'(1)} = \lim_{x \to 1} \norm{\frac{z(1) - z(x)}{1 - x}}\\
    &= \lim_{x \to 1} \frac{z(x)}{1-x} \leq \lim_{x \to 1} \frac{S(x)}{1-x}\leq C
  \end{align*}
\end{beweis}

Aus Lemma \ref{lem:4-1} folgt, dass $u$ unabhängig von $\eps$ beschränkt ist und keine Grenzschicht bei $x = 1$ besitzt. Wäre $Lu = - \eps u'' + bu' + cu = f$, so besäße $u$ keine Grenzschicht bei $x = 0$. Daraus folgt, dass das Vorzeichen von $b$ den Ort der Grenzschicht definiert. 

Vorzeichenwechsel von $b$ kann innere Grenzschichten hervorrufen (Wendepunktprobleme). Hier wird dieser Fall nicht betrachtet. Bevor wir mit der asymptotischen Entwicklung des Randwertproblems \eqref{eq:4-1} beginnen, betrachten wir eine algebraische Gleichung. 
\begin{beispiel}\label{ex:4-2} Fortsetzung von Beispiel \ref{ex:3-1}

  Gegeben
  \begin{align}\label{eq:4-2}
    \eps u^{2} + u - 1 = 0, \quad 0 < \eps \ll 1. 
  \end{align}
  \begin{enumerate}
  \item $\eps = 0$: Die Gleichung \eqref{eq:4-2} besitzt genau eine Nullstelle $u = 1$. 
  \item $\eps > 0$: Die Gleichung \eqref{eq:4-2} besitzt zwei Nullstellen
    \begin{align*}
      u_{1/2} &= \frac{- 1 \pm \sqrt{1 + 4 \eps}}{2 \eps}\\
      u_{1} &= \frac{2}{1 + \sqrt{1 + 4 \eps}} \\
      u_{2} &= - \frac{1 + \sqrt{1 + 4 \eps}}{2 \eps} 
    \end{align*}
    Taylorentwicklung für $\eps = 0$:
    \begin{align*}
      u_{1} &= 1 - \eps + 2 \eps^{2} - 5 \eps^{3}+ \dots\\
      u_{2} &=  - \frac 1\eps + 1 + \eps - 2 \eps^{2} +  \dots
    \end{align*}
    Würden wir den Ansatz $u = \sum_{n= 0}^{\infty} a_{n} \eps^{n}$ wählen und in \eqref{eq:4-2} einsetzen, so würden wir $u_{1}$ finden, aber nicht $u_{2}$. Hierfür müssten wir $u = \sum_{n= -1}^{\infty} a_{n} \eps^{n}$ nutzen. 
  \end{enumerate}
\end{beispiel}

\begin{beispiel}\label{ex:4-3}
  \begin{align}\label{eq:4-3}
    - \eps u^{2} -u'(x)  &= f(x) \In (0, 1)\\
    u(0) &= u(1) = 0 \notag
  \end{align}
  mit $0 < \eps \ll 1$ und $f$ glatt. Hier und in Zukunft ist $u(x) = u(x, \eps)$. Setzen wir $\eps = 0$ in \eqref{eq:4-3}, so erhalten wir eine Differentialgleichung erster Ordnung. Probieren wir den Ansatz 
  \begin{align*}
    u(x) = \sum_{n = 0}^{\infty} a_{n}(x) \eps^{n}. 
  \end{align*}
  Aus den Randbedingungen folgt dann $a_{n}(0) = a_{n}(1) = 0$, $n = \N_{0}$. Einsetzen in \eqref{eq:4-3} liefert
  \begin{align*}
    - \eps \cdot \sum_{n = 0}^{\infty} a_{n}''(x) \eps^{n} - \sum_{n = 0}^{\infty} a_{n}'(x) \eps^{n} = f(x). 
  \end{align*}
  \begin{align*}
    \eps^{0} &: -a_{0}'(x) = f(x)\\
    \eps^{1} &: -a_{1}'(x) = a_{0}''(x)\\
    \eps^{2} &: -a_{2}'(x) = a_{1}''(x)\\
    &\dots
  \end{align*}
  Für eine Differentialgleichung erster Ordnung benötigen wir noch eine zusätzliche Bedingung: $a_{0}(0)a_{0}(1) = 0$ sind zwei Bedingungen. Das Problem setzt sich für DGLs höherer Ordnung fort...

  Nach Lemma \ref{lem:4-1} besitzt $u$ bei $x = 1$ keine Grenzschicht. Daher nehmen wir als zusätzliche Bedingung $a_{n}(1) = 0$. ($a_{n}(0) = 0$ würde nicht als Grenzschicht funktionieren, da die Grenzschicht nicht die Struktur besitzt.) Wir erhalten
  \begin{align*}
    a_{0}(x) &= \int_{x}^{1}f(t) dt = [F(1) - F(x)]\\
    a_{1}(x) &= -[f(1) - f(x)]\\
    a_{2}(x) &= x[f'(1) - f'(x)]\\
    &\dots
  \end{align*}
  für $F(x) = -\int_{x}^{1}f(t) dt$. $u$ können wir nun darstellen als
  \begin{align*}
    u(x) &\approx \sum_{n = 0}^{\infty} (-1)^{n}[F^{(n)}(1) - F^{(n)}(x)]\eps^{n}. 
  \end{align*}
  Schauen wir uns dies für den einfachen Fall $f(x) = 1$ an:
  \begin{align*}
    u(x) \approx 1- x. 
  \end{align*}
  Die exakte Lösung ist 
  \begin{align*}
    u(x) &= \frac{1 - e^{-\frac x \eps}}{1 - e^{-\frac 1 \eps}} - x\\
    &= 1 - x + \frac{e^{- \frac 1 \eps} - e^{- \frac x \eps}}{1 - e^{- \frac 1 \eps}} \leq 1 - x + C(x)\eps^{M} 
  \end{align*}
  für beliebiges $M > 0$. Leider hängt $C$ von $x$ ab:
  \begin{align*}
    e^{- \frac x \eps} &= \eps^{M} \eps^{-M}e^{- \frac x \eps}\\
    &= \eps^{M} x^{-M} \( \frac x \eps\)^{M} e^{- \frac x \eps}\\
    &\leq \eps^{M} x^{-M} \( \frac M e\)^{M}\leq C \cdot x^{-M} \eps^{M}
  \end{align*}
  $C$ ist nicht gleichmäßig beschränkt. Das ist schlecht. 
  Also war die Näherung nur für $x > 0$ gut. Näherung von $u$ führt weg von der Grenzschicht.
\end{beispiel}
Die vorherigen Beispiele zeigen, dass die Standart-Ansätze modifiziert werden müssen. Für Randwertprobleme bedeutet dies, dass weitere 'Korrekturfunktionen' hinzugefügt werden müssen. Diese werden \markdef{Grenzschichtkorrekturen} genannt. 

\begin{beispiel*} Fortsetzung von \ref{ex:4-3}

  Wir haben bereits
  \begin{align*}
    u_{0}(x) = \sum_{n = 0}^{\infty} a_{n}(x)\eps^{n}, \quad a_{n}(x) = (-1)^{n}[F^{(n)}(1) - F^{(n)}(x)]. 
  \end{align*}
  Hierbei lösen die Koeffizienten
  \begin{align}
    (L_{0}a_{0})(x) = - a_{0}'(x) = f(x), \quad a_{0}(1) = 0\\
    (L_{0}a_{n})(x) = - a_{n}'(x) = a_{n-1}'', \quad a_{n}(1) = 0
  \end{align}
  für $n > 0$. Diese Probleme werden das \markdef{reduzierte Problem} genannt. Um die Grenzschicht bei $x = 0$ aufzulösen, definieren wir uns eine neue \markdef{gestreckte Variable} $\xi = \frac {x}{\eps^{\alpha}}$. Die Potenz $\alpha> 0$ wird so bestimmt, dass $L$ in der neuen Variable ausgedrückt nicht mehr singulär gestört ist. Sei $\tilde u(\xi) = u(x)$.
  \begin{align*}
    u'(x) &= \frac{du}{dx} =  \frac{d\tilde u}{d \xi}\frac{d\xi}{dx} = \tilde u' \eps^{-\alpha}, \\
    u''(x) &= \tilde u'' \eps^{-2\alpha}\\
    \implies \quad (Lu)(x) &= - \eps u'' - u' = - \eps^{1-2\alpha} \tilde u'' - \eps^{-\alpha} \tilde u' = (\tilde L \tilde u)(\xi). 
  \end{align*}
  Falls $1 - 2 \alpha = -\alpha$ ($\alpha = 1$), dann erhalten wir $\xi = \frac x \eps$ und 
  \begin{align*}
    (\tilde L \tilde u)(\xi) = - \eps^{-1}[\tilde u'' + \tilde u']
  \end{align*}
  und
  \begin{align*}
    u(x) = u_{0}(x) + v(x) = u_{0}(x) + \tilde v(\xi) 
  \end{align*}
  mit
  \begin{align*}
    (Lu)(x) &=   \underbrace{(Lu_{0})(x)}_{= f(x)} -   \underbrace{(\tilde L\tilde u)(\xi)}_{= 0} = f(x)\\
    u(0) &= \underbrace{u_{0}(0)}_{\neq 0} + \underbrace{\tilde v(0)}_{\text{Korrektur!}} = 0\\
    u(1) &= \underbrace{u_{0}(1)}_{= 0} + \underbrace{\tilde v(\frac 1 \eps)}_{ = 0} = 0. 
  \end{align*}
  Setzen wir nun $\tilde v$ an durch
  \begin{align}
    \tilde v (\xi) = \sum_{n = 0}^{\infty} \tilde b_{n}(\xi) \eps^{n}, 
  \end{align}
  so erhalten wir 
  \begin{align*}
    (\tilde L \tilde v)(\xi) &= \epsilon^{-1}[- \sum_{n = 0}^{\infty} \tilde b_{n}''(\xi) \eps^{n} - \sum_{n = 0}^{\infty} \tilde b_{n}'(\xi) \eps^{n}] = 0\\
    \tilde v(0) = \sum_{n = 0}^{\infty} \tilde b_{n}(0) \eps^{n} \stackrel != - u_{0}(0) = -\sum_{n = 0}^{\infty} \tilde a_{n}(0) \eps^{n}\\
    \tilde v\(\frac 1 \epsilon\) = 0 
  \end{align*}

  $\tilde v$ soll exponentiell fallend sein in $\xi$, statt die Bedingung $\tilde v(\frac 1 \epsilon) = $ zu erfüllen. So erzeugen wir einen (beliebig) kleinen Fehler. Koeffizientenvergleich:
  \begin{align*}
    -\tilde b_{n}''(\xi) - \tilde b_{n}'(\xi) = 0 \implies \quad \tilde b_{n}(\xi) = c_{1} + c_{2}e^{-\xi}\\
    \tilde b_{n}(0) = - a_{n}(0)
  \end{align*}
  $\tilde b_{n}$ exponentiell fallend (Bedingung).
  \begin{align*}
    b_{n}(\xi) &= - a_{n}(0)e^{-\xi}\\
    \tilde v (\xi) &= - \sum_{n = 0}^{\infty}a_{n}(0) e^{-\xi} \eps^{n} \\
    \implies \quad v(x) &= - \sum_{n = 0}^{\infty} a_{n}(0) e^{- \frac x \eps} \eps^{n}\\
    \implies \quad u(x) &= u_{0}(x) + v(x)\\
    w(x) &= u(x) - (u_{0}(x) + v(x))
  \end{align*}

\end{beispiel*}

% !!!!!!!!!!!!!!!!!!!!!!!!!!!!!!!!!!!!!!!!!!!!!!!!!!!!

\paragraph{Asymptotische Entwicklung für Konvektions-Diffusions-Problem}
\label{sec:asympt-entw-fur-konv}




\begin{align*}
  (Lu)(x)=-\eps u''(x)-b(x) u'(x)+c(x)u(x)&=f(x)\quad\mbox{in }(0,1),\\
  u(0)=u(1)&=0.
\end{align*}


Ansatz mit $M>0$ Termen
\[
u(x)=\sum_{n=0}^Mu_n(x)\eps^n+\sum_{n=0}^M\tilde v_n(\xi)\eps^n+R(x)
\]
mit Rest $R$. Hierbei soll gelten
\begin{align*}
  (Lu)(x)= \underbrace{\sum_{n=0}^M(Lu_n)(x)\eps^n}_{=f(x)+\ord{\eps^{M+1}}}
  +\underbrace{\sum_{n=0}^M(\tilde L\tilde v_n)(\xi)\eps^n}_{=0+\ord{\eps^{M}}}+(LR)(x)&=f(x).
\end{align*}
\subsection*{Äußere Entwicklung}
\[
\sum_{n=0}^M(Lu_n)(x)\eps^n
= \sum_{n=0}^M(-\eps^{n+1} u''_n(x)+\eps^n(-b(x)u'_n(x)+c(x)u_n(x)))\stackrel{!}{=}f(x)+\ord{\eps^{M+1}}
\]
\[
\left.
  \begin{aligned}
    \eps^0:\quad -b(x)u'_0(x)+c(x)u_0(x) &= f(x),\\
    u_0(1) &= 0
  \end{aligned}\quad
\right\}\quad\Rightarrow
u_0(x),
\]
\[
\left.
  \begin{aligned}
    \eps^1:\quad -b(x)u'_1(x)+c(x)u_1(x) &= u''_0(x),\\
    u_1(1) &= 0
  \end{aligned}\quad
\right\}\quad\Rightarrow
u_1(x)
\]
allgemein mit $L_0u=-b(x)u'(x)+c(x)u(x)$

\begin{equation}
  \left.
    \begin{aligned}
      (L_0 u_n)(x)&=\begin{cases}
        f(x),&n=0,\\
        u''_{n-1}(x),&n\geq 1,
      \end{cases}\\
      u_n(1)&=0
    \end{aligned}
    \quad
  \right\}
  \quad\Rightarrow
  u_n(x)\label{eq:un}
\end{equation}

\subsection*{Innere Entwicklung}
Skalierung $\xi=\frac{x}{\eps}$ und damit $(\tilde L\tilde v)(\xi)=-\eps^{-1}(\tilde v''(\xi)+\tilde b(\xi)\tilde v'(\xi))+\tilde c(\xi)\tilde v(\xi)$.
Da $\tilde b(\xi)=b(x)=b(\eps\xi)$ von $\eps$ abhängt, wird eine \textsc{Taylor}-Entwicklung von $b(\eps\xi)$ und
$c(\eps\xi)$ um $\xi=0$ genutzt:
\begin{align*}
  \tilde b(\xi)
  &%=b(0)+b'(0)\eps\xi+\dots
  =\sum_{k=0}^{M+1}b_k(\eps\xi)^k,\quad
  b_k=\frac{b^{(k)}(0)}{k!},\,k=0,\dots,M,\quad
  b_{M+1}=\frac{b^{(M+1)}(t(\xi))}{(M+1)!},\\
  \tilde c(\xi)
  &%=c(0)+c'(0)\eps\xi+\dots
  =\sum_{k=0}^{M}c_k(\eps\xi)^k,\quad
  c_k=\frac{c^{(k)}(0)}{k!},\,k=0,\dots,M-1,\quad
  c_M=\frac{c^{(M)}(s(\xi))}{M!}.
\end{align*}
Es folgt
\begin{align*}
  \sum_{n=0}^M(\tilde L\tilde v_n)(\xi)\eps^n
  &= -\sum_{n=0}^M\eps^{n-1}\tilde v_n''(\xi)
  -\sum_{n=0}^M\eps^{n-1}\tilde v_n'(\xi)\sum_{k=0}^{M+1}b_k(\eps\xi)^k
  +\sum_{n=0}^M\eps^{n}\tilde v_n(\xi)\sum_{k=0}^{M}c_k(\eps\xi)^k\\
  &= \sum_{n=0}^M\eps^{n-1}(-\tilde v_n''(\xi)-b_0\tilde v_n'(\xi))
  +\sum_{n=0}^M\sum_{k=0}^{M}\eps^{n+k}(\tilde v_n(\xi)c_k\xi^k-\tilde v_n'(\xi)b_{k+1}\xi^{k+1})\\
  &= \sum_{n=0}^M\eps^{n-1}(-\tilde v_n''(\xi)-b_0\tilde v_n'(\xi))
  +\sum_{\ell=0}^{2M}\eps^\ell\sum_{k=\max\{\ell-M,0\}}^{\min\{\ell,M\}}(c_{\ell-k}\tilde v_k(\xi)\xi^{\ell-k}-b_{\ell-k+1}\tilde v_k'(\xi)\xi^{\ell+1-k})\\
  &\stackrel{!}{=}0+\ord{\eps^M}
\end{align*}
Koeffizientenvergleich bzgl. $\eps$ liefert dann:
\begin{align*}
  &\left.
    \begin{aligned}
      \eps^{-1}:\quad-\tilde v''_0(\xi)-b_0\tilde v'_0(\xi) &= 0,\\
      \tilde v_0(0)&=-u_0(0),\\
      \tilde v_0 \mbox{ exp.}&\mbox{ fallend in $\xi$}
    \end{aligned}\quad
  \right\}&\quad\Rightarrow \tilde v_0(\xi)&=-u_0(0)e^{-b_0\xi},\\
  &\left.
    \begin{aligned}
      \eps^{0}:\quad-\tilde v''_1(\xi)-b_0\tilde v'_1(\xi) &= -c_0\tilde v_0(\xi)+b_1\tilde v_0'(\xi)\xi,\\
      \tilde v_1(0)&=-u_1(0),\\
      \tilde v_1 \mbox{ exp.}&\mbox{ fallend in $\xi$}
    \end{aligned}\quad
  \right\}&\quad\Rightarrow \tilde v_1(\xi)&=q_2(\xi)e^{-b_0\xi},\,q_2\in\Pi_2.
\end{align*}
Allgemein gilt

\begin{equation}\label{eq:vn}
  \left.
    \begin{aligned}
      -\tilde v''_{n}(\xi)-b_0\tilde v'_{n}
      &=-\sum\limits_{k=0}^{n-1}\left( c_{n-1-k}\tilde v_k(\xi)\xi^{n-1-k}-b_{n-k}\tilde v_k'(\xi)\xi^{n-k}\right) ,\\
      \tilde v_{n}(0)&=-u_{n}(0),\\
      \tilde v_{n} \mbox{ exp.}&\mbox{ fallend in $\xi$}
    \end{aligned}\,
  \right\}
  \quad\Rightarrow
  \begin{aligned}
    \tilde v_{n}(\xi) &= q_{2n}(\xi)e^{-b_0\xi},\\
    q_{2n}&\in\Pi_{2n}.
  \end{aligned}
\end{equation}

Damit sind alle Terme der asymptotischen Entwicklung definiert. Bleibt abzuschätzen, wie groß der Fehler $R$ ist.
\subsection*{Fehlerabschätzung}
Hierfür nutzen wir neben Schrankenfunktionen einen kleinen Trick, um eine Ordnung mehr zu erhalten. Sei dazu
\[
R^*(x)
=u(x)-\left( \sum_{n=0}^Mu_n(x)\eps^n+\sum_{n=0}^{M+1}\tilde v_n(\xi)\eps^n\right) 
= R(x)-\tilde v_{M+1}(\xi)\eps^{M+1},
\] 
wobei
\begin{align}
  -\tilde v''_{M+1}(\xi)-b_0\tilde v'_{M+1}
  &=-\sum\limits_{k=0}^{M}\left( c_{M-k}\tilde v_k(\xi)\xi^{M-k}-b_{M+1-k}\tilde v_k'(\xi)\xi^{M+1-k}\right) ,\label{eq:vm1}\\
  \tilde v_{M+1}(0)&=0,\notag\\
  \tilde v_{M+1} \mbox{ exp.}&\mbox{ fallend in $\xi$}\notag
\end{align}
gilt. Dann folgt für $R^*$
\begin{align*}
  % |R(0)| &= \left|
  %   \underbrace{u(0)}_{=0}
  %   -\left( \sum_{n=0}^Mu_n(0)\eps^n
  %     +\sum_{n=0}^M\underbrace{\tilde v_n(0)}_{=-u_0(0)}\eps^n
  %   \right)
  % \right|=0,\\
  |R^*(0)| &= \left|
    \underbrace{u(0)}_{=0}
    -\left( \sum_{n=0}^Mu_n(0)\eps^n
      +\sum_{n=0}^M\underbrace{\tilde v_n(0)}_{=-u_0(0)}\eps^n
      +\underbrace{\tilde v_{M+1}(0)}_{=0}\eps^{M+1}
    \right)
  \right|=0,\\
  % |R(1)| &= \left|
  %   \underbrace{u(1)}_{=0}
  %   -\left( \sum_{n=0}^M\underbrace{u(1)}_{=0}\eps^n
  %     +\sum_{n=0}^M\tilde v_n(1/\eps)\eps^n
  %   \right)
  % \right|
  % =\left|
  %   \sum_{n=0}^M\underbrace{q_{2n}(1/\eps)}_{\leq C\eps^{-2n}}\eps^n\underbrace{e^{-\frac{b_0}{\eps}}}_{\leq C\eps^{M+n}}
  % \right|
  % \leq C_1\eps^M,\\
  |R^*(1)| &= \left|
    \underbrace{u(1)}_{=0}
    -\left( \sum_{n=0}^M\underbrace{u_n(1)}_{=0}\eps^n
      +\sum_{n=0}^{M+1}\tilde v_n(1/\eps)\eps^n
    \right)
  \right|
  =\left|
    \sum_{n=0}^{M+1}\underbrace{q_{2n}(1/\eps)}_{\leq C\eps^{-2n}}\eps^n\underbrace{e^{-\frac{b_0}{\eps}}}_{\leq C\eps^{M+1+n}}
  \right|
  \leq C_1\eps^{M+1},\\
  % |(LR)(x)| &= \left|
  %   \underbrace{(Lu)(x)}_{=f(x)}
  %   -\left( \sum_{n=0}^M(Lu_n)(x)\eps^n
  %     +\sum_{n=0}^M(\tilde L\tilde v_n)(\xi)\eps^n
  %   \right)
  % \right|.
  |(LR^*)(x)| &= \left|
    \underbrace{(Lu)(x)}_{=f(x)}
    -\left( \sum_{n=0}^M(Lu_n)(x)\eps^n
      +\sum_{n=0}^{M+1}(\tilde L\tilde v_n)(\xi)\eps^n
    \right)
  \right|.
\end{align*}
Mit \eqref{eq:un} folgt
\begin{align*}
  \sum_{n=0}^M(Lu_n)(x)\eps^n
  &= -\eps u_0''(x)+f(x)-\eps^2u_1''(x)+\eps u_0''(x)-\eps^3 u_2''(x)+\eps^2 u_1''(x) \mp\dots-\eps^{M+1}u''_M(x)+\eps^M u_{M-1}''(x)\\
  &= f(x)-\eps^{M+1}u_M''(x).
\end{align*}
Mit \eqref{eq:vn} und \eqref{eq:vm1} folgt
\begin{align*}
  \sum_{n=0}^{M+1}(\tilde L\tilde v_n)(\xi)\eps^n
  % &= \sum_{\ell=M}^{2M}\eps^\ell
  % \sum_{k=\ell-M}^{M}
  % \left(
  %   c_{\ell-k}\tilde v_k(\xi)\xi^{\ell-k}-b_{\ell-k+1}\tilde v_k'(\xi)\xi^{\ell+1-k}
  % \right) \\
  % &= \sum_{\ell=M}^{2M}\eps^\ell
  % \sum_{k=\ell-M}^{M}
  % \left(
  %   c_{\ell-k}q_{2k}(\xi)e^{-b_0\xi}\xi^{\ell-k}-b_{\ell-k+1}p_{2k}(\xi)e^{-b_0\xi}\xi^{\ell+1-k}
  % \right),
  &= \sum_{\ell=M+1}^{2M}\eps^\ell
  \sum_{k=\ell-M}^{M}
  \left(
    c_{\ell-k}\tilde v_k(\xi)\xi^{\ell-k}-b_{\ell-k+1}\tilde v_k'(\xi)\xi^{\ell+1-k}
  \right) \\
  &= \sum_{\ell=M+1}^{2M}\eps^\ell
  \sum_{k=\ell-M}^{M}
  \left(
    c_{\ell-k}q_{2k}(\xi)e^{-b_0\xi}\xi^{\ell-k}-b_{\ell-k+1}p_{2k}(\xi)e^{-b_0\xi}\xi^{\ell+1-k}
  \right),
\end{align*}
wobei $p_{2k},\,q_{2k}\in\Pi_{2k}$ sind. Damit gilt aber
\begin{align*}
  |q_{2k}(\xi)\xi^{\ell-k}e^{-b_0\xi}|
  &\leq C\max\{1,\xi^{2k}\}\xi^{\ell-k}e^{-b_0\xi}
  =    C\max\left\{\xi^{\ell-k}e^{-b_0\xi},\xi^{\ell+k}e^{-b_0\xi}\right\}\\
  &\leq C\max\left\{
    \left(\frac{\ell-k}{b_0}\right)^{\ell-k}e^{-(\ell-k)},
    \left(\frac{\ell+k}{b_0}\right)^{\ell+k}e^{-(\ell+k)}
  \right\}
  \leq C(M)
  \intertext{und analog}
  |p_{2k}(\xi)\xi^{\ell+1-k}e^{-b_0\xi}|
  &\leq C\max\left\{
    \left(\frac{\ell+1-k}{b_0}\right)^{\ell+1-k}e^{-(\ell+1-k)},
    \left(\frac{\ell+1+k}{b_0}\right)^{\ell+1+k}e^{-(\ell+1+k)}
  \right\}
  \leq C(M).
\end{align*}
Also ist
\begin{align*}
  \left|\sum_{n=0}^{M+1}(\tilde L\tilde v_n)(\xi)\eps^n\right|
  &\leq  \sum_{\ell=M+1}^{2M}\eps^\ell (\norm{c}{C^M}+\norm{b}{C^{M+1}})C(M)(M+1)
  \leq C\eps^{M+1}
\end{align*}
und wir erhalten insgesamt
\begin{align*}
  |(LR^*)(x)| &= \left|
    f(x)
    -\left( f(x)-\eps^{M+1}u_M''(x)
      +\ord{\eps^{M+1}}
    \right)
  \right|\leq C_2\eps^{M+1}.
\end{align*}
Wählen wir als Schrankenfunktion $s(x)=(\beta+1-x)\frac{C^*}{\beta}\eps^{M+1}$, wobei $b(x)\geq\beta>0$ und $C^*\geq\max\{C_1,C_2\}$
(eine konstante Schrankenfunktion $s$ geht aufgrund von $c(x)\geq 0$ nicht),
so folgt
\begin{align*}
  (Ls)(x) &= \underbrace{b(x)}_{\geq\beta}\frac{C^*}{\beta}\eps^{M+1}+\underbrace{(\beta+1-x)}_{\geq\beta}\frac{C^*}{\beta}\underbrace{c(x)}_{\geq 0}\eps^{M+1}
  \geq C^*\eps^M
  \geq C_2\eps^{M+1}
  \geq |(LR^*)(x)|,\\
  s(0) &= \frac{\beta+1}{\beta}C^*\eps^{M+1}
  \geq 0=|R^*(0)|,\\
  s(1) &= C^*\eps^{M+1}
  \geq C_1\eps^{M+1}
  \geq |R^*(1)|.
\end{align*}
Also ist
\[
|R^*(x)|\leq s(x)\leq \frac{\beta+1}{\beta}C^*\eps^{M+1}=\ord{\eps^{M+1}}.
\]
Mit 
\[
|R(x)-R^*(x)|=|\tilde v_{M+1}(\xi)|\eps^{M+1}\leq C\eps^{M+1}
\]
folgt
\[    
|R(x)|=\ord{\eps^{M+1}}.
\]





%%% Local Variables: 
%%% mode: latex
%%% TeX-master: "vorlesung"
%%% End: 
