
\section{Theorie der Zwei-Punkt Randwertprobleme}
Für diesen Abschnitt bezeichne
\begin{subequations}
  \label{eq:2-1}
\begin{align}
  Lu (x) \coloneqq p_{2}(x) u''(x) + p_{1}(x)u'(x) + p_{0}u(x) =& f(x), \quad x \in (a, b)\label{eq:2-1a}\\
  u(a) = u(b)=& 0\label{eq:2-1b}
\end{align}
\end{subequations}
das Randwertproblem zweiter Ordnung. Hierbei sind $f$ und $p_{i}$ Funktionen aus $C[a, b]$. Insbesondere sei $p_{2}(x) \neq 0 \, \forall x \in [a, b]$. Für Anfangswertprobleme liefern die Sätze von Peano und Picard-Lindelöf die Existenz und Eindeutigkeit. Besitzen \eqref{eq:2-1a} und \eqref{eq:2-1b}  eine eindeutige Lösung? \eqref{eq:2-1a}  kann als System von Gleichungen erster Ordung beschrieben werden:
\begin{align*}
  u' &= w\\
  w' & = - \frac 1 {p_{2}}\( p_{1} w + p_{0}u - f\)
\end{align*}
ist äquivalent zu
\begin{align}\label{eq:2-2}
  \begin{pmatrix}
    u \\ w
  \end{pmatrix}' = 
  \begin{pmatrix}
    0 & 1 \\ - \frac {p_{1}}{p_{2}} & - \frac{p_{0}}{p_{2}}
  \end{pmatrix} 
  \begin{pmatrix}
    u \\ w
  \end{pmatrix}
 +   \begin{pmatrix}
    0 \\ \frac f {p_{2}}
  \end{pmatrix}
\end{align}
\begin{satz} Picard-Lindelöf für AWP
  
Gegeben sei das System $v' = V(v, x)$ mit $v = (v_{1}, \dots, v_{n})^{T}$ und $V = (V_{1}, \dots, V_{n})^{T}$. Dabei seinen $V_{i} \in C[\R^{n} \times [a, b]]$, sodass jedes $V_{i}$ der Lipschitz-Bedingung
\begin{align*}
  \norm{V_{i}(v, x) - V_{i}(v^{*}, x)} \leq K \nnorm{v - v^{*}}, \quad i = 1, \dots, n, \, x \in [a, b]
\end{align*}
für ein $K > 0$ unabhängig von $v, v^{*}$ und $x$ genügt. Dann beitzt das AWP
\begin{align*}
  v' &= V(v, x)\\
  v(a) &= c
\end{align*}
für jeden Vektor $c \in \R^{n}$ eine eindeutige Lösung $v$ mit $v \in C^1[a, b]$. Das Gleiche gilt, falls $v(a) = c$ durch $v(b) = c$ (Endwertproblem) ersetzt wird. 
\end{satz}
\begin{folgerung}\label{fol:2-2}
  Gegeben sei $g \in C[a, b]$. Dann besitzt das AWP
  \begin{align*}
    L u_{1} &= g \quad \text{auf } (a, b)\\
    u_{1}(a) &=0\\
    u_{1}'(a) &=1
  \end{align*}
eine Lösung $u_{1} \in C^{2}[a, b]$. Ebenso besitzt 
  \begin{align*}
    L u_{2} &= g \quad \text{auf } (a, b)\\
    u_{2}(b) &=0\\
    u_{2}'(b) &=1
  \end{align*}
eine Lösung $u_{2} \in C^{2}[a, b]$
\end{folgerung}
\begin{beweis}
  $V$ ist Lipschitzstetig, $w \in C^{1}$, $u' = w \in C^{1} \implies u \in C^{2}[a, b]$. 
\end{beweis}
\begin{satz}\label{thm:2-3}Lösbarkeit und Eindeutigkeit von RWP

Angenommen,
\begin{align}\label{eq:2-3}
    L u &= 0 \quad \text{in } (a, b)\\
    u(a) &= u(b) =0 \notag
\end{align}
besitzt nur die Lösung $u = 0$ (starke Annahme, Erfüllbarkeit wird später gezeigt).   

Dann besitzt
\begin{align}\label{eq:2-4}
  L u &= f \quad \text{in } (a, b)\\
  u(a) &= u(b) =0 \notag
\end{align}
für $f \in C[a, b]$ genau eine Lösung $u \in C^{2}[a, b]$. 
\end{satz}
\begin{beweis}
  Eindeutigkeit: Angenommen, $u_{1}$ und $u_{2}$ sind Lösungen von \eqref{eq:2-4}. Dann löst $w = u_{1} - u_{2}$ das Problem
  \begin{align*}
    Lw = Lu_{1} - Lu_{2} = f-f = 0\\
    w(a) = u_{1}(a) - u_{2}(a) =0 \\
    w(b) = u_{1}(b) - u_{2}(b) =0. 
  \end{align*}
Nach Voraussetzung \eqref{eq:2-3} ist $w = 0$, also $u_{1} = u_{2}$. 

Existenz: Die Lösungen $u_{1}$ und $u_{2}$ der Folgerung \ref{fol:2-2} sind linear unabhängig und lösen $Lu = 0$. Sie besitzen daher eine Wronski-Determinante
\begin{align*}
  W(x) = u_{1}(x) u_{2}' (x) - u_{1}' u_{2}(x) \neq 0 \quad \forall x \in [a, b]. 
\end{align*}
Für $(x, \xi) \in [a, b]^{2}$ sei die Greensche Funktion $G$ zum Differentialoperator $L$ mit
\begin{align*}
  G(x, \xi) =
  \begin{cases}
    \frac{u_{1}(x) u_{2}(\xi)}{p_{2}(\xi) w(\xi)} & a \leq x \leq \xi \leq b\\
    \frac{u_{1}(\xi) u_{2}(x)}{p_{2}(\xi) w(\xi)} & a \leq \xi \leq x \leq b
  \end{cases}
\end{align*}
gegeben. Dann löst
\begin{align*}
  u(x) = \int_{a}^{b} G(x, \xi)f(\xi) d\xi
\end{align*}
das System \eqref{eq:2-4}.
Beweis dessen: 
\begin{enumerate}
\item Erfüllung der Randbedingungen: 
  \begin{align*}
    u(a) = \int_{a}^{b} G(a, \xi)f(\xi) d\xi &= 0\\
    u(b) = \int_{a}^{b} G(b, \xi)f(\xi) d\xi &= 0
  \end{align*}
\item Die Darstellung erfüllt die Differentialgleichung:
  \begin{align*}
    u(x) &= \int_{a}^{x} G(a, \xi)f(\xi) d\xi  + \int_{x}^{b} G(x, \xi)f(\xi) d\xi\\
    u'(x) &= G(x, x) f(x) + \int_{a}^{x} G_{x}(x, \xi)f(\xi) d\xi  - G(x, x)f(x)  + \int_{x}^{b} G_{x}(x, \xi)f(\xi) d\xi \\
    &= \int_{a}^{x} G_{x}(x, \xi)f(\xi) d\xi    + \int_{x}^{b} G_{x}(x, \xi)f(\xi) d\xi\\
    u''(x) &= G_{x}(x, x^{-})f(x^{-}) + \int_{a}^{x} G_{xx}(x, \xi)f(\xi) d\xi - G_{x}(x, x^{+})f(x^{+}) + \int_{x}^{b} G_{xx}(x, \xi)f(\xi) d\xi\\
    &= \int_{a}^{x} G_{xx}(x, \xi)f(\xi) d\xi + \int_{x}^{b} G_{xx}(x, \xi)f(\xi) d\xi + f(x) \frac{u_{1}(x)u_{2}'(x) - u_{1}'(x)u_{2}(x)}{p_{2}(x)W(x)}\\
    &= \int_{a}^{x} G_{xx}(x, \xi)f(\xi) d\xi + \int_{x}^{b} G_{xx}(x, \xi)f(\xi) d\xi + f(x) \frac{1}{p_{2}(x)}\\
\implies \qquad Lu &= p_{2}u'' + p_{1}u' + p_{0}u \\
&= \int_{a}^{x} \(p_{2}G_{xx}(x, \xi) + p_{1}(x)G_{x}(x, \xi) + p_{0} (x) G(x, \xi)\) f(\xi) d\xi + \\
&\int_{x}^{b} \(p_{2}G_{xx}(x, \xi) + p_{1}(x)G_{x}(x, \xi) + p_{0} (x) G(x, \xi)\) f(\xi) d\xi + f(x)\\
&= f(x). 
  \end{align*}
(das Minus kommt von der unteren Grenze, $x^{-}$: Grenzwert von links, da die Integrale im letzten Schritt $0$ sind.) 
\end{enumerate}
\end{beweis}
Betrachte \eqref{eq:2-1} mit $p_{2}(x) = 1$, d.h.
\begin{align*}
  Lu = - u'' + p_{1} u' + p_{0}u. 
\end{align*}
\begin{lemma}
  Es seien $Lu < 0$ und $p_{0} \geq 0$ auf $(a, b)$ sowie $u \in C^{2}[a, b]$. Dann gilt
  \begin{align*}
    u(x) \leq \max \set{0, u(a), u(b)} \quad \text{auf }[a, b]. 
  \end{align*}
\end{lemma}
\begin{beweis}
  Angenommen, $u$ besitzt ein positives Maximum in $c \in (a, b)$, also $u'(c) = 0, u''(c)\leq 0$. Dann
  \begin{align*}
    (L u)(c) = - u''(c) + p_{1}(c)u'(c) + p_{0}(c) u(c)\geq 0. 
  \end{align*}
Widerspruch!
\end{beweis}

% \datum{22. Oktober 2015}
\begin{satz}\label{thm:2-5}(Maximumprinzip)

Es seien $Lu \leq 0$ und $p_{0} \geq 0$ auf $(a, b)$ sowie $u \in C^{2}[a, b]$ und $p_{0}, p_{1}$ beschränkt auf $[a, b]$. Dann gilt
\begin{align*}
      u(x) \leq \max \set{0, u(a), u(b)} \quad \text{auf }[a, b]. 
\end{align*}
Wir sagen dann, dass $L$ ein Maximumprinzip erfüllt. 
\end{satz}
\begin{beweis}
  $u$ besitzt ein positives Maximum in $c \in (a, b)$ und für $d \in (c, b)$ gelte $u(d)< u(c)$. Setze
  \begin{align*}
    z(x) = e^{\alpha(x - c)} - 1
  \end{align*}
mit geeignetem $\alpha > 0$. Dann gilt $z(x) < 0$ für $x \in [a, c)$, $z(c) = 0$, $z(x)>0$ für $x \in (c, b]$. Dann folgt
\begin{align*}
  Lz &= -z'' + p_{1}z' + p_{0}z\\
  &= -\alpha^{2} e^{\alpha(x - c)} + p_{1} \alpha e^{\alpha(x - c)} + p_{0}(e^{\alpha(x - c)} - 1)\\
  &= \(-\alpha^{2} + p_{1}(x) \alpha + p_{0}\)\cdot e^{\alpha(x - c)} - p_{0}(x)\\
  &\leq \(-\alpha^{2} + \bar p_{1}(x) \alpha + \bar p_{0}\)\cdot e^{\alpha(x - c)} - p_{0}(x)
\end{align*}
mit $\bar p_{1} = \max p_{1}(x)$, $\bar p_{0} = \max p_{0}(x)$. Setzen wir nun
\begin{align*}
  \alpha \coloneqq \frac{\bar p_{1} + \sqrt{\bar p_{1}^{2} + 4\cdot(\bar p_{0} + 1)}}2, 
\end{align*}
so folgt $- \alpha^{2} + \bar p_{1}\alpha + \bar p_{0} = -1$ und $\alpha > 0$ ($\bar p_{0} > 0$). Weiter
\begin{align*}
  Lz \leq - e^{\alpha(x-c)} - p_{0}(x) 
\end{align*}
für $a < x\leq b$. Sei nun $\delta \in (0, \frac{u(c) - u(d)}{z(d)})$ und $w(x)\coloneqq u(x) + \delta z(x)$. Dann ist
\begin{align*}
Lw \coloneqq Lu + \delta Lz < 0 \In (a, b)
\end{align*}
($Lu \leq 0$, $\delta> 0$, $Lz<0$), und es folgt
\begin{align*}
  w(x) \leq \max \set{0, w(a), w(d)} 
\end{align*}
für $x \in [a, d]$. Aber $w(c) = u(c) + \delta z(c) = u(c) \geq  \max \set{0, u(a), u(b)} \geq  \max \set{0, u(a)}\geq  \max \set{0, w(a)}$ und $u(c) > u(d) + \delta z(d) = w(d)$. Also ist
\begin{align*}
  w(c) >  \max \set{0, u(a), u(d)}. 
\end{align*}
Das ist ein Widerspruch zu
\begin{align*}
   w(x) \leq \max \set{0, w(a), w(d)}, 
\end{align*}
also war die Annahme, dass $u(d)$ immer positiv ist, falsch. 
\end{beweis}
\begin{lemma}\label{lem:2-6}
  Es seien $p_{0}(x) \geq 0$ auf $(a, b)$ und $p_{0}, p_{1}$ beschränkt auf $[a, b]$. Dann besitzt \eqref{eq:2-3} nur die triviale Lösung $u = 0$. 
\end{lemma}
\begin{beweis}
  Sei $\tilde u$ eine beliebige Lösung von \eqref{eq:2-3}. Mit $L \tilde u = 0$, $\tilde u (a) = 0$ und $\tilde u (b) = 0$  und Satz \ref{thm:2-5} folgt $\tilde u (x)\leq 0$ auf $(a, b)$. Mit $L(- \tilde u) = 0$, $-\tilde u(a) = 0$, $-\tilde u(b) = 0$ und Satz \ref{thm:2-5} folgt $-\tilde u(x) \leq 0$ auf $(a, b)$. 
Also $\tilde u(x) = 0$ auf $(a, b)$. 
\end{beweis}
\begin{lemma}\label{lem:2-7}(Vergleichsprinzip)
  Es seien $p_{0}(x) \geq 0$ auf $(a, b)$ und $p_{0}, p_{1}$ beschränkt auf $[a, b]$. Seien $u_{1}, u_{2} \in C^{2}[a, b]$ mit
  \begin{enumerate}
  \item $Lu_{1}(x) \leq Lu_{2}(x)$ für alle $x \in (a, b)$
  \item $u_{1}(x)\leq u_{2}(x)$ für $x = a$ und $x = b$
  \end{enumerate}
Dann gilt $u_{1}(x)\leq u_{2}(x)$ für alle $x \in [a, b]$. 
\end{lemma}
\begin{beweis}
  Satz \ref{thm:2-5} auf $u_{1} - u_{2}$ anwenden. 
\end{beweis}
\begin{lemma}\label{lem:2-8} (Schrankenfunktionen)
  Seien die Voraussetzungen wie in Lemma \ref{lem:2-7}, aber mit
  \begin{enumerate}
  \item $\norm{Lu_{1}(x)} \leq Lu_{2}(x)$ für alle $x \in (a, b)$ 
  \item $\norm{u_{1}(x)}\leq u_{2}(x)$ für $x = a$ und $x = b$
  \end{enumerate}
Dann gilt $\norm{u_{1}(x)}\leq u_{2}(x)$. 
\end{lemma}
\begin{beweis}
  $u_{1}\leq u_{2}$ und $-u_{1} \leq u_{2}$, daraus folgt $\norm{u_{1}}\leq u_{2}$. 
\end{beweis}
\begin{beispiel}
  \begin{align*}
    Lu_{1} &= -u_{1}'' + au_{1}' + b(x)u_{1} + c\\
    u_{1}(0) &= u_{1}(1) = 0
  \end{align*}
$a, c \geq 0$, $b(x) \geq 0$ beschränkt. 

Ansatz:
\begin{enumerate}
\item $u_{2} = c_{1}x + c_{0}$,
  \begin{itemize}
  \item $Lu_{2} = c_{1} a + b(x)(c_{1}x + c_{2})\geq c$,

$u_{2}(0) = c_{2} \geq 0$, $u_{2}(1) = c_{1} + c_{2} \geq 0$.
\begin{align*}
  0 \leq c_{2} \leq c_{1}x + c_{2}\leq c_{1} + c_{2}
\end{align*}
falls $c_{1}\geq 0$, setze $c_{1} = \frac ca$, $c_{2} = 0$. 
Mit Lemma \ref{lem:2-7}: $u_{1}(x) \leq \frac c a x$ in $[0, 1]$
\item $Lu_{2} = c_{1} a + b(x)(c_{1}x + c_{2})\leq c$, 
$u_{2}(0) = c_{2} \leq 0$, $u_{2}(1) = c_{1} + c_{2} \leq 0$, also $c_{1} = c_{2} = 0$ und $0\leq u_{1}(x) = \frac c a x$ in $[0, a]$ mit $a \geq 0$ beliebig. 
  \end{itemize}
\item Blasenfunktion: $u_{2} = c_{1}x (1 - x)$, es folgt $u_{2}(1) = 0$, $u_{2}(1) = 0$, 
wenn $c_{1} \geq 0$:

$Lu_{2} = 2c_{1} + a_{c1}(1 - 2x) + b(x)(c_{1}x(1-x))\geq c_{1}(2 + a(1-2x))\geq c_{1}(2 - a)\geq c$, 
$c_{1}$ in Abhängigkeit von $a$ bestimmen: 
$a \in (0, 2)$: $c_{1} \geq \frac c {2-a}$. Schranke: $0 \leq u_{1}(x) \leq \frac c{2-a} x(1 - x)$, $a \in (0, 2)$. 
\end{enumerate}
\end{beispiel}

%%% Local Variables: 
%%% mode: latex
%%% TeX-master: "vorlesung"
%%% End: 
