%\datum{05. November 2015}
\section{A priori-Abschätzungen für Ableitungen und Lösungszerlegungen}
\label{sec:priori-absch-fur}

Aus den asymptotischen Entwicklungen kennen wir bereits die Struktur der Lösung $u$ eines Konvektions-Diffusions-Problems. Für die numerische Analysis benötigen wir zusätzlich noch Informationen über die Ableitungen (niedrigerer Ordnung) von $u$
\begin{enumerate}
\item direkt aus der Differentialgleichung, 
\item aus asymptotischen Entwicklungen. 
\end{enumerate}
Es sei
\begin{subequations}
  \label{eq:5-1}
\begin{align}
  (Lu)(x) = - \eps u''(x) - b(x) u'(x) + c(x) u(x) = f(x)& \In (0, 1)\\
u(0) = u(1) = 0 &
\end{align}
\end{subequations}
mit $b(x)\geq \beta > 0$, $c(x) \geq 0$, $0 < \epsilon \ll 1$ und $b, c, f$ glatt genug.
\paragraph{(a) Abschätzungen direkt aus der DGL}
\label{sec:a-absch-direkt}

\begin{satz}\label{thm:5-1} (nach \cite{KT_MC})
  
Für $\epsilon$ klein genug und $i = 1, 2, \dots$ gilt 
\begin{align*}
  \norm{u^{(i)}(x)}\leq C\(1 + \epsilon^{-i} e^{-\frac{\beta x}\epsilon}\)\quad 0\leq x\leq 1. 
\end{align*}
\end{satz}
\begin{beweis} Per Induktion über $i$: 

Induktionsanfang:   Es sei $w(x) = f(x) - c(x)u(x)$ und damit
  \begin{align*}
    - \epsilon u'' - bu' = w. 
   \end{align*}
Mit
\begin{align*}
  B(x) = \int_{0}^{x} b(s) ds
\end{align*}
und dem integrierenden Faktor
\begin{align*}
  - \frac 1 \epsilon e^{\frac {B(s)}\epsilon}
\end{align*}
erhalten wir 
\begin{align*}
   e^{\frac {B(s)}\epsilon}\( u''(s) + \frac 1 \epsilon b(s) u'(s) \) = \(e^{\frac {B(s)}\epsilon} u'(s)\)' = -\frac 1 \epsilon e^{\frac {B(s)}\epsilon} w(s). 
\end{align*}
Integriere:
\begin{align*}
  e^{\frac {B(s)}\epsilon} u'(x) - \underbrace{e^{\frac {B(0)}\epsilon}}_{= 1} u'(0) = - \frac 1 \epsilon \int_{0}^{x} e^{\frac {B(s)}\epsilon} w(s) ds
\end{align*}
Umstellen nach $u'(x)$:
\begin{align*}
  u'(x) = e^{-\frac {B(s)}\epsilon}u'(0) - \frac 1 \epsilon \int_{0}^{x} e^{-\frac {(B(x) - B(s)}\epsilon} w(s) ds,
\end{align*}
bis auf $u'(0)$ ist das beschränkt/bekannt. Integriere erneut: 
\begin{align*}
    \underbrace{u(1)  - u(0)}_{= 0} &= u'(0)\int_{0}^{1}e^{-\frac {B(x)}\epsilon} dx - \frac 1 \epsilon \int_{0}^{1}\int_{0}^{x} e^{-\frac {(B(x) - B(s)}\epsilon} w(s) ds\, dx\\
\implies \quad \norm{u'(0)} &\leq \frac 1 \epsilon \frac{\int_{0}^{1}\int_{0}^{x} e^{-\frac {(B(x) - B(s)}\epsilon} \norm{w(s)} ds\, dx}{\int_{0}^{1}e^{- \frac{B(s)}{\epsilon}}dx}.
\end{align*}
Mit
\begin{align*}
  B(x) - B(s) &= \int_{s}^{x} b(t) dt \geq (x - s)\beta
\end{align*}
und $\norm w \leq C$ nach Lemma \ref{lem:4-1} folgt
\begin{align*}
  \int_{0}^{1}\int_{0}^{x} e^{-\frac {(B(x) - B(s)}\epsilon} \norm{w(s)} ds\, dx &\leq C\cdot \int_{0}^{1}\int_{0}^{x} e^{-\frac {( x-s)\beta}\epsilon} \norm{w(s)} ds\, dx\\ 
&= C \frac \epsilon \beta \cdot \int_{0}^{1} 1 -  e^{-\frac { x\cdot\beta}\epsilon} dx\\ 
&= C \frac \epsilon \beta \cdot \( 1 -  \frac \epsilon \beta \(1 - e^{-\frac \beta \epsilon}\)\) \leq C_{1}\epsilon 
\end{align*}
und
\begin{align*}
  \int_{0}^{1}e^{- \frac{B(x)}{\epsilon}}dx &\geq \int_{0}^{1} e^{-\nnorm b_{L_{\infty}} \frac x \epsilon} dx\\
&= \frac \epsilon {\nnorm b_{L_{\infty}}} \( 1 -  e^{-\frac{\nnorm b_{L_{\infty}}} \epsilon} \) \geq C_{2}\epsilon. 
\end{align*}
Zusammen
\begin{align*}
  \norm{u'(0)} \leq \frac 1 \epsilon \frac{C_{1} \epsilon}{C_{2}\epsilon} \leq \frac {C_{3}}\epsilon
\end{align*}
für genügend kleines $\epsilon$. Also
\begin{align*}
  \norm{u'(x)} &\leq \frac {C_{3}}\epsilon e^{-\frac {B(x)}\epsilon} +  \frac 1 \epsilon \int_{0}^{x} e^{-\frac {(B(x) - B(s)}\epsilon} \norm {w(s)} ds\\
&\leq \frac {C_{3}}\epsilon  e^{-\frac {\beta\cdot x}\epsilon} +  \frac {C_{4}} \epsilon \cdot \int_{0}^{x} e^{-\frac {\beta(x-s)}\epsilon}  ds\\
&= \frac {C_{3}}\epsilon  e^{-\frac {\beta\cdot x}\epsilon} +  \frac {C_{4}} \epsilon \cdot \(1 -  e^{-\frac {\beta x)}\epsilon}\)\\
&= C_{5}\cdot\( 1 + \epsilon^{-1} e^{-\frac {\beta\cdot x}\epsilon}\). 
\end{align*}
Für $i> 1$ nutzen wir Induktion bezüglich $i$. Nehmen wir an, die Aussage des Satzes ist wahr für $i = 1, \dots, k$. Dann leiten wir die Differentialgleichung $k$-mal ab und setzen
\begin{align*}
  v(x) = u^{(k)}(x). 
\end{align*}
Dann folgt, dass wir für $v$ auch eine DGL aufschreiben können:
\begin{align*}
    - \epsilon v''(x) - bv'(x) = \Phi(x) = F(u, u', u'', \dots, u^{(k)}, b, b', \dots, b^{(k)}, c, c', \dots, c^{(k)}, f, f', \dots, f^{(k)}). 
\end{align*}
Mit der Technik von zuvor erhalten wir ($w(s)$ wird durch $\Phi(s)$ ersetzt)
\begin{align*}
  v'(x) = e^{-B(x) \frac 1 \epsilon} v'(0) - \frac 1 \epsilon \int_{0}^{x} \Phi(s) e^{-(B(x)- B(s)) \frac 1 \epsilon}ds.
\end{align*}
Integration und umstellen nach $v'(0)$: 
\begin{align*}
  v'(0) = \frac{v(1)- v(0)}{\int_{0}^{1} e^{-B(x) \frac 1 \epsilon} dx} + \frac 1 \epsilon \frac{  \int_{0}^{1} \int_{0}^{x} \Phi(s) e^{-(B(x)- B(s)) \frac 1 \epsilon}ds dx}{ \int_{0}^{1} e^{B(x) \frac 1 \epsilon} dx}, 
\end{align*}
also
\begin{align*}
  \norm{v(1) - v(0)} \leq C \epsilon^{-k}
\end{align*}
und
\begin{align*}
  \Phi(s) &\leq C\epsilon^{-k}\\
\norm{v'(0)}&\leq C\cdot e^{-(k+1)}\\
\norm{v'(x)} &= \norm{u^{(k+1)}(x)}\leq C\( 1 + \epsilon^{-(k+1)} e^{- \beta \frac x \epsilon}\).
\end{align*}
\end{beweis}
Satz \ref{thm:5-1} liefert uns die Existenz einer Grenzschicht bei $x = 0$ ohne asymptotische Entwicklung. Benötigt werden aber sehr glatte Daten $b, c, f \in C^{i_{\max} - 1}[0, 1]$. 

\paragraph{(b) Aus asymptotischer Entwicklung}
\label{sec:b-aus-asymptotischer}

Auch hieraus können Informationen über Ableitungen gewonnen werden. Siehe z.B. \cite{DR_ZAA}.


Zu den Zerlegungen:
\begin{satz}\label{thm:5-2}
  Sei $u$ die Lösung von \eqref{eq:5-1} und $q > 0$ eine natürliche Zahl. Dann ist $u = S + E$, wobei für $0 \leq j\leq q$
  \begin{align*}
    \nnorm{S^{(j)}}_{L_{\infty}} &\leq C\\
    \norm{E^{(j)}(x)}&\leq C \cdot\epsilon^{-j} e^{-\beta \frac x \epsilon} \quad 0\leq x\leq 1
  \end{align*}
(Smooth, glatter Anteil und Grenzschichtanteil). 
\end{satz}
\begin{beweis} Linß
  
  Sei $x^{*} = \frac {q\cdot \epsilon}{\beta} \ln \(\frac{1}{\epsilon}\)$ für kleines $\epsilon$. Dann ist $x^{*} \in (0, 1)$. 

Wie kommt man drauf?
\begin{align*}
  E(x^{*}) = e^{-{\beta q \epsilon \ln (1/\epsilon)} {\beta\epsilon}} = \epsilon^{q}. 
\end{align*}
Außerdem sei $S(x) = u(x)$ für $x^{*}\leq x \leq1$. Dann folgt mit Satz \ref{thm:5-1}
\begin{align*}
  \norm{S^{(i)}(x)}&\leq C \cdot \( 1 + \epsilon^{-j} e^{ - \frac {\beta x}\epsilon}\) \\
& \leq C\cdot \( 1 + \epsilon^{q-j}\) \leq C. 
\end{align*}
Mittels Taylor-Entwicklung um $x = x^{*}$ kann $S$ auf $[0, 1]$ erweitert werden mit
\begin{align*}
  \norm{S^{(i)}(x)}\leq C
\end{align*}
für $0\leq x\leq 1$ und $0 \leq i\leq q$. Setzen wir $E = u-S$, so folgt $E(x) = 0$ für $x \in [x^{*}, 1]$ und in $[0, x^{*})$ gilt mit Satz \ref{thm:5-1}
\begin{align*}
  \norm{E^{(q)}(x)}&\leq \norm{u^{(q)}(x)} + \norm{S^{(q)}(x)} \leq C\( 1 + \epsilon^{-q} e^{- \beta \frac x \epsilon}\) \leq C \epsilon^{-q} e^{- \beta \frac x \epsilon}. 
\end{align*}
Per Induktion und Integration von $E^{(j)}$ für $j = q, q-1, \dots, 1$ folgt
\begin{align*}
  \norm{E^{(j-1)}(x)}&= \norm{ \int_{x^{*}}^{x} E^{(j)}(t) dt} \leq C \int_{x^{*}}^{x} \epsilon^{-j} e^{- \beta \frac t \epsilon} dt \leq C \epsilon^{-(j-1)} e^{- \beta \frac x \epsilon}. 
\end{align*}
\end{beweis}
\begin{bemerkung*}
  Satz \ref{thm:5-1} $\iff$ Satz \ref{thm:5-2}
\end{bemerkung*}
Satz \ref{thm:5-2} kann noch verschärft werden.
\begin{lemma}\label{lem:5-3} Shishkin-Zerlegung
  
Es sei $u$ die Lösung von \eqref{eq:5-1} und $q >0$ eine natürliche Zahl. Dann ist
\begin{align*}
  u = S^{*} + E^{*}
\end{align*}
und mit $0 \leq j\leq q$ haben wir die Abschätzungen
\begin{align*}
   \nnorm{S^{*(j)}}_{L_{\infty}} &\leq C\\
\norm{E^{*(j)} (x)} &\leq C\cdot \epsilon^{-j}\cdot e^{- \beta \frac x \epsilon}, \quad 0 \leq x \leq 1\\
LE^{*} &= 0\\
LS^{*} &= f. 
\end{align*}
\end{lemma}
\begin{beweisidee}
  \begin{align*}
      S^{*} = u_{0} + \epsilon u_{1} + \epsilon^{2} u_{2} + \dots + \epsilon^{M} u_{M} + \epsilon^{M+1}u^{*}_{M+1}\\
      E^{*} = v_{0} + \epsilon v_{1} + \epsilon^{2} v_{2} + \dots + \epsilon^{M} v_{M} + \epsilon^{M+1}v^{*}_{M+1}\\
  \end{align*}
mit
\begin{align*}
  Lu^{*}_{M+1} &= - u_{M}''\\
  u_{M+1}^{*}(0) &= u^{*}_{M+1}(1) = 0\\
  Lv^{*}_{M+1} &= - \epsilon^{-(M+1)} L(v_{0} + \epsilon v_{1} + \dots + \epsilon^{M}v_{M})\\
  v_{M+1}^{*}(0) &= - \sum_{k = 0}^{M} v_{k}(0) \epsilon^{k}\\
  v_{M+1}^{*}(1) &= 0
\end{align*}
\end{beweisidee}

%%% Local Variables: 
%%% mode: latex
%%% TeX-master: "vorlesung"
%%% End: 
